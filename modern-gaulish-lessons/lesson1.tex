\section{Menghavan 1: Br\'{e}thr\'{e} \textendash\ Aman Dhathach:  Gweran\'{u}\'{e} Donach}
(\textit{Lesson 1: Verbs \textendash\ Present Tense: Personal Pronouns})\\

In the first lesson you will learn how to put a verb in the present tense and how to use it with a personal pronoun.

\subsection{Gwepchoprith: Verbs in the present tense}

\subsubsection{Root form}

Each verb has a basic form or \textit{root form}, known as a \textit{verbal noun}. It has the same function as the infinitive in English.

Verbal nouns can end in a consonant, in -i, -a, -e, and in just one case in -\'{o}, never in -u.\\\\
Examples:
\begin{table}[H]
\centering
\begin{tabular}{cc}
  \toprule
  \textbf{Gal\'{a}thach} & \textbf{English}\\
  \cmidrule(lr){1-1}\cmidrule{2-2}
  \'{a}pis & to see\\
  men & to think\\
  gwel & to want\\
  gar & to call\\
  carni & to build\\
  argha & to shine\\
  delghe & to hold\\
  \'{a}v\'{o} & to do, to make\\
  berwi & to boil\\
  gn\'{\i} & to know\\
  \bottomrule
\end{tabular}
\label{examples_verbal_nouns}
\end{table}

\subsubsection{Present tense}

To form the present tense of these verbs, an -a is added to the verbal noun in the following ways. Note that vowels in modern Gaulish can be either long or short. Vowel length changes with emphasis. The emphasis is always on the second last syllable. When words are extended the emphasis shifts accordingly.\\

Verbs on a consonant:
\begin{table}[H]
\begin{tabular}{l}
  \'{a}pis $\rightarrow$ ap\'{\i}sa\\
  men $\rightarrow$ m\'{e}na\\
  gwel $\rightarrow$ gw\'{e}la\\
  gar $\rightarrow$ g\'{a}ra\\
\end{tabular}
\label{examples_verbs_oac}
\end{table}

Verbs on -i:
\begin{table}[H]
\begin{tabular}{ll}
  carni $\rightarrow$ carna
\end{tabular}
\label{examples_verbs_on_i}
\end{table}

Verbs on -a:
\begin{table}[H]
\begin{tabular}{ll}
  argha $\rightarrow$ argha $\Rightarrow$ nothing changes
\end{tabular}
\label{examples_verbs_on_a}
\end{table}

Verbs on -e:
\begin{table}[H]
\begin{tabular}{ll}
  delghe $\rightarrow$ delgha
\end{tabular}
\label{examples_verbs_on_e}
\end{table}

Verbs on -\'{o} exchange the \'{o} for an -a:
\begin{table}[H]
\begin{tabular}{ll}
  \'{a}v\'{o} $\rightarrow$ \'{a}va
\end{tabular}
\label{examples_verbs_on_oo}
\end{table}

Verbs on -wi retain the final -i:
\begin{table}[H]
\begin{tabular}{ll}
  berwi $\rightarrow$ berw\'{\i}a
\end{tabular}
\label{examples_verbs_on_wi}
\end{table}

Verbs on -i, where -i is the only vowel, retain the final -i:
\begin{table}[H]
\begin{tabular}{ll}
  gn\'{\i} $\rightarrow$ gn\'{\i}a
\end{tabular}
\label{examples_verbs_on_i_only_vowel}
\end{table}

\newpage
\subsection{Exercises: Present tense}
\subsubsection{Put verbs in the present tense}
Put the following verbs into the present tense:
\begin{table}[H]
\centering
\begin{tabular}{|c|c|M{10.0cm}|}
  \toprule
  \textbf{Verb} & \textbf{Present tense}\\
  \toprule
  prin & to buy & \\
  \midrule
  ber & to carry & \\
  \midrule
  gal & to be able to do & \\
  \midrule
  br\'{\i}s & to break & \\
  \midrule
  \'{\i}vi & to drink & \\
  \midrule
  c\'{a}ra & to love & \\
  \midrule
  cinge & to wage war & \\
  \midrule
  \'{a}v\'{o} & to do, to make & \\
  \midrule
  camwi & to bend, to curve & \\
  \midrule
  l\'{\i} & to lie down & \\
  \bottomrule
\end{tabular}
\caption{Exercise: present tense}
\label{exercise_present_tense}
\end{table}

\newpage
\subsubsection{Solution}
\begin{table}[H]
\centering
\rotatebox{180}{%
  \begin{tabular}{|c|c|>{\itshape}c|}
    \toprule
    \textbf{Gal\'{a}thach} & \textbf{English} & \textbf{Answer (Gal\'{a}thach)}\\
    \toprule
    prin & to buy & prina\\
    \midrule
    ber & to carry & bera\\
    \midrule
    gal & to be able to do & gala\\
    \midrule
    br\'{\i}s & to break & br\'{\i}sa\\
    \midrule
    \'{\i}vi & to drink & \'{\i}va\\
    \midrule
    c\'{a}ra & to love & c\'{a}ra\\
    \midrule
    cinge & to wage war & cinga\\
    \midrule
    \'{a}v\'{o} & to do, to make & \'{a}va\\
    \midrule
    camwi & to bend, to curve & camwia\\
    \midrule
    l\'{\i} & to lie down & l\'{\i}a\\
    \bottomrule
  \end{tabular}
}
\label{solution_present_tense}
\caption{Solution: present tense}
\end{table}
\newpage

\subsection{Gwepchoprith: Personal pronouns}

\subsubsection{Personal pronouns as subject}

The personal pronouns when used as subject are as follows:
\begin{table}[H]
\centering
\begin{tabular}{cc}
  \toprule
  \textbf{Gal\'{a}thach} & \textbf{English}\\
  \cmidrule(lr){1-1}\cmidrule{2-2}
  mi & I \\
  ti & you\\
  \'{e} & he\\
  \'{\i} & she\\
  \'{\i} & it\\
  ni & we\\
  s\'{u} & you (pl.)\\
  s\'{\i} & they\\
  \bottomrule
\end{tabular}
\caption{Personal pronouns, when used as subject}
\label{personal_pronouns_as_subject}
\end{table}

There is no difference in the third pronoun plural between the masculine and the feminine form.\\

In modern Gaulish the personal pronoun follows the verb it accompanies:
\begin{table}[H]
\centering
\begin{tabular}{cc}
  \toprule
  \textbf{Gal\'{a}thach} & \textbf{English}\\
  \cmidrule(lr){1-1}\cmidrule{2-2}
  ap\'{\i}sa mi & I see\\
  m\'{e}na ti & you think\\
  gw\'{e}la \'{e} & he wants\\
  g\'{a}ra \'{\i} & she calls\\
  carna ni & we build\\
  argha s\'{u} & you shine\\
  delgha s\'{\i} & they hold\\
  \'{a}va \'{\i} & it  does, it makes\\
  berw\'{\i}a \'{\i} & it boils\\
  gn\'{\i}a \'{\i} & it knows\\
  \bottomrule
\end{tabular}
\label{examples_personal_pronoun}
\end{table}

\newpage
\subsection{Exercises: Pronouns}

\subsubsection{Make phrases}

Make the following phrases:\\
\begin{table}[H]
\centering
\begin{tabular}{|l|M{10.0cm}|}
  \toprule
  \textbf{Phrase (English)} & \textbf{Answer (Gal\'{a}thach)}\\
  \toprule
  I buy & \\
  \midrule
  you carry & \\
  \midrule
  he can & \\
  \midrule
  she breaks & \\
  \midrule
  we drink & \\
  \midrule
  you (pl.) love & \\
  \midrule
  they wage war & \\
  \midrule
  it does, it makes & \\
  \midrule
  it bends & \\
  \midrule
  it lies down & \\
  \bottomrule
\end{tabular}
\label{exercise_personal_pronouns}
\caption{Exercise: personal pronouns}
\end{table}

\newpage
\subsubsection{Solution}
\begin{table}[H]
\centering
\rotatebox{180}{%
  \begin{tabular}{|l|>{\itshape}c|}
    \toprule
    \textbf{Phrase (English)} & \textbf{Answer (Gal\'{a}thach)}\\
    \toprule
    I buy & prina mi\\
    \midrule
    you carry & bera ti\\
    \midrule
    he can & gala \'{e}\\
    \midrule
    she breaks & br\'{\i}sa \'{\i}\\
    \midrule
    we drink & \'{\i}va ni\\
    \midrule
    you (pl.) love & c\'{a}ra s\'{u}\\
    \midrule
    they wage war & cinga s\'{\i}\\
    \midrule
    it does, it makes & \'{a}va \'{\i}\\
    \midrule
    it bends & camwia \'{\i}\\
    \midrule
    it lies down & l\'{\i}a \'{\i}\\
    \bottomrule
  \end{tabular}
}
\label{solution_phrases}
\caption{Solution: phrases}
\end{table}
