\section{Menghavan 16: In Goghn\'{i}t hAmanach}
(\textit{Lesson 16: The Temporal System})\\

In the sixteenth lesson, you will learn how the temporal system works in Gal\'{a}thach.

\subsection{Gwepchoprith: Conversation}
\subsubsection{Conversation}

% TODO: Add bubbly conversation

\subsubsection{Colav\'{a}ru \textendash\ Tr\'{e}lav\'{a}ru}
(Conversation \textendash\ Translation)

A man, Arthu, is talking to a woman, Melina.

Arthu = The Bear < Artos
Melina = The Honey < Melina

A: \'{E}i, esi didh\'{u}rach riem aman. Aman a chw\'{e}ri canith, in chwerpenach.
M: \'{A}, esi aman m\'{o} chwerpenachu. 
A: A hesi \'{\i} in vithw\'{\i}r?
M: Esi \'{\i}. D\'{a}ma m\'{o} hemp\'{a} adhith in gaman o cerdha \'{\i} ...
A: Suvis.
M: Diantha aman can shech\'{o}n\'{e}, och \'{a}va s\'{\i} min\'{u}th\'{e}, och \'{a}va s\'{\i} \'{o}r\'{e}.
A: Esi s\'{e} certh.
M: N\'{e} ghavis\'{\i} mini sin \'{e}ithr sim \'{o}r.

A: I\'{a}nu sath.
M: Esi d\'{a}dhech \'{o}r en dh\'{\i}, ach d\'{a}dhech \'{o}r en nh\'{o}ith, a han\'{e}tham a’n n\'{o}ith’sam\'{a}l\'{e}.
A: In gerth.
M: Esi er\'{e}dhl gwochon-pethr \'{o}r anw\'{\i}thu “lath\'{\i}u”.
A: Esi \'{\i}.
M: Enelgha lath\'{\i}u b\'{a}r\'{e}i, methin, m\'{e}dhi, \'{o}sw\'{e}dhi, nesn\'{o}ith, ach m\'{e}dhn\'{o}ith.

A: I\'{a}nu sath. Arw\'{e}ra adhim in rhanalch in n\'{o}ith. Ach ti?
M: Penarw\'{e}ra adhim in dh\'{\i}.
A: Ap\'{\i}sa mi. P\'{e} a havos\'{\i} ti av\'{a}r\'{e}i?
M: In peth samal o r\'{e} h\'{a}v\'{o} mi d\'{\i}es ach cin dh\'{\i}es.
A: Ach p\'{e} a v\'{u} s\'{e}?
M: B\'{\i} mi en dhiluthri m\'{o} shuchnusan u b\'{e}n\'{e} ser\'{\i}thu.
A: Echan dhin\'{e}an. Ach \'{o}s hav\'{a}r\'{e}i?
M: In shamal.
A: Certh.
M: N\'{u}, \'{a}va s\'{e}ith n\'{o}ith s\'{e}ithn\'{o}ith.
A: In vithw\'{\i}r?
M: Ach esi in d\'{a} d\'{\i} \'{o}s anw\'{\i}thu in pens\'{e}ithn\'{o}ith.
A: \'{a}, tr\'{e}v\'{\i}u lav\'{a}ra ti am\'{\i} ... p\'{e} a havos\'{\i} ti in pens\'{e}ithn\'{o}ith-sin?
M: B\'{\i} mi en lhauni m\'{o} chwolth en dh\'{u}an’\'{e}p ach duvr’wargh.
A: Certh.
M: Av\'{o}thu mi chi sin\'{\i} ach avos\'{\i} mi ch\'{\i} ath\'{e} sin\'{o}ith.
A: Sulichach.
M: Duch, esi pethr s\'{e}ithn\'{o}ith en on m\'{\i}s.
A: Esi.
M: Ach esi tr\'{\i} m\'{\i}s en on sonching.
A: A hesi?
M: Esi. Ach esi pethr sonching en on bledhn: gw\'{\i}son, sam, meth, g\'{\i}am.
A: Swausa \'{\i} certh \'{e}th gw\'{e} an\'{e}th.
M: Ach litha ni gwer\'{a}n\'{e} chwerpenach tar in bledhn. Com\'{\i}u penvledhn\'{e}.
A: Esi s\'{e} certh! Ponch a hesi t\'{o} benvl\'{e}dhn ach p\'{e} a havos\'{\i} ti?
M: B\'{u} \'{\i} in s\'{e}ithn\'{o}ith \'{o}s.
A: \'{a}.
M: Aun\'{e} nep ni-esi lith\'{a}n\'{e} trin\'{o}ith ach dechn\'{o}ith, com\'{\i}u ri sh\'{a}n’su\'{e}l\'{e} ach samal’n\'{o}ith\'{e}.
A: In gerth! P\'{e} a havos\'{\i} ti ton?
M: B\'{\i} mi en shuling dherthol \'{e}r in ten ...

A: \'{A}! N\'{u} lav\'{a}ra ni ...
M: ... can m\'{o} gar\'{a}n\'{e} geneth ...
A: M\'{o}i!
M: ... en lhoscr\'{\i}thi cl\'{a}dh\'{e} rh\'{e} h\'{a}chw\'{a}r ...
A: A h\'{a}va s\'{u}?
M: ... ach suvi\'{o}na ni sal\'{a}th\'{e} en lhithalach.
A: Ap\'{\i}sa mi.
M: Ton, esi in bl\'{e}dhn\'{e} suchnus\'{\i}thu en sh\'{e}thl\'{e} ... 
A: P\'{e} a hesi s\'{\i}?
M: Esi s\'{e} e gh\'{e}nu ith\'{\i} a gh\'{e}nu t\'{o} wapath cin.

A: \'{a}, n\'{u}, gwer chw\'{o}chatha gnath\'{a}l\'{e} ...
M: Ach ton esi in bl\'{e}dhn\'{e} cans\'{o}ith\'{u} en hai\'{u}\'{e}.
A: P\'{e} a hesi \'{a}iu?
M: Esi \'{\i} gwochon-dech bl\'{e}dhn, gw\'{e} swech \'{o}chan amanar.
A: Certh.
M: N\'{u}, \'{a}iu ...
A: \'{a}iu?
M: ... esi s\'{e} in gerth in \'{e}r\'{e}dhl o rinchas\'{\i} ti an\'{e}li aven gals\'{\i} ti \'{a}dha to dhalam gwerim.

A: Hey, I’m interested in time. Spending time with you, specifically. 
M: Ah yes, time is my specialty.
A: Is it really?
M: It is. Let me tell you how it works ...

A: All right.
M: Time starts with seconds, which make up minutes, which make up hours.
A: That’s right.
M: Explaining this will only take half an hour.
A: Fair enough.
M: There are 12 hours in a day, and 12 hours in a night, at least at the equinoxes.
A: Exactly.
M: A whole 24 hour period is called a “day period”.
A: It is.
M: A day period consists of a dawn, a morning, a midday, an afternoon, an evening, and a midnight.
A: Fair enough. I particularly like the night. What about you?
M: I prefer the day.
A: I see. What are you doing tomorrow?
M: The same thing I did yesterday and the day before yesterday.
A: And what was that?
M: I will be cleaning my collection of severed heads.
A: Of course. And the day after tomorrow?
M: The same.
A: Right.
M: Now, seven days make up a week.
A: Really?
M: And the last two days are called the weekend.
A: Ah yes, since you mention it ... what are you doing this weekend?
M: I’ll be washing my hair in horsepiss and lime.
A: Right.
M: I did it today and I’ll do it again tonight.

A: Charming.
M: So, there are four weeks in a month.
A: There are.
M: And there are three months in a season.
A: Is there?
M: There is. And there are four seasons in one year: spring, summer, autumn, winter.
A: Sounds about right.
M: And we celebrate special events throughout the year. Such as birthdays.
A: That’s right! When’s your birthday and what will you be doing?
M: It was last week.
A: Ah.
M: At times we have three-night and ten-night celebrations, such as at the solstices and equinoxes.
A: Exactly! What will you be doing then?
M: I will be dancing around the fire in the nude ...
A: Ah! Now we’re talking ...
M: ... with all my girlfriends ...
A: Great!
M: ... wielding very sharp swords ...
A: You do?
M: ... and we ritually slice up sausages.
A: I see.
M: Then, the years are gathered in generations ...
A: What are they?
M: That’s from your birth to the birth of your first child.
A: Ah, now, on the subject of babies ...
M: And then the years are counted in ages.

A: What’s an age?
M: It is thirty years, or six calendar completions.
A: Right.
M: Now, an age ...
A: Yes?
M: ... that’s exactly how long you’re going to have to wait before you ever get your hands on me.

In the conversation above two people use all the terms relating to time in Gal\'{a}thach. They are listed here. Some of the words are modern loans.

sechon: second 
minuth: minute 
pimdhech minuth: fifteen minutes, quarter of an hour 
sim \'{o}r: half hour (< sim “half”, attested)
\'{o}r: hour 

b\'{a}r\'{e}i: dawn (< proto-Celtic *ba:re:gom, loss of –om \& –eg > \'{e}i cf. Lambert 2003, p. 43)
methin: morning (< Latin “matina”, cf. Br. mintin, C. metten, Ir. maidin)
m\'{e}dhi: midday (< medh “middle” + d\'{\i})
\'{o}sw\'{e}dhi: afternoon (< \'{o}s “after” + medh + d\'{\i})
nesn\'{o}ith: evening (< nes “near, close to” + n\'{o}ith
medhn\'{o}ith: midnight (< medh + noith)

d\'{\i}: day
sin\'{\i}: today
n\'{o}ith: night
sin\'{o}ith: tonight (by analogy with sin\'{\i} < sindiu > sindenocta > sin\'{o}ith)
lath\'{\i}u: period of 24 hours, daytime, day and night

av\'{a}r\'{e}i: tomorrow (< a “to, at” + b\'{a}r\'{e}i)
\'{o}s hav\'{a}r\'{e}i: after tomorrow
d\'{\i}es: yesterday (< proto-Celtic *gdijes)
cin dh\'{\i}es: before yesterday

pens\'{e}ithn\'{o}ith: weekend (< pen “head” + s\'{e}thn\'{o}ith)
s\'{e}ithn\'{o}ith: week (“seven-night”, by analogy with e.g. trinoctia)
m\'{\i}s: month

trin\'{o}ith: three-night feast
dechn\'{o}ith: ten-night feast

sam: summer
meth: autumn (“harvest” < *met- “to harvest)
g\'{\i}am: winter
gwison: spring

sonching: season (period)
bl\'{e}dhn: year
penvl\'{e}dhn: anniversary, birthday

s\'{e}thl: generation
\'{a}iu: age
aman: time

\subsection{Exercises: The temporal system}

\subsubsection{Translate}

Use the words given above and the vocabulary learned so far to construct the following sentences.

Notes: 

* To say how old a person is the verb “to have” is used > I am twenty years old = I have twenty years. 

* To indicate the time the number of hours is indicated > it is five o’clock = it is five hours.

Can you run a hundred metres in ten seconds?
No, it would take me five minutes.
Will you be here after fifteen minutes?
No, I will be gone for half an hour.

I will run for an hour.
She swims in the ocean at dawn.
He slept until ten o’clock of the morning.
We could eat at midday.
They like to sleep in the afternoon
Will you (pl.) tell stories this evening?
You danced until midnight.

Today is a beautiful day.
Tonight will also be a beautiful night.
He has not slept for a period of 24 hours.

Tomorrow you will be very tired.
After tomorrow you have to go to work.
She went to see her mother yesterday.
But her mother died the day before yesterday.

This weekend we will go to walk through the mountains.
Next week we will swim in the river.
This is the first month of summer.

We will dance in the nude for the three-night feast.
We will eat and drink without sleeping for the ten-night feast.

They worked hard in the autumn.
Last winter their feet froze.
Spring always makes her nose run.

It is a difficult season for her.
What year is this again?
Happy birthday to you, may you have many returns.
How old is that girl?
You are not allowed to ask, she will cut your head off.

The old people like to complain about the new generation.
They lived in an age of freedom.
Now it is a time of slavery.

\newpage
\subsubsection{Solution}

Can you run a hundred metres in ten seconds? > A gh\'{a}la ti r\'{e}thi can methr en dhech sechon?
No, it would take me five minutes. > N\'{e} gh\'{a}la mi, r\'{e} ghavis\'{\i} \'{\i} mi pimp minuth.
Will you be here after fifteen minutes? > A v\'{\i} ti insin \'{o}s pimdhech minuth?
No, I will be gone for half an hour. > N\'{e} v\'{\i} mi, b\'{\i} mi \'{a}ithu ri shim \'{o}r.

I will run for one hour. > Rethis\'{\i} mi ri on \'{o}r.
She swims in the ocean at dawn. > Sn\'{a} \'{\i} en in m\'{o}rw\'{a}r a v\'{a}r\'{e}i.
He slept until ten o’clock of the morning. > R\'{e} sh\'{o}ni \'{e} aven dech \'{o}r in methin.
We could eat at midday. > R\'{e} ghals\'{\i} ni depri a w\'{e}dhi.
They like to sleep in the afternoon. > Arw\'{e}ra adh\'{\i}s s\'{o}ni en in \'{o}sw\'{e}dhi.
Will you (pl.) tell stories this evening? > A hempas\'{\i} s\'{u} sp\'{a}thl\'{e} in nesn\'{o}ith-sin?
You danced until midnight. > R\'{e} shuling ti aven medhn\'{o}ith.

Today is a beautiful day. > Esi sin\'{\i} d\'{\i} dech.
Tonight will also be a fantastic night. > B\'{\i} sin\'{o}ith n\'{o}ith swapis\'{o}ich c\'{o}\'{e}th.
I have not slept for a period of 24 hours. > N\'{e} shon\'{\i}thu mi ri lhath\'{\i}u.

Tomorrow you will be very tired. > B\'{\i} ti r\'{e} lhisc av\'{a}r\'{e}i.
After tomorrow you have to go to work. > \'{o}s hav\'{a}r\'{e}i rincha ti \'{a}i a gerdhi.
She went to see her mother yesterday. > R\'{e} h\'{a}i \'{\i} a h\'{a}pis \'{o} m\'{a}thir d\'{\i}es.
But her mother died the day before yesterday. > \'{e}ithr r\'{e} warwi \'{o} m\'{a}thir cin dh\'{\i}es.

This weekend we will go to walk through the mountains. > In pens\'{e}ithn\'{o}ith-sin \'{a}is\'{\i} ni a gama 
                                                                                               tar in vr\'{\i}\'{e}.
Next week we will swim in the river. > In s\'{e}itn\'{o}ith conesam sn\'{a}s\'{\i} ni en in avon.
This is the first month of summer. > Esi sin m\'{\i}s gin in sham.

We will dance in the nude for the three-night feast. > Sulings\'{\i} ni derthol ri drin\'{o}ith.
We will eat and drink without sleeping for the ten-night feast. > Depris\'{\i} ach ivis\'{\i} ni echan 
                                                                                                       sh\'{o}ni ri dhechn\'{o}ith.

They worked hard in the autumn. > R\'{e} gerdhi s\'{\i} en galeth en chwison.
Last winter their feet froze. > In gh\'{\i}am h\'{o}sim r\'{e} hoghri s\'{o} drai\'{e}th\'{e}.
Spring always makes her nose run. > \'{a}va in gwison r\'{e}thi \'{o} trughn aman hol.

It is a difficult season for her. > Esi \'{\i} sonching guth rich\'{\i}.
What year is this again? > P\'{e} vl\'{e}dhn a hesi sin ath\'{e}?
Happy birthday to you, may you have many returns. > Penvl\'{e}dhn l\'{a}en adhith, o ti-v\'{\i}  
                                                                                        ath\'{e}chwerth\'{a}n\'{e} lh\'{a}en h\'{e}lu.
How old is that girl? > P\'{e} h\'{a}iu a h\'{\i}-esi in gheneth-s\'{e}?
You are not allowed to ask, she will cut your head off. > N\'{e} shudhamor p\'{e}tha ich\'{\i}, b\'{e}s\'{\i} \'{\i} t\'{o} 
                                                                                           ben.

The old people like to complain about the new generation. > Arw\'{e}ra \'{\i} a’n don\'{e} sen cothr\'{o}ia 
                                                                                                  am in s\'{e}thl n\'{o}i.
They lived in an age of freedom. > R\'{e} v\'{\i}thi s\'{\i} en h\'{a}iu r\'{\i}as.
Now it is a time of slavery. > N\'{u} esi \'{\i} aman caithan.
