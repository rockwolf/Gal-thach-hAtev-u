\section{Menghavan 16: In Goghn\'{i}t hAmanach}
(\textit{Lesson 16: The Temporal System})\\

In the sixteenth lesson, you will learn how the temporal system works in Gal\'{a}thach.

\subsection{Gwepchoprith: Conversation}
\subsubsection{Conversation}

%\tcbox[colback=red!10!green!10, colframe=green!20!black!80]
\begin{table}[H]
\centering
    \begin{tabular}{M{3.0cm}M{10.0cm}M{3.0cm}}
    \cellcolor{lightgreen} & \cellcolor{lightgreen} & \cellcolor{lightgreen}\\
    \cellcolor{lightgreen}\textcolor{darkgreen}{\textbf{Arthu}} & \cellcolor{lightgreen} & \cellcolor{lightgreen}\textcolor{darkgreen}{\textbf{Melina}}\\
    \cellcolor{lightgreen} & \cellcolor{lightgreen} & \cellcolor{lightgreen}\\
    \cellcolor{lightgreen} & \cellcolor{lightgreen} & \cellcolor{lightgreen}\\
    \cellcolor{lightgreen} & \cellcolor{lightgreen} & \cellcolor{lightgreen}\\
    \cellcolor{lightgreen} & \cellcolor{lightgreen} & \cellcolor{lightgreen}\\
    \cellcolor{lightgreen} & \cellcolor{lightgreen} & \cellcolor{lightgreen}\\
    \cellcolor{lightgreen} & \cellcolor{lightgreen} & \cellcolor{lightgreen}\\
%    \cellcolor{lightgreen}\multirow{-7}{*}{\includegraphics[height=4.0cm]{img/menghavan11_1}} & \cellcolor{lightgreen} & \cellcolor{lightgreen}\multirow{-7}{*}{\includegraphics[height=4.0cm]{img/menghavan11_2}}\\
    \cellcolor{lightgreen} & \cellcolor{lightgreen} & \cellcolor{lightgreen}
    \end{tabular}
\end{table}

\begingroup
\fontsize{10pt}{12pt}\selectfont
\begin{leftbubbles}\'{E}i, esi didh\'{u}rach riem aman. Aman a chw\'{e}ri canith, in chwerpenach.\end{leftbubbles}
\begin{rightbubbles}\'{A}, esi aman m\'{o} chwerpenachu.\end{rightbubbles}
\begin{leftbubbles}A hesi \'{\i} in vithw\'{\i}r?\end{leftbubbles}
\begin{rightbubbles}Esi \'{\i}. D\'{a}ma m\'{o} hemp\'{a} adhith in gaman o cerdha \'{\i} ...\end{rightbubbles}
\begin{leftbubbles}Suvis.\end{leftbubbles}
\begin{rightbubbles}Diantha aman can shech\'{o}n\'{e}, och \'{a}va s\'{\i} min\'{u}th\'{e}, och \'{a}va s\'{\i} \'{o}r\'{e}.\end{rightbubbles}
\begin{leftbubbles}Esi s\'{e} certh.\end{leftbubbles}
\begin{rightbubbles}N\'{e} ghavis\'{\i} mini sin \'{e}ithr sim \'{o}r.\end{rightbubbles}
\begin{leftbubbles}I\'{a}nu sath.\end{leftbubbles}
\begin{rightbubbles}Esi d\'{a}dhech \'{o}r en dh\'{\i}, ach d\'{a}dhech \'{o}r en nh\'{o}ith, a han\'{e}tham a’n n\'{o}ith’sam\'{a}l\'{e}.\end{rightbubbles}
\begin{leftbubbles}In gerth.\end{leftbubbles}
\begin{rightbubbles}Esi er\'{e}dhl gwochon-pethr \'{o}r anw\'{\i}thu ``lath\'{\i}u''.\end{rightbubbles}
\begin{leftbubbles}Esi \'{\i}.\end{leftbubbles}
\begin{rightbubbles}Enelgha lath\'{\i}u b\'{a}r\'{e}i, methin, m\'{e}dhi, \'{o}sw\'{e}dhi, nesn\'{o}ith, ach m\'{e}dhn\'{o}ith.\end{rightbubbles}
\begin{leftbubbles}I\'{a}nu sath. Arw\'{e}ra adhim in rhanalch in n\'{o}ith. Ach ti?\end{leftbubbles}
\begin{rightbubbles}Penarw\'{e}ra adhim in dh\'{\i}.\end{rightbubbles}
\begin{leftbubbles}Ap\'{\i}sa mi. P\'{e} a havos\'{\i} ti av\'{a}r\'{e}i?\end{leftbubbles}
\begin{rightbubbles}In peth samal o r\'{e} h\'{a}v\'{o} mi d\'{\i}es ach cin dh\'{\i}es.\end{rightbubbles}
\begin{leftbubbles}Ach p\'{e} a v\'{u} s\'{e}?\end{leftbubbles}
\begin{rightbubbles}B\'{\i} mi en dhiluthri m\'{o} shuchnusan u b\'{e}n\'{e} ser\'{\i}thu.\end{rightbubbles}
\begin{leftbubbles}Echan dhin\'{e}an. Ach \'{o}s hav\'{a}r\'{e}i?\end{leftbubbles}
\begin{rightbubbles}In shamal.\end{rightbubbles}
\begin{leftbubbles}Certh.\end{leftbubbles}
\begin{rightbubbles}N\'{u}, \'{a}va s\'{e}ith n\'{o}ith s\'{e}ithn\'{o}ith.\end{rightbubbles}
\begin{leftbubbles}In vithw\'{\i}r?\end{leftbubbles}
\begin{rightbubbles}Ach esi in d\'{a} d\'{\i} \'{o}s anw\'{\i}thu in pens\'{e}ithn\'{o}ith.\end{rightbubbles}
\begin{leftbubbles}\'{a}, tr\'{e}v\'{\i}u lav\'{a}ra ti am\'{\i} ... p\'{e} a havos\'{\i} ti in pens\'{e}ithn\'{o}ith-sin?\end{leftbubbles}
\begin{rightbubbles}B\'{\i} mi en lhauni m\'{o} chwolth en dh\'{u}an'\'{e}p ach duvr'wargh.\end{rightbubbles}
\begin{leftbubbles}Certh.\end{leftbubbles}
\begin{rightbubbles}Av\'{o}thu mi chi sin\'{\i} ach avos\'{\i} mi ch\'{\i} ath\'{e} sin\'{o}ith.\end{rightbubbles}
\begin{leftbubbles}Sulichach.\end{leftbubbles}
\begin{rightbubbles}Duch, esi pethr s\'{e}ithn\'{o}ith en on m\'{\i}s.\end{rightbubbles}
\begin{leftbubbles}Esi.\end{leftbubbles}
\begin{rightbubbles}Ach esi tr\'{\i} m\'{\i}s en on sonching.\end{rightbubbles}
\begin{leftbubbles}A hesi?\end{leftbubbles}
\begin{rightbubbles}Esi. Ach esi pethr sonching en on bledhn: gw\'{\i}son, sam, meth, g\'{\i}am.\end{rightbubbles}
\begin{leftbubbles}Swausa \'{\i} certh \'{e}th gw\'{e} an\'{e}th.\end{leftbubbles}
\begin{rightbubbles}Ach litha ni gwer\'{a}n\'{e} chwerpenach tar in bledhn. Com\'{\i}u penvledhn\'{e}.\end{rightbubbles}
\begin{leftbubbles}Esi s\'{e} certh! Ponch a hesi t\'{o} benvl\'{e}dhn ach p\'{e} a havos\'{\i} ti?\end{leftbubbles}
\begin{rightbubbles}B\'{u} \'{\i} in s\'{e}ithn\'{o}ith \'{o}s.\end{rightbubbles}
\begin{leftbubbles}\'{a}\end{leftbubbles}
\begin{rightbubbles}Aun\'{e} nep ni-esi lith\'{a}n\'{e} trin\'{o}ith ach dechn\'{o}ith, com\'{\i}u ri sh\'{a}n’su\'{e}l\'{e} ach samal’n\'{o}ith\'{e}.\end{rightbubbles}
\begin{leftbubbles}In gerth! P\'{e} a havos\'{\i} ti ton?\end{leftbubbles}
\begin{rightbubbles}B\'{\i} mi en shuling dherthol \'{e}r in ten ...\end{rightbubbles}
\begin{leftbubbles}\'{A}! N\'{u} lav\'{a}ra ni ...\end{leftbubbles}
\begin{rightbubbles}... can m\'{o} gar\'{a}n\'{e} geneth ...\end{rightbubbles}
\begin{leftbubbles}M\'{o}i!\end{leftbubbles}
\begin{rightbubbles}... en lhoscr\'{\i}thi cl\'{a}dh\'{e} rh\'{e} h\'{a}chw\'{a}r ...\end{rightbubbles}
\begin{leftbubbles}A h\'{a}va s\'{u}?\end{leftbubbles}
\begin{rightbubbles}... ach suvi\'{o}na ni sal\'{a}th\'{e} en lhithalach.\end{rightbubbles}
\begin{leftbubbles}Ap\'{\i}sa mi.\end{leftbubbles}
\begin{rightbubbles}Ton, esi in bl\'{e}dhn\'{e} suchnus\'{\i}thu en sh\'{e}thl\'{e} ...\end{rightbubbles}
\begin{leftbubbles}P\'{e} a hesi s\'{\i}?\end{leftbubbles}
\begin{rightbubbles}Esi s\'{e} e gh\'{e}nu ith\'{\i} a gh\'{e}nu t\'{o} wapath cin.\end{rightbubbles}
\begin{leftbubbles}\'{a}, n\'{u}, gwer chw\'{o}chatha gnath\'{a}l\'{e} ...\end{leftbubbles}
\begin{rightbubbles}Ach ton esi in bl\'{e}dhn\'{e} cans\'{o}ith\'{u} en hai\'{u}\'{e}.\end{rightbubbles}
\begin{leftbubbles}P\'{e} a hesi \'{a}iu?\end{leftbubbles}
\begin{rightbubbles}Esi \'{\i} gwochon-dech bl\'{e}dhn, gw\'{e} swech \'{o}chan amanar.\end{rightbubbles}
\begin{leftbubbles}Certh.\end{leftbubbles}
\begin{rightbubbles}N\'{u}, \'{a}iu ...\end{rightbubbles}
\begin{leftbubbles}\'{a}iu?\end{leftbubbles}
\begin{rightbubbles}... esi s\'{e} in gerth in \'{e}r\'{e}dhl o rinchas\'{\i} ti an\'{e}li aven gals\'{\i} ti \'{a}dha to dhalam gwerim.\end{rightbubbles}
\endgroup

\newpage
\subsubsection{Colav\'{a}ru \textendash\ Tr\'{e}lav\'{a}ru}
(Conversation \textendash\ Translation)\\

A man, Arthu, is talking to a woman, Melina.

Arthu = The Bear < Artos
Melina = The Honey < Melina

A: \'{E}i, esi didh\'{u}rach riem aman. Aman a chw\'{e}ri canith, in chwerpenach.
(A: Hey, I’m interested in time. Spending time with you, specifically.)

M: \'{A}, esi aman m\'{o} chwerpenachu. 
(M: Ah yes, time is my specialty.)

A: A hesi \'{\i} in vithw\'{\i}r?
(A: Is it really?)

M: Esi \'{\i}. D\'{a}ma m\'{o} hemp\'{a} adhith in gaman o cerdha \'{\i} ...
(M: It is. Let me tell you how it works ...)

A: Suvis.
(A: All right.)

M: Diantha aman can shech\'{o}n\'{e}, och \'{a}va s\'{\i} min\'{u}th\'{e}, och \'{a}va s\'{\i} \'{o}r\'{e}.
(M: Time starts with seconds, which make up minutes, which make up hours.)

A: Esi s\'{e} certh.
(A: That’s right.)

M: N\'{e} ghavis\'{\i} mini sin \'{e}ithr sim \'{o}r.
(M: Explaining this will only take half an hour.)

A: I\'{a}nu sath.
(A: Fair enough.)

M: Esi d\'{a}dhech \'{o}r en dh\'{\i}, ach d\'{a}dhech \'{o}r en nh\'{o}ith, a han\'{e}tham a’n n\'{o}ith’sam\'{a}l\'{e}.
(M: There are 12 hours in a day, and 12 hours in a night, at least at the equinoxes.)

A: In gerth.
(A: Exactly.)

M: Esi er\'{e}dhl gwochon-pethr \'{o}r anw\'{\i}thu ``lath\'{\i}u''.
(M: A whole 24 hour period is called a ``day period''.)

A: Esi \'{\i}.
(A: It is.)

M: Enelgha lath\'{\i}u b\'{a}r\'{e}i, methin, m\'{e}dhi, \'{o}sw\'{e}dhi, nesn\'{o}ith, ach m\'{e}dhn\'{o}ith.
(M: A day period consists of a dawn, a morning, a midday, an afternoon, an evening, and a midnight.)

A: I\'{a}nu sath. Arw\'{e}ra adhim in rhanalch in n\'{o}ith. Ach ti?
(A: Fair enough. I particularly like the night. What about you?)

M: Penarw\'{e}ra adhim in dh\'{\i}.
(M: I prefer the day.)

A: Ap\'{\i}sa mi. P\'{e} a havos\'{\i} ti av\'{a}r\'{e}i?
(A: I see. What are you doing tomorrow?)

M: In peth samal o r\'{e} h\'{a}v\'{o} mi d\'{\i}es ach cin dh\'{\i}es.
(M: The same thing I did yesterday and the day before yesterday.)

A: Ach p\'{e} a v\'{u} s\'{e}?
(A: And what was that?)

M: B\'{\i} mi en dhiluthri m\'{o} shuchnusan u b\'{e}n\'{e} ser\'{\i}thu.
(M: I will be cleaning my collection of severed heads.)

A: Echan dhin\'{e}an. Ach \'{o}s hav\'{a}r\'{e}i?
(A: Of course. And the day after tomorrow?)

M: In shamal.
(M: The same.)

A: Certh.
(A: Right.)

M: N\'{u}, \'{a}va s\'{e}ith n\'{o}ith s\'{e}ithn\'{o}ith.
(M: Now, seven days make up a week.)

A: In vithw\'{\i}r?
(A: Really?)

M: Ach esi in d\'{a} d\'{\i} \'{o}s anw\'{\i}thu in pens\'{e}ithn\'{o}ith.
(M: And the last two days are called the weekend.)

A: \'{a}, tr\'{e}v\'{\i}u lav\'{a}ra ti am\'{\i} ... p\'{e} a havos\'{\i} ti in pens\'{e}ithn\'{o}ith-sin?
(A: Ah yes, since you mention it ... what are you doing this weekend?)

M: B\'{\i} mi en lhauni m\'{o} chwolth en dh\'{u}an'\'{e}p ach duvr'wargh.
(M: I’ll be washing my hair in horsepiss and lime.)

A: Certh.
(A: Right.)

M: Av\'{o}thu mi chi sin\'{\i} ach avos\'{\i} mi ch\'{\i} ath\'{e} sin\'{o}ith.
(M: I did it today and I’ll do it again tonight.)

A: Sulichach.
(A: Charming.)

M: Duch, esi pethr s\'{e}ithn\'{o}ith en on m\'{\i}s.
(M: So, there are four weeks in a month.)

A: Esi.
(A: There are.)

M: Ach esi tr\'{\i} m\'{\i}s en on sonching.
(M: And there are three months in a season.)

A: A hesi?
(A: Is there?)

M: Esi. Ach esi pethr sonching en on bledhn: gw\'{\i}son, sam, meth, g\'{\i}am.
(M: There is. And there are four seasons in one year: spring, summer, autumn, winter.)

A: Swausa \'{\i} certh \'{e}th gw\'{e} an\'{e}th.
(A: Sounds about right.)

M: Ach litha ni gwer\'{a}n\'{e} chwerpenach tar in bledhn. Com\'{\i}u penvledhn\'{e}.
(M: And we celebrate special events throughout the year. Such as birthdays.)

A: Esi s\'{e} certh! Ponch a hesi t\'{o} benvl\'{e}dhn ach p\'{e} a havos\'{\i} ti?
(A: That’s right! When’s your birthday and what will you be doing?)

M: B\'{u} \'{\i} in s\'{e}ithn\'{o}ith \'{o}s.
(M: It was last week.)

A: \'{A}.
(A: Ah.)

M: Aun\'{e} nep ni-esi lith\'{a}n\'{e} trin\'{o}ith ach dechn\'{o}ith, com\'{\i}u ri sh\'{a}n’su\'{e}l\'{e} ach samal’n\'{o}ith\'{e}.
(M: At times we have three-night and ten-night celebrations, such as at the solstices and equinoxes.)

A: In gerth! P\'{e} a havos\'{\i} ti ton?
(A: Exactly! What will you be doing then?)

M: B\'{\i} mi en shuling dherthol \'{e}r in ten ...
(M: I will be dancing around the fire in the nude ...)

A: \'{A}! N\'{u} lav\'{a}ra ni ...
(A: Ah! Now we’re talking ...)

M: ... can m\'{o} gar\'{a}n\'{e} geneth ...
(M: ... with all my girlfriends ...)

A: M\'{o}i!
(A: Great!)

M: ... en lhoscr\'{\i}thi cl\'{a}dh\'{e} rh\'{e} h\'{a}chw\'{a}r ...
(M: ... wielding very sharp swords ...)

A: A h\'{a}va s\'{u}?
(A: You do?)

M: ... ach suvi\'{o}na ni sal\'{a}th\'{e} en lhithalach.
(M: ... and we ritually slice up sausages.)

A: Ap\'{\i}sa mi.
(A: I see.)

M: Ton, esi in bl\'{e}dhn\'{e} suchnus\'{\i}thu en sh\'{e}thl\'{e} ... 
(M: Then, the years are gathered in generations ...)

A: P\'{e} a hesi s\'{\i}?
(A: What are they?)

M: Esi s\'{e} e gh\'{e}nu ith\'{\i} a gh\'{e}nu t\'{o} wapath cin.
(M: That’s from your birth to the birth of your first child.)

A: \'{A}, n\'{u}, gwer chw\'{o}chatha gnath\'{a}l\'{e} ...
(A: Ah, now, on the subject of babies ...)

M: Ach ton esi in bl\'{e}dhn\'{e} cans\'{o}ith\'{u} en hai\'{u}\'{e}.
(M: And then the years are counted in ages.)

A: P\'{e} a hesi \'{a}iu?
(A: What’s an age?)

M: Esi \'{\i} gwochon-dech bl\'{e}dhn, gw\'{e} swech \'{o}chan amanar.
(M: It is thirty years, or six calendar completions.)

A: Certh.
(A: Right.)

M: N\'{u}, \'{a}iu ...
(M: Now, an age ...)

A: \'{A}iu?
(A: Yes?)

M: ... esi s\'{e} in gerth in \'{e}r\'{e}dhl o rinchas\'{\i} ti an\'{e}li aven gals\'{\i} ti \'{a}dha to dhalam gwerim.
(M: ... that’s exactly how long you’re going to have to wait before you ever get your hands on me.)


In the conversation above two people use all the terms relating to time in Gal\'{a}thach. They are listed here. Some of the words are modern loans.

\begin{table}[H]
\centering
\begin{tabular}{ccc}
  \toprule
  \textbf{Gal\'{a}thach} & \textbf{English} & \textbf{Origin}\\
  \cmidrule(lr){1-1}\cmidrule(lr){2-2}\cmidrule(lr){3-3}
  sechon & second & \\
  minuth & minute & \\
  pimdhech minuth & fifteen minutes, quarter of an hour\\
  sim \'{o}r & half hour & (< sim \textit{half}, attested)\\
  \'{o}r & hour & \\
  \midrule
  b\'{a}r\'{e}i & dawn & (< proto-Celtic *ba:re:gom, loss of –om \& –eg > \'{e}i cf. Lambert 2003, p. 43)\\
  methin & morning & (< Latin \textit{matina}, cf. Br. mintin, C. metten, Ir. maidin)\\
  m\'{e}dhi & midday & (< medh \textit{middle} + d\'{\i})\\
  \'{o}sw\'{e}dhi & afternoon & (< \'{o}s \textit{after} + medh + d\'{\i})\\
  nesn\'{o}ith & evening & (< nes \textit{near, close to} + n\'{o}ith\\
  medhn\'{o}ith & midnight & (< medh + noith)\\
  \midrule
  d\'{\i} & day & \\
  sin\'{\i} & today & \\
  n\'{o}ith & night & \\
  sin\'{o}ith & tonight & (by analogy with sin\'{\i} < sindiu > sindenocta > sin\'{o}ith)\\
  lath\'{\i}u & period of 24 hours, daytime, day and night & \\
  \midrule
  av\'{a}r\'{e}i & tomorrow & (< a \textit{to, at} + b\'{a}r\'{e}i)\\
  \'{o}s hav\'{a}r\'{e}i & after tomorrow & \\
  d\'{\i}es & yesterday & (< proto-Celtic *gdijes)\\
  cin dh\'{\i}es & before yesterday & \\
  \midrule
  pens\'{e}ithn\'{o}ith & weekend & (< pen \textit{head} + s\'{e}thn\'{o}ith)\\
  s\'{e}ithn\'{o}ith & week & (\textit{seven-night}, by analogy with e.g. trinoctia)\\
  m\'{\i}s & month & \\
  \midrule
  trin\'{o}ith & three-night feast & \\
  dechn\'{o}ith & ten-night feast & \\
  \midrule
  sam & summer & \\
  meth & autumn (\textit{harvest} < *met- \textit{to harvest})\\
  g\'{\i}am & winter & \\
  gwison & spring & \\
  \midrule
  sonching & season (period) & \\
  bl\'{e}dhn & year & \\
  penvl\'{e}dhn & anniversary, birthday & \\
  \midrule
  s\'{e}thl & generation & \\
  \'{a}iu & age & \\
  aman & time & \\
  \bottomrule
\end{tabular}
\label{vocab_conversation_lesson16}
\caption{Vocabulary conversation lesson 16}
\end{table}

\subsection{Exercises: The temporal system}

\subsubsection{Translate}

Use the words given above and the vocabulary learned so far to construct the following sentences.

Notes: 

To say how old a person is the verb \textit{to have} is used
> I am twenty years old = I have twenty years.

To indicate the time the number of hours is indicated
> it is five o’clock = it is five hours.

\begin{table}[H]
\centering
\begin{tabular}{|c|M{5.0cm}|}
  \toprule
  \textbf{English}\\
  \textbf{Gal\'{a}thach}\\
  \toprule
  Can you run a hundred metres in ten seconds?\\
  \\
  \midrule
  No, it would take me five minutes.\\
  \\
  \midrule
  Will you be here after fifteen minutes?\\
  \\
  \midrule
  No, I will be gone for half an hour.\\
  \\
  \midrule
  I will run for an hour.\\
  \\
  \midrule
  She swims in the ocean at dawn.\\
  \\
  \midrule
  He slept until ten o’clock of the morning.\\
  \\
  \midrule
  We could eat at midday.\\
  \\
  \midrule
  They like to sleep in the afternoon\\
  \\
  \midrule
  Will you (pl.) tell stories this evening?\\
  \\
  \midrule
  You danced until midnight.\\
  \\
  \midrule
  Today is a beautiful day.\\
  \\
  \midrule
  Tonight will also be a beautiful night.\\
  \\
  \midrule
  He has not slept for a period of 24 hours.\\
  \\
  \midrule
  Tomorrow you will be very tired.\\
  \\
  \midrule
  After tomorrow you have to go to work.\\
  \\
  \midrule
  She went to see her mother yesterday.\\
  \\
  \midrule
  But her mother died the day before yesterday.\\
  \\
  \midrule
  This weekend we will go to walk through the mountains.\\
  \\
  \midrule
  Next week we will swim in the river.\\
  \\
  \midrule
  This is the first month of summer.\\
  \\
  \midrule
  We will dance in the nude for the three-night feast.\\
  \\
  \midrule
  We will eat and drink without sleeping for the ten-night feast.\\
  \\
  \midrule
  They worked hard in the autumn.\\
  \\
  \midrule
  Last winter their feet froze.\\
  \\
  \midrule
  Spring always makes her nose run.\\
  \\
  \midrule
  It is a difficult season for her.\\
  \\
  \midrule
  What year is this again?\\
  \\
  \midrule
  Happy birthday to you, may you have many returns.\\
  \\
  \midrule
  How old is that girl?\\
  \\
  \midrule
  You are not allowed to ask, she will cut your head off.\\
  \\
  \midrule
  The old people like to complain about the new generation.\\
  \\
  \midrule
  They lived in an age of freedom.\\
  \\
  \midrule
  Now it is a time of slavery.\\
  \\
  \bottomrule
\end{tabular}
\label{exercise_temporal_system}
\caption{Exercise: temporal system}
\end{table}

\newpage
\subsubsection{Solution}

\begin{table}[H]
\centering
\rotatebox{180}{%
  \begin{tabular}{|c|}
  \toprule
  \textbf{English}\\
  \textbf{Gal\'{a}thach}\\
  \toprule
  Can you run a hundred metres in ten seconds?\\
  \textit{A gh\'{a}la ti r\'{e}thi can methr en dhech sechon?}\\
  \midrule
  No, it would take me five minutes.\\
  \textit{N\'{e} gh\'{a}la mi, r\'{e} ghavis\'{\i} \'{\i} mi pimp minuth.}\\
  \midrule
  Will you be here after fifteen minutes?\\
  \textit{A v\'{\i} ti insin \'{o}s pimdhech minuth?}\\
  \midrule
  No, I will be gone for half an hour.\\
  \textit{N\'{e} v\'{\i} mi, b\'{\i} mi \'{a}ithu ri shim \'{o}r.}\\
  \midrule
  I will run for one hour.\\
  \textit{Rethis\'{\i} mi ri on \'{o}r.}\\
  \midrule
  She swims in the ocean at dawn.\\
  \textit{Sn\'{a} \'{\i} en in m\'{o}rw\'{a}r a v\'{a}r\'{e}i.}\\
  \midrule
  He slept until ten o’clock of the morning.\\
  \textit{R\'{e} sh\'{o}ni \'{e} aven dech \'{o}r in methin.}\\
  \midrule
  We could eat at midday.\\
  \textit{R\'{e} ghals\'{\i} ni depri a w\'{e}dhi.}\\
  \midrule
  They like to sleep in the afternoon.\\
  \textit{Arw\'{e}ra adh\'{\i}s s\'{o}ni en in \'{o}sw\'{e}dhi.}\\
  \midrule
  Will you (pl.) tell stories this evening?\\
  \textit{A hempas\'{\i} s\'{u} sp\'{a}thl\'{e} in nesn\'{o}ith-sin?}\\
  \midrule
  You danced until midnight.\\
  \textit{R\'{e} shuling ti aven medhn\'{o}ith.}\\
  \midrule
  Today is a beautiful day.\\
  \textit{Esi sin\'{\i} d\'{\i} dech.}\\
  \midrule
  Tonight will also be a fantastic night.\\
  \textit{B\'{\i} sin\'{o}ith n\'{o}ith swapis\'{o}ich c\'{o}\'{e}th.}\\
  \midrule
  I have not slept for a period of 24 hours.\\
  \textit{N\'{e} shon\'{\i}thu mi ri lhath\'{\i}u.}\\
  \midrule
  Tomorrow you will be very tired.\\
  \textit{B\'{\i} ti r\'{e} lhisc av\'{a}r\'{e}i.}\\
  \midrule
  After tomorrow you have to go to work.\\
  \textit{\'{o}s hav\'{a}r\'{e}i rincha ti \'{a}i a gerdhi.}\\
  \midrule
  She went to see her mother yesterday.\\
  \textit{R\'{e} h\'{a}i \'{\i} a h\'{a}pis \'{o} m\'{a}thir d\'{\i}es.}\\
  \midrule
  But her mother died the day before yesterday.\\
  \textit{\'{e}ithr r\'{e} warwi \'{o} m\'{a}thir cin dh\'{\i}es.}\\
  \midrule
  This weekend we will go to walk through the mountains.\\
  \textit{In pens\'{e}ithn\'{o}ith-sin \'{a}is\'{\i} ni a gama tar in vr\'{\i}\'{e}.}\\
  \midrule
  Next week we will swim in the river.\\
  \textit{In s\'{e}itn\'{o}ith conesam sn\'{a}s\'{\i} ni en in avon.}\\
  \midrule
  This is the first month of summer.\\
  \textit{Esi sin m\'{\i}s gin in sham.}\\
  \midrule
  We will dance in the nude for the three-night feast.\\
  \textit{Sulings\'{\i} ni derthol ri drin\'{o}ith.}\\
  \midrule
  We will eat and drink without sleeping for the ten-night feast.\\
  \textit{Depris\'{\i} ach ivis\'{\i} ni echan sh\'{o}ni ri dhechn\'{o}ith.}\\
  \midrule
  They worked hard in the autumn.\\
  \textit{R\'{e} gerdhi s\'{\i} en galeth en chwison.}\\
  \midrule
  Last winter their feet froze.\\
  \textit{In gh\'{\i}am h\'{o}sim r\'{e} hoghri s\'{o} drai\'{e}th\'{e}.}\\
  \midrule
  Spring always makes her nose run.\\
  \textit{\'{a}va in gwison r\'{e}thi \'{o} trughn aman hol.}\\
  \midrule
  It is a difficult season for her.\\
  \textit{Esi \'{\i} sonching guth rich\'{\i}.}\\
  \midrule
  What year is this again?\\
  \textit{P\'{e} vl\'{e}dhn a hesi sin ath\'{e}?}\\
  \midrule
  Happy birthday to you, may you have many returns.\\
  \textit{Penvl\'{e}dhn l\'{a}en adhith, o ti-v\'{\i} ath\'{e}chwerth\'{a}n\'{e} lh\'{a}en h\'{e}lu.}\\
  \midrule
  How old is that girl?\\
  \textit{P\'{e} h\'{a}iu a h\'{\i}-esi in gheneth-s\'{e}?}\\
  \midrule
  You are not allowed to ask, she will cut your head off.\\
  \textit{N\'{e} shudhamor p\'{e}tha ich\'{\i}, b\'{e}s\'{\i} \'{\i} t\'{o} ben.}\\
  \midrule
  The old people like to complain about the new generation.\\
  \textit{Arw\'{e}ra \'{\i} a’n don\'{e} sen cothr\'{o}ia am in s\'{e}thl n\'{o}i.}\\
  \midrule
  They lived in an age of freedom.\\
  \textit{R\'{e} v\'{\i}thi s\'{\i} en h\'{a}iu r\'{\i}as.}\\
  \midrule
  Now it is a time of slavery.\\
  \textit{N\'{u} esi \'{\i} aman caithan.}\\
  \bottomrule
  \end{tabular}
}
\label{solution_temporal_system}
\caption{Solution: temporal system}
\end{table}
