\section{Menghavan 8: Rivrethr\'{e}}
(\textit{Lesson 8: Adverbs})\\

In the eighth lesson you will learn about adverbs.

\subsection{Adverbs}

An adverb is a word that describes a quality or a characteristic of a verbal action.\\
English examples are: quickly, quietly, calmly, naturally, normally. In English they are usually formed with the suffix \textit{-ly}.\\
In Gal\'{a}thach adverbs are formed by placing the particle \textit{in} in front of an adjective. The adjective undergoes mutation of its initial consonant.
\begin{table}[H]
\centering
\begin{tabu}{c|c}
  \textbf{Gal\'{a}thach} & \textbf{English}\\
  \toprule
  \'{a}chu & quick\\
  in h\'{a}chu & quickly\\
  \'{a}va mi ch\'{i} & I do it\\
  \'{a}va mi ch\'{i} in h\'{a}chu & I do it quickly\\
  \midrule
  tau & quiet\\
  in dau & quietly\\
  sp\'{a} \'{i} ch\'{i} & she says it\\
  sp\'{a} \'{i} ch\'{i} in dau & she says it quietly\\
  \midrule
  aram & calm\\
  in haram & calm\\
  r\'{e}na in avon & the river flows\\
  r\'{e}na in avon in haram & the river flows calmly\\
  \midrule
  amv\'{i}thach & natural\\
  in hamv\'{i}thach & naturally\\
  gw\'{o}ra cr\'{a}r\'{e} mel & bees produce honey\\
  gw\'{o}ra cr\'{a}r\'{e} mel in hamv\'{i}thach & bees produce honey naturally\\
  \midrule
  suves & normal\\
  in shuves & normally\\
  n\'{e} chwergha \'{i} co sh\'{e} & she doesn't act like that\\
  n\'{e} chwergha \'{i} co sh\'{e} in shuves & she doesn't normally act like that
\end{tabu}
\label{examples_adverb}
\end{table}

\subsubsection{Position Of The Adverb}
The adverb always follows the verb as closely as possible, after the subject and object of the phrase.
%Gw�ra cr�r� mel in hamv�thach // �produce bees honey naturally�
%Verb   Sub. Obj.    Adverb               V          S     O          Adv

\newpage
\subsubsection{Exercises}

Construct the adverbial form of the following adjectives. You can check your answers at the end of the lesson.

%c�il (narrow) &gt;
%lithan (wide) &gt;
%dianauch (poor) &gt;
%t�ithw�r (wealthy) &gt;
%ardhu (high) &gt;
%�th (low) &gt;
%pethrarpenach (square) &gt;
%r�thach (round) &gt;
%d�i (good) &gt;
%druch (bad) &gt;

\newpage
Solution:
%c�il &gt; in g�il (narrowly)
%lithan &gt; in lhithan (widely)
%dianauch &gt; in dhianauch (poorly)
%t�ithw�r &gt; in d�ithw�r (wealthily)
%ardhu &gt; in hardhu (highly)
%�th &gt; in h�th (lowly)
%pethrarpenach &gt; in betharpenach (squarely)
%r�thach &gt; in rhothach (roundly)
%d�i &gt; in dh�i (well)
%druch &gt; in dhruch (badly)

