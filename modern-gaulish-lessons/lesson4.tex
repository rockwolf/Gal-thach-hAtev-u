%\noindent <strong>                                                                                                           </strong>
%<h1><strong><span style="color:#ff0000;">Menghavan 4: Alghnas Anúé</span> </strong></h1>
%<h2><strong> <span style="color:#ff0000;">Lesson 4: Gender Of Nouns</span> </strong></h2>
%\noindent In the fourth lesson you will learn how to determine the gender of nouns.
%\noindent <strong>Gender Of Nouns</strong>
%\noindent In modern Gaulish all nouns have a gender, which is either masculine or feminine. If the meaning of the noun indicates a gender, then that noun is that gender:
%\noindent gwir: man
%\noindent ben: woman
%\noindent mapath: boy
%\noindent geneth: girl
%\noindent map: son
%\noindent dúithir: daughter
%\noindent áther: father
%\noindent máthir: mother
%\noindent moth: penis
%\noindent tuthu: vagina
%\noindent If the meaning of a noun does not indicate a gender, its gender is determined by the last vowel. If the last vowel is /a/ or /i/ the noun is feminine.
%\noindent lam: hand &gt; fem.
%\noindent bis: finger &gt; fem
%\noindent tír: land &gt; fem
%\noindent cnam: bone &gt; fem
%\noindent If the last vowel is /e/, /o/ or /u/ the noun is masculine.
%\noindent pen: head &gt; masc.
%\noindent mór: sea &gt; masc.
%\noindent cánu: song &gt; masc.
%\noindent tráieth: foot: masc.
%\noindent coch: leg: masc.
%\noindent curu: beer &gt; masc.
%\noindent The –i in the diphthongs –ái-, -éi-, -ói- and –úi- is not a vowel, it is a semi-consonant, like /y/ in English. It does not count as a vowel, and its presence does not make a noun’s gender feminine:
%\noindent brói: country &gt; last vowel is /o/, the –i is the semi-consonant &gt; masc.
%\noindent mái: place, plain &gt; last vowel is /a/ &gt; fem.
%\noindent téi: house &gt; last vowel is /e/ &gt; masc.
%\noindent gwólúith: strain &gt; last vowel is /u/ &gt; masc.
%\noindent Some nouns end in a double consonant where the last consonant is l, n or r. When pronounced there is a dull indistinct sound between the second last consonant and the l, n or r. This sound is called schwa, and is represented by the symbol [?]. It is not considered a vowel and is not written. It does not affect the gender of a noun. The gender of such a noun is determined by the last vowel before the schwa:
%\noindent sédhl: seat &gt; last vowel is /e/ &gt; masc.
%\noindent sparn: thorn &gt; last vowel is /a/ &gt; fem
%\noindent livr: book &gt; last vowel is /i/ &gt; fem.
%\noindent Some nouns end in a diphthong followed by a double consonant where the last consonant is l, n or r. The gender of these nouns is determined by the last vowel before the –i of the diphthong:
%\noindent anéithl: protection &gt; last vowel is /e/ &gt; masc.
%\noindent lúithr: struggle &gt; last vowel is /u/ &gt; masc.
%\noindent bóithl: hit &gt; last vowel is /o/ &gt; masc.
%\noindent amáithl: service &gt; last vowel is /a/ &gt; fem.
%\noindent Nouns of animals are masculine by default, even if the vowels are /a/ or /i/:
%\noindent garan: heron &gt; masc.
%\noindent cun: dog &gt; masc.
%\noindent lóern: fox &gt; masc.
%\noindent ép: horse &gt; masc.
%\noindent caval: [draught] horse &gt; masc.
%\noindent bó: cow [generic name for cattle]
%\noindent These nouns can be made feminine by adding the suffix –is:
%\noindent garanis: female heron
%\noindent cunis: bitch
%\noindent lóernis: vixen
%\noindent épis: mare (also casich)
%\noindent cavalis: female draught horse
%\noindent Nouns indicating human functions or activities are also masculine by default:
%\noindent drúidh: scholar &gt; masc.
%\noindent gwerchovreth: magistrate &gt; masc.
%\noindent tiern: boss, chief &gt; masc.
%\noindent dan: official, manager &gt; masc.
%\noindent These nouns can also be made feminine by adding the suffix –is:
%\noindent drúidhis: female scholar
%\noindent gwerchovrethis: female magistrate
%\noindent tiernis: female boss, chief
%\noindent danis: female official, manager
%\noindent <strong>Exercises </strong>
%\noindent Determine the gender of the following nouns:
%\noindent car: car
%\noindent sesa: chair
%\noindent roth: wheel
%\noindent aríthis: table
%\noindent dulu: paper
%\noindent cumlath: plate
%\noindent cladhal: knife
%\noindent gaval: fork
%\noindent bóthéi: stable
%\noindent bochwídhu: spoon
%\noindent ethn: bird
%\noindent táru: bull
%\noindent amáiath: servant
%\noindent cerdhíath: worker
%\noindent menrodhiath: teacher
%\noindent gnisáiath: student
%\noindent pethlói: stuff
%\noindent pren: tree
%\noindent bil: tree trunk
%\noindent clétha: ladder
%\noindent cilurn: bucket
%\noindent scothir: shovel
%\noindent cerdhl: work
%\noindent tráith: beach
%\noindent crósu: wave
%\noindent sir: star
%\noindent nem: sky
%\noindent brí: hill
%\noindent brói: country
%\noindent bélói: culture
%\noindent tengu: language
%\noindent tarinch: nail (fastening implement)
%\noindent cingeth: warrior
%\noindent delgheth: holder
%\noindent druthas: courage
%\noindent dumnas: darkness
%\noindent échal: hoof
%\noindent You can check your answers below.
%\noindent <strong>Answers</strong>
%\noindent car: car &gt; f
%\noindent sesa: chair &gt; f
%\noindent roth: wheel &gt; m
%\noindent aríthis: table &gt; f
%\noindent dulu: paper &gt; m
%\noindent cumlath: plate &gt; f
%\noindent cladhal: knife &gt; f
%\noindent gaval: fork &gt; f
%\noindent bóthéi: stable &gt; m
%\noindent bochwídhu: spoon &gt; m
%\noindent ethn: bird &gt; m
%\noindent táru: bull &gt; m
%\noindent amáiath: servant &gt; m
%\noindent cerdhíath: worker &gt; m
%\noindent menrodhiath: teacher &gt; m
%\noindent gnisáiath: student &gt; m
%\noindent pethlói: stuff &gt; m
%\noindent pren: tree &gt; m
%\noindent bil: tree trunk &gt; f
%\noindent clétha: ladder &gt; f
%\noindent cilurn: bucket &gt; m
%\noindent scothir: shovel &gt; f
%\noindent cerdhl: work &gt; m
%\noindent tráith: beach &gt; f
%\noindent crósu: wave &gt; m
%\noindent sir: star &gt; f
%\noindent nem: sky &gt; m
%\noindent brí: hill &gt; f
%\noindent brói: country &gt; m
%\noindent bélói: culture &gt; m
%\noindent tengu: language &gt; m
%\noindent tarinch: nail (for hammering) &gt; f
%\noindent cingeth: warrior &gt; m
%\noindent delgheth: holder &gt; m
%\noindent druthas: courage &gt; f
%\noindent dumnas: darkness &gt; f
%\noindent échal: hoof &gt; f
%\noindent <strong>                                                                                                           </strong>
%
