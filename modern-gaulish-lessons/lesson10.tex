\section{Menghavan 10: Gweran\'{u}\'{e} T\'{e}ithach \textemdash\ Adhavach\'{u}\'{e} \textendash\ Nithach\'{u}\'{e}}
(\textit{Lesson 10: Possessive Pronouns \textemdash\ Demonstratives \textendash\ Locatives})\\

In the tenth lesson you will learn what possessive pronouns and demonstratives are, and how they work.

\subsection{Gwepchoprith: Possessive pronouns}
\subsubsection{Possessive pronouns}

A possessive pronoun is a word that indicates possession of a thing. In Lesson 2 and 3 we learned two ways of indicating possession: by using the possession indicating particle \textit{i-}, and by putting two things next to each other, with or without the article between them. Using a possessive pronoun is the third way of indicating possession.\\

A possessive pronoun is a word that comes before an other word and that indicates possession of the second word: my, your, his, her etc.
\begin{table}[H]
\centering
\begin{tabular}{cc}
  \toprule
  \textbf{Gal\'{a}thach} & \textbf{English}\\
  \cmidrule(lr){1-1}\cmidrule(lr){2-2}
  cun & dog\\
  m\'{o} & my\\
  m\'{o} gun & my dog\\
  \bottomrule
\end{tabular}
\label{examples_possessive_pronoun}
\end{table}

The possessive pronoun causes mutation on the following word in most cases, but not for the 3\textsuperscript{rd} person singular feminine and not for the second person plural.
\begin{table}[H]
\centering
\begin{tabular}{cc}
  \toprule
  \textbf{Gal\'{a}thach} & \textbf{English}\\
  \cmidrule(lr){1-1}\cmidrule(lr){2-2}
  m\'{o} & my\\
  t\'{o} & your\\
  \'{o} & his\\
  \'{o} & her\\
  n\'{o} & our\\
  s\'{o} & your (pl.)\\
  s\'{o} & their\\
  \bottomrule
\end{tabular}
\label{summary_possessive_pronoun}
\end{table}

Because the third and the fourth and the sixth and the seven are the same, they are distinguished by the mutation: the third and the sixth cause mutation, the fourth and the seventh don't.
\begin{table}[H]
\centering
\begin{tabular}{ccc}
  \toprule
  \textbf{Gal\'{a}thach} & \textbf{English} & \textbf{Mutation}\\
  \cmidrule(lr){1-1}\cmidrule(lr){2-2}\cmidrule(lr){3-3}
  m\'{o} & my & mut.\\
  t\'{o} & your & mut.\\
  \'{o} & his & mut.\\
  \'{o} & her & \textbf{no} mut.\\
  n\'{o} & our & mut.\\
  s\'{o} & your (pl.) & \textbf{no} mut.\\
  s\'{o} & their & mut.\\
  \bottomrule
\end{tabular}
\label{summary_possessive_pronoun_mutation}
\end{table}

\begin{table}[H]
\centering
\begin{tabular}{cc}
  \toprule
  \textbf{Gal\'{a}thach} & \textbf{English}\\
  \cmidrule(lr){1-1}\cmidrule(lr){2-2}
  m\'{o} \textbf{g}un & my dog\\
  t\'{o} \textbf{g}un & your dog\\
  \'{o} \textbf{g}un & his dog\\
  \'{o} \textbf{c}un & her dog\\
  n\'{o} \textbf{g}un & our dog\\
  s\'{o} \textbf{c}un & your (pl.) dog\\
  s\'{o} \textbf{g}un & their dog\\
  \bottomrule
\end{tabular}
\label{examples_possessive_pronoun_mutation}
\end{table}

Remember that possession can also be expressed with the particle \textit{i-}. In that case it uses the object pronouns, and they follow after the possessed thing.
\begin{table}[H]
\centering
\begin{tabular}{cc}
  \toprule
  \textbf{Gal\'{a}thach} & \textbf{English}\\
  \cmidrule(lr){1-1}\cmidrule(lr){2-2}
  cun im\'{\i} & a dog of mine\\
  cun ith\'{\i} & a dog of yours\\
  cun ich\'{e} & a dog of his\\
  cun ich\'{\i} & a dog of hers\\
  cun in\'{\i} & a dog of ours\\
  cun is\'{u} & a dog of yours (pl.)\\
  cun ich\'{\i}s & a dog of theirs\\
  \bottomrule
\end{tabular}
\label{examples_possessive_pronoun_particle_i}
\end{table}

\newpage
\subsection{Exercises: Possessive pronouns}

\subsubsection{Translate}

Using the words below translate the following phrases.
\begin{table}[H]
\centering
\begin{tabular}{|c|c|}
  \toprule
  \textbf{Gal\'{a}thach} & \textbf{English}\\
  \toprule
  \'{e}pis & mare\\
  b\'{o} & cow\\
  t\'{a}ru & bull\\
  molth & sheep\\
  gavr & goat\\
  camoch & mountain goat\\
  \'{e}lan & doe\\
  \'{u}ru & aurochs\\
  c\'{a}ru & deer\\
  avanch & water monster\\
  bevr & beaver\\
  m\'{o}rchun & dolphin\\
  anchr\'{a}i & salmon\\
  garan & heron\\
  \bottomrule
\end{tabular}
\label{exercise_possessive_pronoun_vocab}
\end{table}

\begin{table}[H]
\centering
\begin{tabular}{|c|M{5.0cm}|}
  \toprule
  \textbf{English} & \textbf{Gal\'{a}thach}\\
  \toprule
  you go with me & \\
  \midrule
  my mare & \\
  \midrule
  your cow & \\
  \midrule
  his bull & \\
  \midrule
  her sheep & \\
  \midrule
  our goat & \\
  \midrule
  your (pl.) mountain goat & \\
  \midrule
  their doe & \\
  \midrule
  my aurochs & \\
  \midrule
  your deer & \\
  \midrule
  his water monster & \\
  \midrule
  her beaver & \\
  \midrule
  our dolphin & \\
  \midrule
  your (pl.) salmon & \\
  \midrule
  their heron & \\
  \bottomrule
\end{tabular}
\label{exercise_possessive_pronouns}
\caption{Exercise: possessive pronouns}
\end{table}

\newpage
\subsubsection{Solution}

\begin{table}[H]
\centering
\rotatebox{180}{%
  \begin{tabular}{|c|>{\itshape}c|}
  \toprule
  \textbf{English} & \textbf{Gal\'{a}thach}\\
  \toprule
  my mare & m\'{o} h\'{e}pis\\
  \midrule
  your cow & t\'{o} v\'{o}\\
  \midrule
  his bull & \'{o} d\'{a}ru\\
  \midrule
  her sheep & \'{o} molth\\
  \midrule
  our goat & n\'{o} ghavr\\
  \midrule
  your (pl.) mountain goat & s\'{o} camoch\\
  \midrule
  their doe & s\'{o} h\'{e}lan\\
  \midrule
  my aurochs & m\'{o} h\'{u}ru\\
  \midrule
  your deer & t\'{o} g\'{a}ru\\
  \midrule
  his water monster & \'{o} havanch\\
  \midrule
  her beaver & \'{o} bevr\\
  \midrule
  our dolphin & n\'{o} w\'{o}rchun\\
  \midrule
  your (pl.) salmon & s\'{o} anchr\'{a}i\\
  \midrule
  their heron & s\'{o} gharan\\
  \bottomrule
  \end{tabular}
}
\label{solution_possessive_pronouns}
\caption{Solution: possessive pronouns}
\end{table}

\newpage
\subsection{Gwepchoprith: Demonstratives}
\subsubsection{Demonstratives}

A demonstrative is a word that indicates another word: this, that, those, these.\\
In Gal\'{a}thach the demonstratives have two parts:
\begin{enumerate}
\item{the article \textit{in} before the word}
\item{the words \textit{sin} or \textit{s\'{e}} hyphenated to the end of the word}
\end{enumerate}

\begin{table}[H]
\centering
\begin{tabular}{cccc}
  \toprule
  \multicolumn{2}{c}{\textbf{Noun}} & \multicolumn{2}{c}{\textbf{Noun with demonstrative}}\\
  \midrule
  \textbf{Gal\'{a}thach} & \textbf{English} & \textbf{Gal\'{a}thach} & \textbf{English}\\
  \cmidrule(lr){1-1}\cmidrule(lr){2-2}\cmidrule(lr){3-3}\cmidrule(lr){4-4}
  cun & dog & in cun-sin & this dog\\
  cun & dog &  in cun-s\'{e} & that dog\\
  \bottomrule
\end{tabular}
\label{examples_demonstrative}
\end{table}

If the word is feminine the article causes mutation.
\begin{table}[H]
\centering
\begin{tabular}{cccc}
  \toprule
  \multicolumn{2}{c}{\textbf{Noun}} & \multicolumn{2}{c}{\textbf{Noun with demonstrative}}\\
  \midrule
  \textbf{Gal\'{a}thach} & \textbf{English} & \textbf{Gal\'{a}thach} & \textbf{English}\\
  \cmidrule(lr){1-1}\cmidrule(lr){2-2}\cmidrule(lr){3-3}\cmidrule(lr){4-4}
  cunis & bitch & in \textbf{g}unis-sin & this bitch\\
  cunis & bitch & in \textbf{g}unis-s\'{e} & that bitch\\
  \bottomrule
\end{tabular}
\label{examples_demonstrative_mutation}
\end{table}

If the word is plural the article and the demonstratives stay the same.
\begin{table}[H]
\centering
\begin{tabular}{cccc}
  \toprule
  \multicolumn{2}{c}{\textbf{Noun}} & \multicolumn{2}{c}{\textbf{Noun with demonstrative}}\\
  \midrule
  \textbf{Gal\'{a}thach} & \textbf{English} & \textbf{Gal\'{a}thach} & \textbf{English}\\
  \cmidrule(lr){1-1}\cmidrule(lr){2-2}\cmidrule(lr){3-3}\cmidrule(lr){4-4}
  cun & dog & in c\'{u}n\'{e}-sin & these dogs\\
  cunis & bitch & in gun\'{\i}s\'{e}-s\'{e} & those bitches\\
  \bottomrule
\end{tabular}
\label{examples_demonstrative_plural}
\end{table}

\newpage
\subsection{Exercises: Demonstratives}

\subsubsection{Translate}

Translate the following phrases.
\begin{table}[H]
\centering
\begin{tabular}{|c|M{5.0cm}|}
  \toprule
  \textbf{English} & \textbf{Gal\'{a}thach}\\
  \toprule
  this mare & \\
  \midrule
  that cow & \\
  \midrule
  these bulls & \\
  \midrule
  those sheep & \\
  \midrule
  this goat & \\
  \midrule
  that mountain goat & \\
  \midrule
  these does & \\
  \midrule
  those aurochses & \\
  \midrule
  this deer & \\
  \midrule
  that water monster & \\
  \midrule
  these beavers & \\
  \midrule
  those dolphins & \\
  \midrule
  this salmon & \\
  \midrule
  that heron & \\
  \bottomrule
\end{tabular}
\label{exercise_possessive_pronouns_plural}
\caption{Exercise: possessive pronouns plural}
\end{table}

\newpage
\subsubsection{Solution}

\begin{table}[H]
\centering
\rotatebox{180}{%
  \begin{tabular}{|c|>{\itshape}c|}
  \toprule
  \textbf{English} & \textbf{Gal\'{a}thach}\\
  \toprule
  this mare & in h\'{e}pis-sin\\
  \midrule
  that cow & in b\'{o}-s\'{e}\\
  \midrule
  these bulls & in tar\'{u}\'{e}-sin\\
  \midrule
  those sheep & in molth\'{e}-s\'{e}\\
  \midrule
  this goat & in ghavr-sin\\
  \midrule
  that mountain goat & in camoch-s\'{e}\\
  \midrule
  these does & in hel\'{a}n\'{e}-sin\\
  \midrule
  those aurochses & in hur\'{u}\'{e}-s\'{e}\\
  \midrule
  this deer & in c\'{a}ru-sin\\
  \midrule
  that water monster & in havanch-s\'{e}\\
  \midrule
  these beavers & in bevr\'{e}-sin\\
  \midrule
  those dolphins & in morch\'{u}n\'{e}-s\'{e}\\
  \midrule
  this salmon & in hanchr\'{a}i-sin\\
  \midrule
  that heron & in gharan-s\'{e}\\
  \bottomrule
  \end{tabular}
}
\label{solution_possessive_pronouns_plural}
\caption{Solution: possessive pronouns plural}
\end{table}

\subsection{Gwepchoprith: Locatives}
\subsubsection{Locatives}

Locatives are words that indicate a position: here, there.\\
There are two locatives in Gal\'{a}thach.
\begin{quote}
insin: here
ins\'{e}: there
\end{quote}

\begin{table}[H]
\centering
\begin{tabular}{cc}
  \toprule
  \textbf{Gal\'{a}thach} & \textbf{English}\\
  \cmidrule(lr){1-1}\cmidrule(lr){2-2}
  Esi in cun insin & the dog is here\\
  Esi in gunis ins\'{e} & the bitch is there\\
  Esi in cun-sin insin & this dog is here\\
  Esi in gunis-s\'{e} ins\'{e} & that bitch is there\\
  \bottomrule
\end{tabular}
\label{examples_locatives}
\end{table}

\newpage
\subsection{Exercises: Locatives}

\subsubsection{Translate}

Translate the following phrases.
\begin{table}[H]
\centering
\begin{tabular}{|c|M{5.0cm}|}
  \toprule
  \textbf{English} & \textbf{Gal\'{a}thach}\\
  \toprule
  the mare is here & \\
  \midrule
  the cows are there & \\
  \midrule
  this bull is here & \\
  \midrule
  that sheep is there & \\
  \midrule
  these goats are here & \\
  \midrule
  those mountain goats are there & \\
  \midrule
  the doe is here & \\
  \midrule
  the aurochses are there & \\
  \midrule
  this deer is here & \\
  \midrule
  that water monster is there & \\
  \midrule
  these beavers are here & \\
  \midrule
  those dolphins are there & \\
  \midrule
  this salmon is here & \\
  \midrule
  that heron is there & \\
  \bottomrule
\end{tabular}
\label{exercise_possessive_pronouns_locatives}
\caption{Exercise: possessive pronouns locatives}
\end{table}

\newpage
\subsubsection{Solution}

\begin{table}[H]
\centering
\rotatebox{180}{%
  \begin{tabular}{|c|>{\itshape}c|}
  \toprule
  \textbf{English} & \textbf{Gal\'{a}thach}\\
  \toprule
  the mare is here & esi in h\'{e}pis insin\\
  \midrule
  the cows are there & esi in b\'{o}\'{e} ins\'{e}\\
  \midrule
  this bull is here & esi in t\'{a}ru-sin insin\\
  \midrule
  that sheep is there & esi in molth-s\'{e} ins\'{e}\\
  \midrule
  these goats are here & esi in ghavr\'{e}-sin insin\\
  \midrule
  those mountain goats are there & esi in cam\'{o}ch\'{e}-s\'{e} ins\'{e}\\
  \midrule
  the doe is here & esi in h\'{e}lan insin\\
  \midrule
  the aurochses are there & esi in ur\'{u}\'{e} ins\'{e}\\
  \midrule
  this deer is here & esi in c\'{a}ru-sin insin\\
  \midrule
  that water monster is there & esi in havanch-s\'{e} ins\'{e}\\
  \midrule
  these beavers are here & esi in bevr\'{e}-sin insin\\
  \midrule
  those dolphins are there & esi in morch\'{u}n\'{e} ins\'{e}\\
  \midrule
  this salmon is here & esi in hanchr\'{a}i-sin insin\\
  \midrule
  that heron is there & esi in gharan-s\'{e} ins\'{e}\\
  \bottomrule
  \end{tabular}
}
\label{solution_possessive_pronouns_locatives}
\caption{Solution: possessive pronouns locatives}
\end{table}
