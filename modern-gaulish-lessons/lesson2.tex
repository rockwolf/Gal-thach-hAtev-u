\section{Menghavan 2: Gweran\'{u}\'{e} Donach Co hUrchatha}
(\textit{Lesson 2: Personal Pronouns As Object})\\

In the second lesson you will learn how to use a pronoun when it is the object of a phrase with a verb.

\subsection{Amch\'{a}nu \textendash\ Personal Pronouns}
\subsubsection{As Object Of An Active Verb}

The object of a sentence is the thing to which something is being done. It is the receiving end of the action performed by the verb.\\

When personal pronouns are the object of a sentence they can take two different forms. The first form is identical to the form they take when they are the subject of a sentence, except for one. The subject of a sentence is the giving end of the action performed by the verb.

\begin{table}[H]
\centering
\begin{tabular}{cccc}
  \toprule
  \multicolumn{2}{c}{\textbf{Personal pronouns as subject}} & \multicolumn{2}{c}{\textbf{Personal pronouns as object}}\\
  \midrule
  \textbf{Gal\'{a}thach} & \textbf{English} & \textbf{Gal\'{a}thach} & \textbf{English}\\
  \cmidrule(lr){1-1}\cmidrule(lr){2-2}\cmidrule(lr){3-3}\cmidrule(lr){4-4}
  mi & I & mi & me\\
  ti & you & ti & you\\
  \'{e} & he & \'{e} & him\\
  \'{\i} & she & \'{\i} & her\\
  \'{\i} & it & \'{\i} & it\\
  ni & we & ni & us\\
  s\'{u} & you (pl.) & s\'{u} & you (pl.)\\
  \'{\i}s & they & \'{\i}s & them\\
  \bottomrule
\end{tabular}
\caption{Personal pronouns, as subject vs.\ as object}
\label{personal_pronouns_subject_object}
\end{table}

Only the third person plural pronoun differs: \'{\i}s instead of s\'{\i}.\\

These pronouns are used when they are the receiving end of an active verb. An active verb is a verb that performs the main action of a phrase. It will have a subject which will be performing the action. It will be in a form that indicates the time and the way the action is being performed.\\

V = verb\\
S = subject\\
O = object\\

In modern Gaulish a phrase has the standard order Verb-Subject-Object order. It is an aspect that is characteristic of the Celtic languages and is not common in English.\\

Using the verbs that were introduced in lesson 1 we can construct examples:\\
\begin{table}[H]
\centering
\begin{tabular}{ccc}
    \begin{tabular}{cc}
    \toprule
    \textbf{Gal\'{a}thach} & \textbf{English}\\
    \cmidrule(lr){1-1}\cmidrule(lr){2-2}
    ap\'{\i}sa mi & I see\\
    \bottomrule
    \end{tabular}
    &
    \begin{tabular}{c}\\
    $\rightarrow$
    \end{tabular}
    &
    \begin{tabular}{cc}
    \toprule
    \textbf{Gal\'{a}thach} & \textbf{English}\\
    \cmidrule(lr){1-1}\cmidrule(lr){2-2}
    ap\'{\i}sa mi ti & I see you\\
    \bottomrule
    \end{tabular}
\end{tabular}
\label{examples_verbs_vso}
\end{table}

In the phrase \textit{ap\'{\i}sa mi ti} the verb \textit{ap\'{\i}sa} comes first, the subject \textit{mi} comes second, and the object \textit{ti} comes third. This is indicated like this:\\

\begin{table}[H]
\begin{tabular}{ccc}
ap\'{\i}sa & mi & ti\\
V & S & O
\end{tabular}
\label{examples_verbs_vso_indication}
\end{table}

We can see that the verb \textit{ap\'{\i}sa} is an active verb because it is in the present tense: it has the present tense ending \textit{-a}.\\

Here are more examples:
\begin{table}[H]
\centering
\begin{tabular}{cc}
  \toprule
  \textbf{Gal\'{a}thach} & \textbf{English}\\
  \cmidrule(lr){1-1}\cmidrule(lr){2-2}
  g\'{a}ra \'{\i} mi & she calls me\\
  delgha \'{e} ni & he holds us\\
  gn\'{\i}a s\'{\i} s\'{u} & they know you (pl.)\\
  \bottomrule
\end{tabular}
\label{examples_verbs_vso_more_examples}
\end{table}

When the object pronoun starts with a vowel, such as \'{e}, \'{\i} and \'{\i}s, and they follow a subject pronoun, that object pronoun receives an extra letter \textit{ch-} at the start. This letter ch is pronounced like the -ch in the Scottish word ``loch''.
\begin{table}[H]
\centering
\begin{tabular}{cc}
  \toprule
  \textbf{Gal\'{a}thach} & \textbf{English}\\
  \cmidrule(lr){1-1}\cmidrule(lr){2-2}
  ap\'{\i}sa mi ch\'{\i} & I see her\\
  m\'{e}na mi ch\'{\i} & I think it\\
  gw\'{e}la \'{\i} ch\'{e} & she wants him\\
  \'{a}va \'{e} ch\'{\i} & he does it\\
  gn\'{\i}a s\'{u} ch\'{\i}s & you (pl.) know them\\
  g\'{a}ra \'{\i} ch\'{\i}s & she calls them\\
  \bottomrule
\end{tabular}
\label{examples_verbs_vso_extra_ch}
\end{table}

\subsubsection{As Object Of A Verbal Noun}

The verbal noun is the basic root form of the verb, called infinitive in English.\\

It is easiest to think of the verbal noun of modern Gaulish as the -ing form of the English verb.\\

E.g.\ can: to sing $\rightarrow$ can: singing\\
c\'{a}na mi ch\'{\i}: I sing it\\

When the personal pronouns are the object of a verbal noun, they take on a different form:
\begin{table}[H]
\centering
\begin{tabular}{cc}
  \toprule
  \textbf{Standard} & \textbf{As object of verbal noun}\\
  \cmidrule(lr){1-1}\cmidrule(lr){2-2}
  mi & im\'{\i}\\
  ti & ith\'{\i}\\
  \'{e} & ich\'{e}\\
  \'{\i} & ich\'{\i}\\
  ni & in\'{\i}\\
  s\'{u} & is\'{u}\\
  \'{\i}s & ich\'{\i}s\\
  \bottomrule
\end{tabular}
\caption{Personal pronouns, as object of verbal noun}
\label{personal_pronouns_as_object_of_verbal_noun}
\end{table}

When a verbal noun is used in a phrase with an active verb it comes \textit{immediately after the subject}:\\

gw\'{e}la mi \textit{can}: I want \textit{to sing}\\

In this phrase the verbal noun is the object of the active verb:
\begin{table}[H]
\begin{tabular}{ccc}
  gw\'{e}la & mi & can\\
  V & S & O
\end{tabular}
\label{verbal_noun_object_of_active_verb}
\end{table}

If we think of the verbal noun as the -ing form of the verb, we could literally translate this as:
\begin{table}[H]
\begin{tabular}{cccc}
  want & I & singing & ($\rightarrow$ ``I want singing'')\\
  V & S & O &
\end{tabular}
\label{verbal_noun_ing_form}
\end{table}

If we use a personal pronoun to be the object of the verbal noun we use the special form described above:\\
gw\'{e}la mi can ich\'{\i}: I want to sing it\\

In this phrase the two words \textit{can ich\'{\i}} become the new object of the phrase.
\begin{table}[H]
\begin{tabular}{ccc}
    gw\'{e}la & mi & can ich\'{\i}\\
    V & S & $[$\hspace{0.5cm}O\hspace{0.5cm}$]$
\end{tabular}
\label{personal_pronoun_as_object_of_verbal_noun}
\end{table}

The above phrase can be literally translated as \textit{I want singing of-it}.\\

The particle i- that the pronouns are attached to indicates possession of something:
\begin{table}[H]
\centering
\begin{tabular}{cc}
  \toprule
  \textbf{Gal\'{a}thach} & \textbf{English}\\
  \cmidrule(lr){1-1}\cmidrule(lr){2-2}
  im\'{\i} & of-me\\
  ith\'{\i} & of-you\\
  ich\'{e} & of-him\\
  ich\'{\i} & of-her\\
  ich\'{\i} & of-it\\
  in\'{\i} & of-us\\
  is\'{u} & of-you (pl.)\\
  ich\'{\i}s & of-them\\
  \bottomrule
\end{tabular}
\label{personal_pronoun_particle_i}
\end{table}

The phrase \textit{can ich\'{\i}} translates as \textit{singing of-it}. If we add an imaginary definite article $[$the$]$ to the English version it makes sense:\\
\begin{table}[H]
\centering
\begin{tabular}{ccc}
\begin{tabular}{cc}
\toprule
\textbf{Gal\'{a}thach} & \textbf{English}\\
\cmidrule(lr){1-1}\cmidrule(lr){2-2}
can ich\'{\i} & $[$the$]$ singing of-it\\
\bottomrule
\end{tabular}
&
\begin{tabular}{c}
$\rightarrow$
\end{tabular}
&
\begin{tabular}{cc}
\toprule
\textbf{Gal\'{a}thach} & \textbf{English}\\
\cmidrule(lr){1-1}\cmidrule(lr){2-2}
gw\'{e}la mi can ich\'{\i} & I want $[$the$]$ singing of-it\\
\bottomrule
\end{tabular}
\end{tabular}
\end{table}

\newpage
\subsection{Exercises - Personal Pronouns As Object Of An Active Verb}

\subsubsection{Construct phrases}

Construct the following phrases with any of the following verbs:\\

\begin{quote}
prin (to buy), ber (to carry), br\'{\i}s (to break), \'{\i}vi (to drink), c\'{a}ra (to love), \'{a}v\'{o} (to do, to make), camwi (to bend, to curve), l\'{\i} (to lie down), gn\'{\i} (to know), ap\'{\i}s (to see), gw\'{e}l (to want), g\'{a}ra (to call)
\end{quote}

\begin{table}[H]
\centering
\begin{tabular}{|l|M{10.0cm}|}
  \toprule
  \textbf{Phrase (English)} & \textbf{Answer (Gal\'{a}thach)}\\
  \toprule
  I buy it & \\
  \midrule
  you carry him & \\
  \midrule
  he breaks it & \\
  \midrule
  she drinks it & \\
  \midrule
  we love them & \\
  \midrule
  they call you (pl.) & \\
  \midrule
  you (pl.) bend us & \\
  \midrule
  she sees me & \\
  \midrule
  he knows her & \\
  \midrule
  she wants you & \\
  \bottomrule
\end{tabular}
\label{exercise_phrases_with_verbs}
\caption{Exercise: phrases with verbs}
\end{table}

\newpage
\subsubsection{Solution}
\begin{table}[H]
\centering
\rotatebox{180}{%
  \begin{tabular}{|l|>{\itshape}c|}
    \toprule
    \textbf{Phrase (English)} & \textbf{Answer (Gal\'{a}thach)}\\
    \toprule
    I buy it & pr\'{\i}na mi ch\'{\i}\\
    \midrule
    you carry him & b\'{e}ra ti ch\'{e}\\
    \midrule
    he breaks it & brisa \'{e} ch\'{\i}\\
    \midrule
    she drinks it & \'{\i}va \'{\i} ch\'{\i}\\
    \midrule
    we love them & c\'{a}ra ni ch\'{\i}s\\
    \midrule
    they call you (pl.) & g\'{a}ra s\'{\i} s\'{u}\\
    \midrule
    you (pl.) bend us & camw\'{\i}a s\'{u} ni\\
    \midrule
    she sees me & ap\'{\i}sa \'{\i} mi\\
    \midrule
    he knows her & gn\'{\i}a \'{e} ch\'{\i}\\
    \midrule
    she wants you & gw\'{e}la \'{\i} ti\\
    \bottomrule
  \end{tabular}
}
\label{solution_phrases_with_verbs}
\caption{Solution: phrases with verbs}
\end{table}

\newpage
\subsection{Exercises - Personal Pronouns As Object Of A Verbal Noun}

\subsubsection{Make phrases}

Make the following phrases, using the verbs given above:
\begin{table}[H]
\centering
\begin{tabular}{|l|M{10.0cm}|}
  \toprule
  \textbf{Phrase (English)} & \textbf{Answer (Gal\'{a}thach)}\\
  \midrule
  I want to see it & \\
  \midrule
  you want to hold her & \\
  \midrule
  he wants to know you & \\
  \midrule
  she wants to love him & \\
  \midrule
  it can break me & \\
  \midrule
  we can buy them & \\
  \midrule
  you (pl.) can carry us & \\
  \midrule
  they can know you (pl.) & \\
  \midrule
  you (pl.) can do it & \\
  \bottomrule
\end{tabular}
\label{exercise_attached_pronouns_indicating_possession}
\caption{Exercise: attached pronouns, indicating possession}
\end{table}

\newpage
\subsubsection{Solution}
\begin{table}[H]
\centering
\rotatebox{180}{%
  \begin{tabular}{|l|>{\itshape}c|}
    \toprule
    \textbf{Phrase (English)} & \textbf{Answer (Gal\'{a}thach)}\\
    \toprule
    I want to see it & gw\'{e}la mi \'{a}pis ich\'{\i}\\
    \midrule
    you want to hold her & gw\'{e}la ti delghe ich\'{\i}\\
    \midrule
    he wants to know you & gw\'{e}la \'{e} gn\'{\i} ith\'{\i}\\
    \midrule
    she wants to love him & gw\'{e}la \'{\i} c\'{a}ra ich\'{e}\\
    \midrule
    it can break me & g\'{a}la \'{\i} br\'{\i}s im\'{\i}\\
    \midrule
    we can buy them & g\'{a}la ni prin ich\'{\i}s\\
    \midrule
    you (pl.) can carry us & g\'{a}la s\'{u} ber in\'{\i}\\
    \midrule
    they can know you (pl.) & g\'{a}la s\'{\i} gn\'{\i} is\'{u}\\
    \midrule
    you (pl.) can do it & g\'{a}la s\'{u} \'{a}v\'{o} ich\'{\i}\\
    \bottomrule
  \end{tabular}
}
\label{solution_attached_pronouns_indicating_possession}
\caption{Solution: attached pronouns, indicating possession}
\end{table}
