\section{Menghavan 14: In Goghn\'{i}t Vr\'{e}trach}
(\textit{Lesson 14: The Verbal System})\\

In the fourteenth lesson, you will learn how the verbal system works in Gal\'{a}thach.

\subsection{Gwepchoprith: Conversation}
\subsubsection{Conversation}

Below is a conversation between two people. Gwirchan is a man and Benghal is a woman. Both are attested Gaulish names. The conversation shows you how the verbal system works.

%\tcbox[colback=red!10!green!10, colframe=green!20!black!80]
\begin{table}[H]
\centering
    \begin{tabular}{cM{10.0cm}c}
    \cellcolor{lightgreen} & \cellcolor{lightgreen} & \cellcolor{lightgreen}\\
    \cellcolor{lightgreen}\textcolor{darkgreen}{\textbf{Tarchonwoth}} & \cellcolor{lightgreen} & \cellcolor{lightgreen}\textcolor{darkgreen}{\textbf{Tuthch\'{a}na}}\\
    \cellcolor{lightgreen} & \cellcolor{lightgreen} & \cellcolor{lightgreen}\\
    \cellcolor{lightgreen} & \cellcolor{lightgreen} & \cellcolor{lightgreen}\\
    \cellcolor{lightgreen} & \cellcolor{lightgreen} & \cellcolor{lightgreen}\\
    \cellcolor{lightgreen} & \cellcolor{lightgreen} & \cellcolor{lightgreen}\\
    \cellcolor{lightgreen} & \cellcolor{lightgreen} & \cellcolor{lightgreen}\\
    \cellcolor{lightgreen} & \cellcolor{lightgreen} & \cellcolor{lightgreen}\\
%    \cellcolor{lightgreen}\multirow{-7}{*}{\includegraphics[height=4.0cm]{img/menghavan11_1}} & \cellcolor{lightgreen} & \cellcolor{lightgreen}\multirow{-7}{*}{\includegraphics[height=4.0cm]{img/menghavan11_2}}\\
    \cellcolor{lightgreen} & \cellcolor{lightgreen} & \cellcolor{lightgreen}
    \end{tabular}
\end{table}

\begingroup
\fontsize{10pt}{12pt}\selectfont
\begin{leftbubbles}TBD\end{leftbubbles}
\begin{rightbubbles}TBD\end{rightbubbles}
\endgroup

\newpage
\subsubsection{Colav\'{a}ru \textendash\ Tr\'{e}lav\'{a}ru}
(Conversation \textendash\ Translation)

Gwirchan: \'{E}i geneth, d\'{\i}\'{a}i insin! 
(Gwirchan: Hey girl, come here!)
Benghal: N\'{e} chw\'{e}la mi d\'{\i}\'{a}i insin.
(Benghal: I don't want to come here.)
Gwirchan: M\'{e}na mi o gwels\'{\i} ti \'{a}pis sin.
(Gwirchan: I think that you will want to see this.)
Benghal: Menth\'{u} ti p\'{e}th\'{e} \'{e}lu cin shin.
(Benghal: You have thought a lot of things before this.)
Gwirchan: Gars\'{\i}thu mi ti \'{\i}\'{o}n\'{e} \'{e}lu cin gl\'{u}ia ti.
(Gwirchan: I will have called you a lot of times before you hear.)
Benghal: R\'{e} gl\'{u}i mi d\'{\i}es, ach n\'{e} rh\'{e} ‘p\'{a} ti neveth.
(Benghal: I heard yesterday, and you didn’t say anything.)
Gwirchan: R\'{e} shedhis\'{\i} mi insin en ghar ri haman sh\'{\i}r cin rh\'{e} rheth\'{u} ti insin.
(Gwirchan: I would sit here calling for a long time before you had run here.)
Benghal: Esi s\'{e} riveth n\'{e} rh\'{e} chwel mi r\'{e}thi ins\'{e}. P\'{e}ri a rh\'{e} hav\'{o}s\'{\i}thu mi s\'{e}?
(Benghal: That’s because I didn’t want to run there. Why would I have done that?)
Gwirchan: N\'{e} sh\'{e}th\'{e} duchis. Och apis\'{\i} ti in sudherch-sin tech. Derchi ni sin rochan!
(Gwirchan: Don't be stupid. May you see this beautiful view. Let us look at it together!)
Benghal: D\'{a}i, d\'{a}i, esi mi en dh\'{\i}\'{a}i. Delghe to dh\'{u}an. P\'{e} a hesi \'{\i} o gw\'{e}la ti och ap\'{\i}sa mi?
(Benghal: Good, good, I'm coming. Hold your piss. What is it that you want me to see?)
Gwirchan: Derchi ins\'{e}. Esi in Lithau en h\'{a}na.
(Gwirchan: Look at this. The Earth is breathing.)

names: 

Gwirchan @ Songman (< Uirocantus ``song-man'', Delamarre 2003, p. 321)
Benghal @ Powerwoman (< Banogalis ``power-woman'', Delamarre 2003, p. 72)

\subsection{Gwepchoprith: The verbal system}
\subsubsection{The verbal root/infinitive – the imperative}

The verbal root or infinitive is the basic root form of the verb. It doesn’t have any prefixes or suffixes. 

d\'{\i}\'{a}i: to come
gwel: to want

The imperative (giving a command) is the same as the verbal root. It is said in a commanding tone.

d\'{\i}\'{a}i insin! : come here!

There are seven different kinds of verbal roots, depending on their ending.

a) ending on -n, -r or -l

men: to think
gar: to call
gwel: to want

b) ending on a vowel + -i

d\'{\i}\'{a}i: to go
cl\'{u}i: to hear

c) ending on a consonant + -i

r\'{e}thi: to run
derchi: to look/to watch
s\'{e}dhi: to sit

d) ending on -a

sp\'{a}: to say
\'{a}na: to breathe

e) ending on –e

delghe: to hold

f) ending on -o

\'{a}v\'{o}: to do

g) ending on -s

\'{a}pis: to see


\subsubsection{The present form}

The present form is made by giving the verbal root a suffix (a bit added at the end). This suffix or ending is –a. 

a) If the verbal root ends in -n, -r, -l or -s this ending –a is added at the end.

gwel: to want
> gw\'{e}la mi: I want

men: to think
> m\'{e}na mi: I think

gar: to call
> g\'{a}ra mi: I call

\'{a}pis: to see
> apisa mi: I see

b) If the verbal root ends in a vowel + -i the ending –a is added at the end.

d\'{\i}\'{a}i: to come
> di\'{a}ia mi: I come

cl\'{u}i: to hear
> cl\'{u}ia mi: I hear

c) If the verbal root ends in –a it doesn’t change.

sp\'{a}: to say.
> sp\'{a} mi: I say

d) If the verbal root ends in a consonant + -i, -e or –o that –i, -e or –o is changed into an –a.

r\'{e}thi: to run
> r\'{e}tha mi: I run

delghe: to hold
> delgha mi: I hold

\'{a}v\'{o}: to do
> \'{a}va mi: I do

\subsubsection{The ongoing form}

The ongoing form is made by using the preposition “en”. It is put in front of the verbal root. It causes an initial consonant mutation on the verbal root.

\'{a}na: to breathe
> esi in Lithau en h\'{a}na: the Earth is breathing.

\subsubsection{The simple past form}

The simple past form is made by using the particle “r\'{e}”. It is put in front of the verbal root. It causes an initial consonant mutation on the verbal root.

cl\'{u}i: to hear
> r\'{e} gl\'{u}i mi: I heard

sp\'{a}: to say
> r\'{e} ‘p\'{a} ti neveth: you said nothing

\subsubsection{The completed past form}

The completed past form is made by giving the verbal root a suffix (an ending). 

a) verbal roots on n, r, l, a, e, o, and i not including –thi and -dhi

For these verbal roots the ending is –thu.

men > menthu mi: I have thought
gar > garthu mi: I have called
gwel > gwelthu mi: I have wanted
sp\'{a} > spathu mi: I have said
delghe > delgh\'{e}thu mi: I have held
\'{a}v\'{o} > av\'{o}thu mi: I have done
derchi > derch\'{\i}thu mi: I have looked

b) verbal roots on –thi, -dhi and -s

For these verbal roots the ending –i is changed into –\'{u}. This –\'{u} receives the emphasis.

s\'{e}dhi > sedh\'{u} mi: I have sat
r\'{e}thi > reth\'{u} mi: I have run
\'{a}pis > apis\'{u} mi: I have seen

6. The Simple Future Form

The simple future form is made by giving the verbal root a suffix (an ending).

a) all verbal roots except verbal roots on -s

For these verbal roots the ending is –s\'{\i}. The –\'{\i} receives the mephasis.

men > mens\'{\i} mi: I will think
gar > gars\'{\i} mi: I will call
gwel > gwels\'{\i} mi: I will want
sp\'{a} > spas\'{\i} mi: I will speak
delghe > delghes\'{\i} mi: I will hold
\'{a}v\'{o} > \'{a}vos\'{\i} mi: I will do
derchi > derchis\'{\i} mi: I will see
s\'{e}dhi > sedhis\'{\i} mi: I will sit
r\'{e}thi > rethis\'{\i} mi: I will run

b) verbal roots on -s

For these verbal roots the ending is –\'{\i}. The –\'{\i} receives the emphasis.

\'{a}pis > apis\'{\i} mi: I will see.

\subsubsection{The conditional future form}

The conditional future form is made by putting the particle “r\'{e}” in front of the simple future form. This particle “r\'{e}” causes initial consonant mutation on the verbal root.

men > r\'{e} wens\'{\i} mi: I would think
sp\'{a} > r\'{e} ‘pas\'{\i} mi: I would say
derchi > r\'{e} dherchis\'{\i} mi: I would look
\'{a}pis > r\'{e} hapis\'{\i} mi: I would see

\subsubsection{The completed simple past form}

The completed simple past form is made by putting the particle ‘r\'{e}” in front of the completed past form. This particle “r\'{e}” causes initial consonant mutation on the verbal root.

gar > r\'{e} gharthu mi: I had called
delghe > r\'{e} dhelgh\'{e}thu mi: I had held
sedhi > r\'{e} shedh\'{u} mi: I had sat

\subsubsection{The completed simple future form}

The completed simple future form is made by adding the ending –thu to the simple future form.

gwel > gwels\'{\i}thu mi: I will have wanted
\'{a}v\'{o} > \'{a}vos\'{\i}thu mi: I will have done
r\'{e}thi > rethis\'{\i}thu mi: I will have run

\subsubsection{The completed conditional future form}

The completed conditional form is made by putting the particle “r\'{e}” in front of the completed simple future form.

> r\'{e} chwelsithu mi: I would have wanted
> r\'{e} havos\'{\i}thu mi: I would have done
> r\'{e} rhethis\'{\i}thu mi: I would have run

\subsubsection{Ongoing past, future and conditional forms}

The ongoing form can be put in all the forms by putting the verb “to be”in all the forms.
> b\'{u} mi en h\'{a}na: I was breathing
> b\'{\i} mi en h\'{a}na: I will be breathing
> \'{e}thu mi en h\'{a}na: I have been breathing
> bi\'{e}thu mi en h\'{a}na: I will have been breathing
> r\'{e} v\'{\i} mi en h\'{a}na: I would be breathing
> r\'{e} h\'{e}thu mi en h\'{a}na: I had been breathing
> r\'{e} vi\'{e}thu mi en h\'{a}na: I would have been breathing

\subsubsection{The \textit{may} form}

The \textit{may} form expresses a wish. It is made by putting the particle “o” (“that”) in front of the simple future form.

\'{a}pis: to see
> apis\'{\i} ti: you will see
> och apis\'{\i} ti: may you see

It can also be used with the completed future form.

> och apis\'{\i}thu ti: may you have seen

\subsubsection{The \textit{let us} form}

The \textit{let us} form indicates a command to a group of people including the speaker. It is made by using the imperative form (command) with the personal pronoun \textit{ni} (we).

derchi: to look
> derchi!: look!
> derchi ni!: let us look!

\subsubsection{The indirect command form}

The indirect command form expresses a command indirectly. It is constructed like a subordinate clause with “o” (“that”).

gw\'{e}la ti: you want
ap\'{\i}sa mi: I see
> gw\'{e}la ti och ap\'{\i}sa mi: you want that I see (= you want me to see)

\subsubsection{The impersonal form}

The impersonal form is made by adding the ending –or to the end of the verbal root. It means a verb acts without a subject in a general way.

men: to think
> menor: one thinks / it is thought 

rinchi: to need
> rinchor: one needs / it is needed > “it is necessary”

The impersonal form can be combined with all the forms of the verbs.

> r\'{e} rhinchor: it was necessary
> rinchorthu: it has been necessary
> r\'{e} rhinchorthu: it had been necessary
> rinchors\'{\i}: it will be necessary
> rinchors\'{\i}thu: it will have been necessary
> r\'{e} rhinchors\'{\i}: it would be necessary
> r\'{e} rhinchors\'{\i}thu: it would have been necessary
> rinchor!: be necessary!/ let it be necessary
> o rinchors\'{\i}: may it be necessary
> gw\'{e}la mi o rinchor: I want it to be necessary


\subsection{Excercises}

\subsubsection{Vocabulary}
Use the verbs given below in the exercises, as well as the ones learned in previous lessons.

trughni: to snore
ausi: to listen
to shout: druchar
to fall: c\'{o}imi
to cry: duiar
to laugh: suiar
carni: to build
dicharni: to destroy
dwa\'{e}li: to fart
t\'{a}i: to touch
croswi: to surf
depri: to eat
s\'{o}ni: to sleep
to kiss: buswi
to feel: menth\'{a}i
to walk: c\'{a}ma 
to climb: dres 
to grow: m\'{a}ri 
to celebrate: l\'{\i}thi
to happen: gw\'{e}ri

\subsubsection{Translate}

Translate the following phrases. 

Go there!: 
Listen to me!: 
I speak:
You say:
He is shouting:
She is snoring:
We fell:
You (pl.) sang:
They farted:
They touched:
I have built:
You have destroyed:
He will cry:
She will laugh:
We would see:
You (pl.) would hear:
They had surfed:
I had eaten:
You will have drunken:
He will have slept:
She would have kissed:
We would have felt: 
You (pl.) will have been walking:
They would have been climbing:
May you have grown: 
Let us celebrate!:
They want us to leave:
It would have happened:


Answers:

Go there!: \'{a}i ins\'{e}!
Listen to me!: ausi adhim!
I speak: lav\'{a}ra mi
You say: sp\'{a} ti
He is shouting: esi \'{e} en dhruchar
She is snoring: esi \'{\i} en drughni
We fell: r\'{e} g\'{o}imi ni
You (pl.) sang: r\'{e} gan s\'{u}
They farted: r\'{e} dhwa\'{e}li s\'{\i}
They touched: r\'{e} wenth\'{a}i s\'{\i}
I have built: carn\'{\i}thu mi
You have destroyed: dicharn\'{\i}thu ti
He will cry: duiars\'{\i} \'{e}
She will laugh: suiars\'{\i} \'{\i}
We would see: r\'{e} hapis\'{\i} ni
You (pl.) would hear: r\'{e} gl\'{u}is\'{\i} s\'{u}
They had surfed: r\'{e} grosw\'{\i}thu s\'{\i}
I had eaten: r\'{e} dhepr\'{\i}thu mi
You will have drunken: ivis\'{\i}thu ti
He will have slept: sonis\'{\i}thu \'{e}
She would have kissed: r\'{e} vusw\'{\i}thu \'{\i}
We would have felt: r\'{e} wenth\'{a}isithu ni
You (pl.) will have been walking: bi\'{e}thu s\'{u} en g\'{a}ma
They would have been climbing: r\'{e} vi\'{e}thu s\'{\i} en dhres
May you have grown: o mars\'{\i}thu ti
Let us celebrate!: l\'{\i}thi ni!
They want us to leave: gw\'{e}la s\'{\i} o t\'{e}cha ni
It would have happened: r\'{e} chwerors\'{\i}thu
