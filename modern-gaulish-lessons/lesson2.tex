\section{Menghavan 2: Gweran\'{u}\'{e} Donach Co hUrchatha}
(\textit{Lesson 2: Personal Pronouns As Object})\\

\noindent In the second lesson you will learn how to use a pronoun when it is the object of a phrase with a verb.
\subsection{Personal pronouns as object of an active verb}

\noindent The object of a sentence is the thing to which something is being done. It is the receiving end of the action performed by the verb.\\

\noindent When personal pronouns are the object of a sentence they can take two different forms. The first form is identical to the form they take when they are the subject of a sentence, except for one. The subject of a sentence is the giving end of the action performed by the verb.

\begin{table}[H]
\begin{center}
\begin{tabu}{c|c|c|c}
  \multicolumn{2}{c|}{\textbf{Personal pronouns as subject}} & \multicolumn{2}{c}{\textbf{Personal pronouns as object}}\\
  \toprule
  \textbf{Gal\'{a}thach} & \textbf{English} & \textbf{Gal\'{a}thach} & \textbf{English}\\ 
  \toprule
  mi & I & mi & me\\
  ti & you & ti & you\\
  \'{e} & he & \'{e} & him\\
  \'{\i} & she & \'{\i} & her\\
  \'{\i} & it & \'{\i} & it\\
  ni & we & ni & us\\
  s\'{u} & you (pl.) & s\'{u} & you (pl.)\\
  \'{\i}s & they & \'{\i}s & them\\
\end{tabu}
\end{center}
\caption{Personal pronouns, as subject vs.\ as object}
\label{personal_pronouns_subject_object}
\end{table}

\noindent Only the third person plural pronoun differs: \'{\i}s instead of s\'{\i}.\\

\noindent These pronouns are used when they are the receiving end of an active verb. An active verb is a verb that performs the main action of a phrase. It will have a subject which will be performing the action. It will be in a form that indicates the time and the way the action is being performed.\\

\noindent V = verb\\
\noindent S = subject\\
\noindent O = object\\

\noindent In modern Gaulish a phrase has the standard order Verb-Subject-Object order. It is an aspect that is characteristic of the Celtic languages and is not common in English.\\

\noindent Using the verbs that were introduced in lesson 1 we can construct examples:\\
\begin{table}[H]
\begin{center}
\begin{tabu}{ccc}
    \begin{tabu}{c|c}
    Gal\'{a}thach & English\\
    \toprule
    ap\'{\i}sa mi & I see
    \end{tabu}
    &
    \begin{tabu}{c}
    \\$\rightarrow$ 
    \end{tabu}
    &
    \begin{tabu}{c|c}
    Gal\'{a}thach & English\\
    \toprule
    ap\'{\i}sa mi ti & I see you
    \end{tabu}
\end{tabu}
\end{center}
\label{examples_verbs_vso}
\end{table}

\noindent In the phrase \textit{ap\'{\i}sa mi ti} the verb \textit{ap\'{\i}sa} comes first, the subject \textit{mi} comes second, and the object \textit{ti} comes third. This is indicated like this:\\

\begin{table}[H]
\begin{tabu}{ccc}
ap\'{\i}sa & mi & ti\\
V & S & O
\end{tabu}
\label{examples_verbs_vso_indication}
\end{table}

\noindent We can see that the verb \textit{ap\'{\i}sa} is an active verb because it is in the present tense: it has the present tense ending \textit{-a}.\\

\noindent Here are more examples:
\begin{table}[H]
\begin{center}
\begin{tabu}{c|c}
  \textbf{Gal\'{a}thach} & \textbf{English}\\
  \toprule
%\noindent gára í mi: she calls me
%\noindent delgha é ni: he holds us
%\noindent gnía sí sú: they know you (pl.)
\end{tabu}
\end{center}
\label{examples_verbs_vso_more_examples}
\end{table}

\noindent When the object pronoun starts with a vowel, such as \'{e}, \'{\i} and \'{\i}s, and they follow a subject pronoun, that object pronoun receives an extra letter \textit{ch-} at the start. This letter ch is pronounced like the -ch in the Scottish word ``loch''.
\begin{table}[H]
\begin{center}
\begin{tabu}{c|c}
  \textbf{Gal\'{a}thach} & \textbf{English}\\
  \toprule
\end{tabu}
\end{center}
\label{examples_verbs_vso_extra_ch}
\end{table}
%\noindent apísa mi chí: I see her
%\noindent ména mi chí: I think it
%\noindent gwéla í ché: she wants him
%\noindent áva é chí: he does it
%\noindent gnía sú chís: you (pl.) know them
%\noindent gára í chís: she calls them
%\noindent <strong>Exercises</strong>
%\noindent Construct the following phrases with the verbs given above and underneath:
%\noindent prin (to buy)
%\noindent ber (to carry)
%\noindent brís (to break)
%\noindent ívi (to drink)
%\noindent cára (to love)
%\noindent ávó (to do, to make)
%\noindent camwi (to bend, to curve)
%\noindent lí (to lie down)
%\noindent I buy it:
%\noindent you carry him:
%\noindent he breaks it:
%\noindent she drinks it:
%\noindent we love them:
%\noindent you (pl.) bend us:
%\noindent they call you (pl.):
%\noindent she sees me:
%\noindent he knows her:
%\noindent she wants you:
%\noindent You can check your answers on the last page of this lesson.
%<ol start="2">
%<li><strong> Personal pronouns as object of a verbal noun</strong></li>
%</ol>
%\noindent The verbal noun is the basic root form of the verb, called infinitive in English.
%\noindent It is easiest to think of the verbal noun of modern Gaulish as the –ing form of the English verb.
%\noindent E.g. can: to sing &gt; can: “singing”
%\noindent cána mi chí: I sing it
%\noindent When the personal pronouns are the object of a verbal noun, they take on a different form:
%\noindent mi &gt; imí
%\noindent ti &gt; ithí
%\noindent é &gt; iché
%\noindent í &gt; ichí
%\noindent ni &gt; iní
%\noindent sú &gt; isú
%\noindent ís &gt; ichís
%\noindent When a verbal noun is used in a phrase with an active verb it comes immediately after the subject:
%\noindent gwéla mi can: I want to sing
%\noindent In this phrase the verbal noun is the object of the active verb:
%\noindent gwéla mi can
%\noindent V        S    O
%\noindent If we think of the verbal noun as the –ing form of the verb, we could literally translate this as:
%\noindent want I singing (&gt; “I want singing”)
%\noindent V      S   O
%\noindent If we use a personal pronoun to be the object of the verbal noun we use the special form described above:
%\noindent gwéla mi can ichí: I want to sing it
%\noindent In this phrase the two words “can ichí” become the new object of the phrase.
%\noindent gwéla mi can ichí
%\noindent V       S    [O       ]
%\noindent The above phrase can be literally translated as “I want singing of-it”.
%\noindent The particle i- that the pronouns are attached to indicates possession of something:
%\noindent imí: of-me
%\noindent ithí: of-you
%\noindent iché: of-him
%\noindent ichí: of-her
%\noindent ichí: of-it
%\noindent iní: of-us
%\noindent isú: of-you pl.
%\noindent ichís: of-them
%\noindent The phrase “can ichí” translates as “singing of-it”. If we add an imaginary definite article [the] to the English version it makes sense:
%\noindent can ichí: [the] singing of-it &gt; gwéla mi can ichí: I want [the] singing of-it
%\noindent <strong>Exercises</strong>
%\noindent Make the following phrases, using the verbs given above:
%\noindent I want to see it:
%\noindent you want to hold her:
%\noindent he wants to know you:
%\noindent she wants to love him:
%\noindent it can break me:
%\noindent we can buy them:
%\noindent you (pl.) can carry us:
%\noindent they can know you (pl.):
%\noindent you (pl.) can do it:
%\noindent You can check your answers below.
%\noindent <strong>Answers</strong>
%\noindent Exercises 1
%\noindent I buy it: prína mi chí
%\noindent you carry him: béra ti ché
%\noindent he breaks it: brisa é chí
%\noindent she drinks it: íva í chí
%\noindent we love them: cára ni chís
%\noindent you (pl.) bend us: camwía sú ni
%\noindent they call you (pl.): gára sí sú
%\noindent she sees me: apísa í mi
%\noindent he knows her: gnía é chí
%\noindent she wants you: gwéla í ti
%\noindent Exercises 2
%\noindent I want to see it: gwéla mi ápis ichí
%\noindent you want to hold her: gwéla ti delghe ichí
%\noindent he wants to know you: gwéla é gní ithí
%\noindent she wants to love him: gwéla í cára iché
%\noindent it can break me: gála í brís imí
%\noindent we can buy them: gála ni prin ichís
%\noindent you (pl.) can carry us: gála sú ber iní
%\noindent they can know you (pl.): gála sí gní isú
%\noindent you (pl.) can do it: gála sú ávó ichí

