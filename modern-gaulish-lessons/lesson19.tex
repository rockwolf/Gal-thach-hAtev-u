\section{Menghavan 19: Aus\'{e'}dhl\'{e}}
(\textit{Lesson 19: Expressions})\\

In the nineteenth lesson, you will learn about expressions that can be used in Gal\'{a}thach.

\subsection{Gwepchoprith: Conversation}
\subsubsection{Conversation}
% TODO: Add bubbly conversation

\subsubsection{Colav\'{a}ru \textendash\ Tr\'{e}lav\'{a}ru}
(Conversation \textendash\ Translation)

Here is a conversation between Tasc\'{u}an, a man, and Iurcha, a woman. Iurcha is about to walk over a trench of burning coals on her bare feet when Tasc\'{u}an arrives and starts talking to her.

Tascuan: Methin dh\'{a}i, p\'{e} gaman a hesi ti?
(Tasc\'{u}an: Good morning, how are you?)

Iurcha: Nesn\'{o}ith d\'{a}i, gw\'{e}la ti sp\'{a}. Esi mi wath, p\'{e} gaman a hesi ti ti-s\'{u}\'{e}?
(Iurcha: Good evening, you mean. I am fine, how are you yourself?)

T: N\'{e} dhruch, br\'{a}thu. A ghn\'{i}a ti och esi ti en shuling gwer h\'{a}ch in dolen ins\'{e}?
(T: Not bad, thanks. Do you know that you are dancing on the edge of the blade there?)

I: Esi mi echanal en gl\'{e} in vr\'{i} \'{e}ithra in camu'th\'{i}r.
(I: I am only pushing the mountain beyond the horizon.)

T: N\'{e} a w\'{e}na ti och esi ti en rh\'{e}thi \'{o}s 'c\'{a}th\'{e}?
(T: Don't you think that you are running after shadows?)

I: M\'{e}na mi och esi ti en dhw\'{a}ni en in m\'{e}du. P\'{e}ri n\'{e} a gonechughra ti ach r\'{o}dhi adhim tanch?
(I: I think that you are pissing in the mead. Why don't you fuck off and leave me alone?)

T: Esi ti rh\'{e} vurw\'{a}r.
(T: You are very rude.)

I: Ach esi ti en vedh\'{o}li im\'{i}.
(I: And you are annoying me.)

T: D\'{a}i, athapis\'{i} mi ti gwer sh\'{i}rach, duch.
(T: Well, I'll see you later then.)

I: N\'{e} chw\'{o}m\'{e}na mi.
(I: I hope not.)

Tasc\'{u}an: Badger-killer (< Tascouanus, attested)
Iurcha: Roe Deer (< Iurca, attested)

\subsection{Gwepchoprith: Expressions}
\subsubsection{Idiomatic Expressions (used in the conversation and other)}

suling gwer h\'{a}ch in dolen: to dance on the edge of the blade
> to take a high risk

cl\'{e} in vr\'{i} \'{e}ithra in camu'th\'{i}r: to push the mountain beyond the horizon
> to attempt the impossible

r\'{e}thi \'{o}s 'c\'{a}th\'{e}: to run after shadows
> to believe something wrong

dw\'{a}ni en in m\'{e}dhu: to piss in the mead
> to spoil the fun

conechughri: to fuck off
r\'{o}dhi adhim tanch: give me peace > leave me alone
> d\'{a}ma m\'{o} honachas: accept my aloneness: leave me alone
> l\'{a}i mi \'{o}nach: leave me alone

bedh\'{o}li don nep : to annoy someone
> bedhol: a burr (a prickly plant seedhead that attches itself to clothing, hair and fur, wraps itself into it and is very difficult to remove) esi t\'{o} dh\'{e}n\'{e} gwer m\'{o} drughn: your teeth are on my nose 
> you're very rude

esi ti en ghl\'{i} bedh\'{o}l\'{e} gwerim: you're sticking burrs on me
> you're annoying me

esi ti camath tar dhuvedh\'{o}l\'{e}: you're a walk through thistles
> you are a pain in the arse

\'{a}i a dhauni t\'{o} dhathr\'{a}ieth: go burn your feet
> go away

ponch esi in w\'{i}sa gwer desach co in s\'{u}el: when the moon is warmer than the sun
> never

aven turcha in moch: until the pig grunts
> until next time, see you later

tr\'{e}v\'{i}u br\'{u}ia in v\'{o} en \'{o} lan r\'{e}tha in \'{e}p r\'{i}u: while the cow bellows in its paddock the horse runs free
> some people making noise, i.e. complaining about something, does not prevent other people from doing what they want. > ``while wankers winge we do what we want.''

\subsubsection{Expressions of everyday courtesy and use}

d\'{i} wath: goodday
methin dh\'{a}i: good morning
\'{o}sw\'{e}dh\'{i} dh\'{a}i: good afternoon
nesnoith d\'{a}i: good evening
noith d\'{a}i: good night

br\'{a}thu: thanks
> br\'{a}thu adhith: thank you (thanks to you)
> br\'{a}thu adh\'{u}: thank you (pl.) (thanks to you pl.)

ma harw\'{e}ra \'{i} ti: please (if it pleases you)
ma harw\'{e}ra \'{i} s\'{u}: please (if it pleases you pl.)

d\'{a}i: good
r\'{e} dh\'{a}i: very good
> all right, okay

duch: so

p\'{e} gaman a hesi ti: how are you (lit. ``what way are you'')
> esi mi math: I am fine (male)
> esi mi wath: I am fine (female)
> esi mi d\'{a}i: I am good (male)
> esi mi dh\'{a}i: I am good (female)
> n\'{e} dhruch: not bad
> esi mi in dh\'{a}i: I am well (with adverbial particle \textit{in})
> mi c\'{o}\'{e}th: me too, me as well

athapis\'{i} mi ti gwer sh\'{i}rach: I will see you later
> gwer shirach: later (\textit{more late})
> a chwer sh\'{i}rach: to later
> a chwer: to more
> aven gwer sh\'{i}rach: until later
> aven gwer: until more
> \'{a}pis ith\'{i}: seeing you
> aven \'{a}pis: until seeing
> aven ath\'{a}pis: until seeing again
> ath\'{a}pis: seeing again
> aven in conesam: until the next one
> aven in \'{i}on conesam: until the next time
> \'{i}on conesam: next time
> a'n \'{i}on conesam: to next time

iachas: wellbeing/health
> iachas dh\'{a}i: good health > cheers, drinking toast

sl\'{a}nas: health
> sl\'{a}nas: health > goodbye

sw\'{a}el: good wind > welcome

nep: any, some, neither
> n\'{e} h\'{a}va \'{i} d\'{a}ias nep: it doesn't do any good
> esi duvr nep en in ban: there is some water in the cup
> n\'{e} hesi \'{i} on nep al: it is neither one or the other

r\'{e}ithu: right, law, entitlement
> mi-esi in r\'{e}ithu: I have the right

certh: right, correct
> esi mi certh: I am right (also \textit{mi-esi gw\'{i}roth})

ancherth: wrong
> esi mi ancherth: I am wrong

s\'{o}ru: fault
> mi-esi s\'{o}ru: I have fault > I apologise
> esi \'{i} m\'{o} sh\'{o}ru: it is my fault > I apologise

\subsubsection{Alternative use of \textit{how}}

To express a degree of an adjective that in English would use the word \textit{how} a different construction is used in Gal\'{a}thach:

p\'{e} dam duchw\'{i}s a hesi in dunach\'{i}ath-s\'{e}: how stupid is that politician

p\'{e} dam s\'{i}r a hesi in duru: how far is the town

The phrase \textit{p\'{e} dam} is constructed of p\'{e} (what) and tam (quality). Tam receives ICM after p\'{e} > p\'{e} dam.

\subsubsection{Use of possessive phrase [pronoun]-esi}

The possessive phrase \textit{pronoun-to-be} is the Gal\'{a}thach version of the verb \textit{to have}. It is used in a number of expressions:

    to express a physical attribute

mi-esi d\'{a}dherch b\'{u}i: I have blue eyes

ti-esi d\'{a} coch: you have two legs

\'{i}-esi gwolth s\'{i}r: she has long hair

    to express a physical state

mi-esi oghru: I am cold (I have coldness)
ti-esi tes: you are warm (you have warmth)
\'{e}-esi nan: he is hungry (he has hunger)
\'{i}-esi \'{o}nu: she is thirsty (she has thirst)
mi-esi tr\'{a}ieth bris\'{u}: I have a broken foot
ti-esi panthu'pen: you have a headache
e-\'{e}si ach\'{u}as: he is in a hurry (he has speed)

    to express inclination or intention

mi-esi swanthu a h\'{a}i a'n t\'{e}i: I feel like going home (I have fancy to go to the house)
\'{i}-esi swanthu a gan: she feels like singing (she has fancy to sing)

    to express possession

mi-esi cuchul: I have a hat/hood
ti-esi t\'{e}i: you have a house
\'{e}-esi m\'{e}nu: he has an idea
\'{i}-esi ul\'{a}nu: she is satisfied (she has satisfaction)
mi-esi in bes: I am used to (I have the habit)
ti-esi gwiroth: you are right (you have truth)

    to express possession with a specific subject

To express possession of a specific subject put this subject in front of the possession phrase, and use the right pronoun in the possession phrase:

Tasc\'{u}an \'{e}-esi bath sh\'{i}r: Tasc\'{u}an has a long stick
Iurcha \'{i}-esi gwolth gurm: Iurcha has brown hair

    to express intention of possession

To express intention of possession use the possessive phrase in a subordinate clause:

Gw\'{e}la mi o mi-esi \'{e}p: I want to have a horse (I want that I have a horse)
Gw\'{e}la \'{i} o \'{i}-esi cun: she wants to have a dog (she wants that she has a dog)

\subsubsection{Degrees of wanting}

gwel: to want
> gw\'{e}la mi \'{a}i a'n dr\'{a}ith: I want to go to the beach

gw\'{e}i: to wish
gw\'{e}ia mi \'{a}pis ith\'{i}: I wish to see you

iantha: to desire
> iantha \'{e} ben: he desires a woman

swantha: to fancy/covet
> swantha \'{i} cerdhl in tiern: she fancies/covets the job of the boss

rinchi: to need
> rincha mi depri: I need to eat

\subsubsection{Expression of wishing}

To wish something to someone else use the particle \textit{o} followed by the future form of the verb to be used:

o b\'{i} ti l\'{a}en: may you be happy (that you will be happy)
och uris\'{i} su tanch: may you (pl.) find peace (that you-pl. will find peace)
 
\subsubsection{Expression of encouragement to do something}

Use the imperative with the 1st pers. pl. pronoun \textit{ni}:
 
\'{a}i ni a 'nam: let's go swimming!

cl\'{u}i ni in gw\'{i}roth: let's hear the truth!

\subsubsection{Expression of obligation}

To say something should be done the expression \textit{would be right for someone to do something} is used:

r\'{e} v\'{i} certh riem \'{a}i a'n t\'{e}i: I should go home (would be right for-me going to the house)

n\'{e} rh\'{e} v\'{i} certh rieth cerdhi r\'{o} h\'{e}lu: you shouldn't work too much (not would be right for-you working too much)

\subsubsection{Expression of presence}

To express a presence or something that is happening the verb \textit{esi} is used by itself without a pronoun:

esi amr: it's raining (there is rain)

esi d\'{o}n\'{e} \'{e}lu insin: there are a lot of people here

esi naus\'{e} gwer in m\'{o}r: there are boats on the sea

n\'{e} hesi neveth a h\'{a}v\'{o}: there is nothing to do (not is nothing to do)
 
\subsubsection{Expression of reflexivity}

To express reflexivity the pronoun \textit{s\'{u}\'{e}} is used. The pronoun indicating the person is repeated and the pronoun \textit{s\'{u}\'{e}} is attached to it:

ap\'{i}sa mi: I see
> ap\'{i}sa mi mi-s\'{u}\'{e}: I see myself

mol\'{a}tha \'{e}: he praises
> mol\'{a}tha \'{e} ch\'{e}-s\'{u}\'{e}: he praises himself (the phonetic bridge ch- is attached to avoid two vowels following each other)

dercha in ven: the woman watches
> dercha in ven \'{i}-s\'{u}\'{e}: the woman watches herself

The reflexive pronoun can be used to indicate emphasis:

av\'{o}thu mi ch\'{i}: I did it
> av\'{o}thu mi ch\'{i} mi-s\'{u}\'{e}: I did it myself

apis\'{u} ti ch\'{i}: you have seen it
> apis\'{u} ti ch\'{i} ti-s\'{u}\'{e}: you have seen it yourself

\subsubsection{Degrees of liking}

c\'{a}ra: to love
> c\'{a}ra mi ti: I love you

\'{a}ma: to like
> \'{a}ma \'{i} mi: she likes me

n\'{a}ma: to dislike
> n\'{a}ma s\'{i} ch\'{e}: they dislike him

l\'{u}vi: to adore
> l\'{u}va in cun \'{o} diern: the dog adores his master

arw\'{e}ri: to please
> arw\'{e}ra \'{i} mi: I like it (it pleases me)

To use the verb \textit{arw\'{e}ri} with a non-pronoun subject it is used with the preposition adh- with the pronoun appropriate for the person it is directed to. The subject then follows after the preposition+pronoun:
> arw\'{e}ra adhim: I like (pleases to-me)
> arw\'{e}ra adhim depri esc: I like eating fish (pleases to-me eating fish)
> arw\'{e}ra adh\'{i} \'{i}vi curu: she likes drinking beer (pleases to-her drinking beer)

The form with the preposition adh- can also be used when the subject is a pronoun. The subject then comes before the preposition+pronoun:

arw\'{e}ra \'{i} adhim: I like it (it pleases to-me)

\subsubsection{Use of preposition+pronoun to indicate object of a sentence}

The form with the preposition adh- can also be used in other sentences when the subject is a complex or long phrase, to indicate quite clearly and without confusion exactly what the object is:

adhreth\'{u} d\'{o}n\'{e} nep adhin: some people have attacked us

In this sentence the object is the phrase \textit{some people}, \textit{d\'{o}n\'{e} nep} and the object is \textit{us}. Because the object is quite a long way from the verb and to avoid confusion it can be stated as \textit{adhin}, \textit{to-us}.

The same sentence could also be said as:

adhreth\'{u} d\'{o}n\'{e} nep ni: some people have attacked us

\subsubsection{Use of pronouns in subordinate clauses}

Subordinate clauses are made with the particle \textit{o}. This particle connects two phrases but doesn't say anything about who or what is involved in the phrases. Therefore, in subordinate clauses where the subject is a pronoun this pronoun has to be used:

Esi sin in ven: this is the woman

Gw\'{e}la \'{i} \'{i}vi curu: she wants to drink beer
> Esi sin in ven o gw\'{e}la \'{i} \'{i}vi curu: this is the woman who wants to drink beer (this is the woman that she wants to drink beer)

Esi sin in arth: this is the bear

Gw\'{e}la \'{e} depri im\'{i}: he wants to eat me
> Esi sin in arth o gw\'{e}la \'{e} depri im\'{i}: this is the bear who wants to eat me (this is the bear that he wants to eat me)

\subsubsection{Use of preposition \textit{a} to indicate intention or action}

The preposition \textit{a} (to, towards, at) can be used to indicate an intention:

esi mi sudhar\'{i}thu a dhiantha mengh\'{a}vi Gal\'{a}thach: I am excited to begin to learn Gal\'{a}thach.

Literal translation: am I excited to/at beginning learning Gal\'{a}thach.

\subsubsection{Days of the week}

Dil\'{u}iu: Monday (Day of Lugus)
Divelen: Tuesday (Day of Belenos)
Dith\'{o}thath: Wednesday (Day of Totatis)
Ditharan: Thursday (Day of Taranis)
Dimathron: Friday (Day of Matrona)
Dicharnon: Saturday (Day of Carnonos)
Diroswerth: Sunday (Day of Rosmerta)

Note: the days of the week are feminine, regardless of their respective final vowels, because they are made of the word d\'{i} (day), which is a feminine word, to which a name is added.

\subsubsection{Months of the year}

Samon: June (start of the year; start of the summer half of the year \textendash\ Month of Summer)
Duman: July (Month of Smoke)
R\'{i}ur: August (Month of Plenty)
An\'{a}ian: September (Month of Ablutions)
Oghron: October (Month of Cold)
Cuth: November (Month of Hardship)
Giamon: December (start of the winter half of the year; Month of Winter)
Simison: January (Month of Half-Sun)
\'{e}chu: February (Month of Animal Tracks)
Elem: March (Month of Deer)
\'{e}dhrin: April (Month of Fire)
Canthl: May (Month of Music)

\newpage
\subsection{Exercises}

\subsubsection{Vocabulary}

Use the vocabulary found in this lesson for the exercises, as well as the words listed below.

beautiful: tech
day: d\'{i}
certainly: in shucherth
to lie: dithonchi
cow: b\'{o}
to talk: lavar
to leave: techi
to surprise: gwergh\'{a}vi
d\'{o}n\'{e} \'{e}lu: a lot of people
emp\'{a}: to tell
s\'{e}: that
really: in chw\'{i}r
to steal: cimri
all right: suvis

\subsubsection{Translate}

Translate the following phrases:

A: Goodday, how are you?
B: I'm very well, thanks. And you?

A: I'm fine, myself. It's a beautiful day.
B: It certainly is. So, what do you want?

A: I want to lie to you.
B: Why?

A: Because I'm a politician.
B: I think you are running after shadows.

A: Don't piss in the mead.
B: Go burn your feet. You're sticking burrs on me.

A: Your teeth are on my nose.
B: And you're a walk through thistles.

A: Well, until the pig grunts then.
B: When the moon is warmer than the sun.

A: You know you're dancing on the edge of the blade.
B: While the cow bellows in its paddock the horse runs free.

A: How stupid do you think I am?
B: Very stupid.

A: I have a headache now.
B: I feel like making your headache worse.

A: I want to talk more.
B: You need to leave now.

A: I wish to say something.
B: Let's hear the truth now!

A: I covet your cow.
B: You shouldn't talk too much.

A: There's nothing else to do.
B: I have lots of work to do myself.

A: I don't like work.
B: That is a thing that does not surprise me.

A: There are a lot of people who tell me that all the time.
B: Really.

A: I am excited to steal your cow.
B: That's allright. You can come and take it on the fifth Saturday in June.

A: Really?
B: No. [not really]

\newpage
\subsubsection{Solution}

A: D\'{i} wath, p\'{e} gaman a hesi ti?
B: Esi mi in rh\'{e} dh\'{a}i, br\'{a}thu. Ach ti?

A: Esi mi math, mi-s\'{u}\'{e}. Esi \'{i} d\'{i} dech.
B: Esi \'{i} in gerth. Duch, p\'{e} a chw\'{e}la ti?

A: Gw\'{e}la mi dithonchi adhith.
B: P\'{e}ri?

A: Riveth esi mi dunach\'{i}ath.
B: M\'{e}na mi och esi ti en rh\'{e}thi \'{o}s 'c\'{a}th\'{e}.

A: N\'{e} dhw\'{a}ni en in m\'{e}dhu.
B: \'{a}i a dhauni to dhathr\'{a}ieth. Esi ti en ghl\'{i} bedh\'{o}l\'{e} gwerim.

A: Esi t\'{o} dh\'{e}n\'{e} gwer m\'{o} drughn.
B: Ach esi ti camath tar dhuvedh\'{o}l\'{e}.

A: D\'{a}i, aven turcha in moch, tun.
B: Ponch esi in w\'{i}sa gwer desach co in s\'{u}el.

A: Gn\'{i}a ti och esi ti en shuling gwer h\'{a}ch in dolen.
B: Tr\'{e}v\'{i}u br\'{u}ia in v\'{o} en \'{o} lan r\'{e}tha in \'{e}p r\'{i}u.

A: P\'{e} dam duchw\'{i}s a w\'{e}na ti och esi mi?
B: R\'{e} dhuchw\'{i}s.

A: Mi-esi panthu'pen n\'{u}.
B: Mi-esi swanthu a h\'{a}v\'{o} t\'{o} banthu'pen gwer dhruch (gwer w\'{e}s, gw\'{a}ith).

A: Gw\'{e}la mi lavar \'{e}th.
B: Rincha ti t\'{e}chi n\'{u}.

A: Gw\'{e}ia mi sp\'{a} peth nep.
B: Cl\'{u}i ni in gw\'{i}roth n\'{u}!

A: Swantha mi t\'{o} v\'{o}.
B: N\'{e} rh\'{e} v\'{i} certh rieth lavar r\'{o} h\'{e}lu.

A: N\'{e} hesi neveth al a h\'{a}v\'{o}.
B: Mi-esi cerdhl \'{e}lu a h\'{a}v\'{o} mi-s\'{u}\'{e}.

A: N\'{e} harw\'{e}ra adhim cerdhl.
B: Esi s\'{e} peth o n\'{e} chwergh\'{a}vi \'{i} mi.

A: Esi d\'{o}n\'{e} \'{e}lu och emp\'{a} s\'{i} s\'{e} adhim aman hol.
B: In chw\'{i}r.

A: Esi mi sudhar\'{i}thu a gimri t\'{o} v\'{o}.
B: Esi s\'{e} suvis. G\'{a}la ti d\'{i}\'{a}i a gh\'{a}vi ich\'{i} in Dhicharnon bimpeth en Shamon.

A: In chw\'{i}r?
B: N\'{e} in chw\'{i}r.
