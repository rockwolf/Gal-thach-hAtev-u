\section{Menghavan 6: An\'{u}\'{e} Benin Can In hAmosanal \textendash\ Gwerthan\'{a}l\'{e} Coswaus Anolsam}
(\textit{Lesson 6: Feminine Nouns With The Article \textendash\ Initial Consonant Mutations})\\
In the sixth lesson you will learn what happens to feminine nouns after the article.

\subsection{Feminine Nouns And The Article}

When feminine nouns are preceded by the article \textit{in} their initial consonant changes according to regular patterns. This change is what marks the nouns as feminine.
\begin{table}[H]
\centering
\begin{tabular}{cccc}
  \toprule
  \multicolumn{2}{c}{\textbf{Without article}} & \multicolumn{2}{c}{\textbf{With article}}\\
  \midrule
  \textbf{Gal\'{a}thach} & \textbf{English} & \textbf{Gal\'{a}thach} & \textbf{English}\\
  \cmidrule(lr){1-1}\cmidrule(lr){2-2}\cmidrule(lr){3-3}\cmidrule(lr){4-4}
  par & cauldron & in \textbf{b}ar & the cauldron\\
  taran & storm & in \textbf{d}aran & the storm\\
  carch & rock & in \textbf{g}arch & the rock\\
  ben & woman & in \textbf{v}en & the woman\\
  d\'{u}ithir & daughter & in d\textbf{h}\'{u}ithir & the daughter\\
  glan & river bank & in g\textbf{h}lan & the river bank\\
  m\'{a}thir & mother & in \textbf{w}\'{a}thir & the mother\\
  sir & star & in s\textbf{h}ir & the star\\
  sp\'{a}thl & story & in \textbf{'p}\'{a}thl & the story\\
  nath & fate & in n\textbf{h}ath & the fate\\
  rath & fern & in r\textbf{h}ath & the fern\\
  latha & swamp & in latha & the swamp\\
  fich & fig & in f\textbf{h}ich & the fig\\
  \bottomrule
\end{tabular}
\label{examples_article_feminine_nouns}
\end{table}

The semi-vowel i- (/y/ in English, $[$j$]$) is prefixed with a \textit{ch'}:
\begin{table}[H]
\centering
\begin{tabular}{cccc}
  \toprule
  \multicolumn{2}{c}{\textbf{Without article}} & \multicolumn{2}{c}{\textbf{With article}}\\
  \midrule
  \textbf{Gal\'{a}thach} & \textbf{English} & \textbf{Gal\'{a}thach} & \textbf{English}\\
  \cmidrule(lr){1-1}\cmidrule(lr){2-2}\cmidrule(lr){3-3}\cmidrule(lr){4-4}
  \textbf{i}ar & chicken & in \textbf{ch'i}ar & the chicken\\
  \bottomrule
\end{tabular}
\label{examples_article_semi_vowel_i}
\end{table}

Vowels are prefixed with a \textit{h-}:
\begin{table}[H]
\centering
\begin{tabular}{cccc}
  \toprule
  \multicolumn{2}{c}{\textbf{Without article}} & \multicolumn{2}{c}{\textbf{With article}}\\
  \midrule
  \textbf{Gal\'{a}thach} & \textbf{English} & \textbf{Gal\'{a}thach} & \textbf{English}\\
  \cmidrule(lr){1-1}\cmidrule(lr){2-2}\cmidrule(lr){3-3}\cmidrule(lr){4-4}
  \textbf{a}val & apple & in \textbf{h}aval & the apple\\
  \textbf{\'{e}}pis & mare & in \textbf{h}\'{e}pis & the mare\\
  \textbf{\'{u}}lidh & feast & in \textbf{h}\'{u}lidh & the feast\\
  \textbf{\'{\i}}dh & chain & in \textbf{h}\'{\i}dh & the chain\\
  \textbf{o}ghran & frost & in \textbf{h}oghran & the frost\\
  \bottomrule
\end{tabular}
\label{examples_article_prefix_h}
\end{table}

If the nouns are in the plural, exactly the same thing happens:
\begin{table}[H]
\centering
\begin{tabular}{cc}
  \toprule
  \multicolumn{2}{c}{\textbf{With article}}\\
  \midrule
  \textbf{Gal\'{a}thach} & \textbf{English}\\
  \cmidrule(lr){1-1}\cmidrule(lr){2-2}
  p\'{a}r\'{e} & in \textbf{b}\'{a}r\'{e}\\
  tar\'{a}n\'{e} & in \textbf{d}ar\'{a}n\'{e}\\
  carch\'{e} & in \textbf{g}arch\'{e}\\
  mn\'{a} & in \textbf{w}n\'{a}\\
  d\'{u}ith\'{\i}r\'{e} & in \textbf{dh}\'{u}ith\'{\i}r\'{e}\\
  gl\'{a}n\'{e} & in \textbf{gh}l\'{a}n\'{e}\\
  \bottomrule
\end{tabular}
\label{examples_fix_h}
\end{table}

\subsection{Initial Consonant Mutation Changes}
The changes shown above are regular, they happen like that all the time. They are summarised below:
\begin{table}[H]
\centering
\begin{tabular}{ccc}
  \toprule
  \textbf{Original char} & \textbf{Mutated char} & \textbf{Comment}\\
  \cmidrule(lr){1-1}\cmidrule(lr){2-2}\cmidrule(lr){3-3}
  p & b & \\
  t & d & \\
  c & g & \\
  b & v & \\
  d & dh & \\
  g & gh & \\
  m & w & \\
  s & sh & if followed by a wowel\\
  s & ' & $[$nothing$]$ if followed by a consonant\\
  n & nh & \\
  r & rh & \\
  l & lh & \\
  f & fh & \\
  i- & ch'i- & if followed by a vowel\\
  i- & hi- & if followed by a consonant\\
  a & ha & \\
  e & he & \\
  u & hu & \\
  o & ho & \\
  \bottomrule
\end{tabular}
\label{summary_mutated_chars}
\end{table}

Some of these sounds are unusual.\\
dh = as in English \textbf{th}e, \textbf{th}ere\\
gh = as in Greek $\acute{\varepsilon}$\textbf{$\gamma$}$\acute{\omega}$ (I, me), Breton del\textbf{c'h} (to hold). The sound is called \textit{voiced velar fricative}. Like a Scottish \textit{-ch} (as in lo\textbf{ch}) but said with a ``thick voice'' (i.e.\ voiced).\\
sh = as in English \textbf{sh}ip\\
h- = as in English \textbf{h}ouse\\
ch' = as in Scottish lo\textbf{ch}\\
nh = ch- as in Scottish lo\textbf{ch}, followed by -n $\rightarrow$ sounds like \textit{chn-}.\\
rh = ch- as in Scottish lo\textbf{ch}, followed by -r $\rightarrow$ sounds like \textit{chr-}.\\
lh = ch- as in Scottish lo\textbf{ch}, followed by -l $\rightarrow$ sounds like \textit{chl-}.\\
fh = an /f/ produced without the tongue touching the teeth. The sound is called \textit{voiceless bilabial fricative}.\\

Of these sounds, \textit{h-}, \textit{sh-}, \textit{fh-}, \textit{nh-}, \textit{rh-} and \textit{lh-} only ever occur as initial consonant mutations.

\newpage
\subsubsection{Exercises}

Put the article in front of the following words and change the consonants if necessary.
\begin{table}[H]
\centering
\resizebox{30pc}{!}{%
\begin{tabular}{|c|c|M{10.0cm}|}
  \toprule
  \textbf{Gal\'{a}thach} & \textbf{English} & \textbf{Answer (m/f)}\\
  \toprule
  pan & glas, cup & \\
  \midrule
  pen\'{a}l\'{e} & chapters & \\
  \midrule
  tal & front & \\
  \midrule
  t\'{a}m\'{e} & classes & \\
  \midrule
  c\'{a}i & hedge & \\
  \midrule
  calgh\'{e} & points & \\
  \midrule
  bach & burden & \\
  \midrule
  bal\'{a}n\'{e} & brooms & \\
  \midrule
  daghl & torch & \\
  \midrule
  d\'{a}r\'{e} & rages & \\
  \midrule
  gaval & fork & \\
  \midrule
  gn\'{a}th\'{e} & children & \\
  \midrule
  fich & fig & \\
  \midrule
  f\'{\i}ch\'{e} & figs & \\
  \midrule
  m\'{a}i & place & \\
  \midrule
  m\'{a}n\'{e} & necks & \\
  \midrule
  sachrap & evil eye & \\
  \midrule
  s\'{a}l\'{e} & sheds & \\
  \midrule
  sc\'{a}th & shadow & \\
  \midrule
  spr\'{\i}\'{e} & twigs & \\
  \midrule
  ial & clearing & \\
  \midrule
  iath\'{a}n\'{e} & loans & \\
  \midrule
  nan & hunger & \\
  \midrule
  nasc\'{a}l\'{e} & rings & \\
  \midrule
  ran & part & \\
  \midrule
  rath\'{a}n\'{e} & guarantees & \\
  \midrule
  lam & hand & \\
  \midrule
  l\'{a}n\'{e} & fields & \\
  \midrule
  \'{a}chn & milk & \\
  \midrule
  avanch\'{e} & water spirits & \\
  \midrule
  ili & ivy & \\
  \midrule
  \'{\i}thw\'{e}r\'{a}n\'{e} & distributions & \\
  \midrule
  \'{o}dhan & smell & \\
  \midrule
  olchrav\'{a}n\'{e} & descriptions & \\
  \midrule
  uchan & rise & \\
  \midrule
  ul\'{a}\'{e} & powders & \\
  \midrule
  \'{e}chal & hoof & \\
  \midrule
  echwichn\'{a}\'{e} & losses & \\
  \bottomrule
\end{tabular}
}
\label{exercise_consonant_mutation}
\caption{Exercise: consonant mutation}
\end{table}

\newpage
Solution:
\begin{table}[H]
\centering
\resizebox{18pc}{!}{%
  \rotatebox{180}{%
    \begin{tabular}{|c|c|>{\itshape}c|}
      \toprule
      \textbf{Gal\'{a}thach} & \textbf{English} & \textbf{Answer (m/f)}\\
      \toprule
      pan & glas, cup & in ban\\
      \midrule
      pen\'{a}l\'{e} & chapters & in ben\'{a}l\'{e}\\
      \midrule
      tal & front & in dal\\
      \midrule
      t\'{a}m\'{e} & classes & in d\'{a}m\'{e}\\
      \midrule
      c\'{a}i & hedge & in g\'{a}i\\
      \midrule
      calgh\'{e} & points & in galgh\'{e}\\
      \midrule
      bach & burden & in vach\\
      \midrule
      bal\'{a}n\'{e} & brooms & in val\'{a}n\'{e}\\
      \midrule
      daghl & torch & in dhaghl\\
      \midrule
      d\'{a}r\'{e} & rages & in dh\'{a}r\'{e}\\
      \midrule
      gaval & fork & in ghaval\\
      \midrule
      gn\'{a}th\'{e} & children & in ghn\'{a}th\'{e}\\
      \midrule
      fich & fig & in fhich\\
      \midrule
      f\'{\i}ch\'{e} & figs & in fh\'{\i}ch\'{e}\\
      \midrule
      m\'{a}i & place & in w\'{a}i\\
      \midrule
      m\'{a}n\'{e} & necks & in w\'{a}n\'{e}\\
      \midrule
      sachrap & evil eye & in shachrap\\
      \midrule
      s\'{a}l\'{e} & sheds & in sh\'{a}l\'{e}\\
      \midrule
      sc\'{a}th & shadow & in \'c\'{a}th\\
      \midrule
      spr\'{\i}\'{e} & twigs & in \'pr\'{\i}\'{e}\\
      \midrule
      ial & clearing & in ch'ial\\
      \midrule
      iath\'{a}n\'{e} & loans & in ch'iath\'{a}n\'{e}\\
      \midrule
      nan & hunger & in nhan\\
      \midrule
      nasc\'{a}l\'{e} & rings & in nhasc\'{a}l\'{e}\\
      \midrule
      ran & part & in rhan\\
      \midrule
      rath\'{a}n\'{e} & guarantees & in rhath\'{a}n\'{e}\\
      \midrule
      lam & hand & in lham\\
      \midrule
      l\'{a}n\'{e} & fields & in lh\'{a}n\'{e}\\
      \midrule
      \'{a}chn & milk & in h\'{a}chn\\
      \midrule
      avanch\'{e} & water spirits & in havanch\'{e}\\
      \midrule
      ili & ivy & in hili\\
      \midrule
      \'{\i}thw\'{e}r\'{a}n\'{e} & distributions & in h\'{\i}thwer\'{a}n\'{e}\\
      \midrule
      \'{o}dhan & smell & in h\'{o}dhan\\
      \midrule
      olchrav\'{a}n\'{e} & descriptions & in holchrav\'{a}n\'{e}\\
      \midrule
      uchan & rise & in huchan\\
      \midrule
      ul\'{a}\'{e} & powders & in hul\'{a}\'{e}\\
      \midrule
      \'{e}chal & hoof & in h\'{e}chal\\
      \midrule
      echwichn\'{a}\'{e} & losses & in hechwichn\'{a}\'{e}\\
      \bottomrule
    \end{tabular}
  }
}
\label{solution_consonant_mutation}
\caption{Solution: consonant mutation}
\end{table}
