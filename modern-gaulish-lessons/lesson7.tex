\section{Menghavan 7: Achath\'{e}n\'{e} \textemdash\ In Br\'{e}thr \textit{esi}}
(\textit{Lesson 7: Adjectives \textemdash\ The Verb \textit{to be}})\\

In the seventh lesson you will learn about adjectives, and how to use the verb \textit{to be}.

\subsection{Adjectives}

An adjective is a word that describes a quality or a characteristic of a noun.
\begin{table}[H]
\centering
\begin{tabu}{c|c}
  \textbf{Gal\'{a}thach} & \textbf{English}\\
  \toprule
  sen & old\\
  i\'{o}inch & young\\
  s\'{i}r & long\\
  bir & short\\
  m\'{a}r & big\\
  m\'{e}i & small\\
  ardhu & high\\
  \'{i}th & low\\
  duv & black\\
  gwin & white\\
  galv & fat\\
  d\'{a}i & good\\
  druch & bad
\end{tabu}
\label{examples_adjective}
\end{table}

There are different kinds of adjectives. The ones listed above are all natural adjectives. Adjectives can also be formed from other words by using suffixes and prefixes.

\subsubsection{The Suffix \textit{-ach}}

The suffix \textit{-ach} can be used to make an adjective from another word, most usually a noun.
\begin{table}[H]
\centering
\begin{tabu}{c|c|c|c}
  \textbf{Gal\'{a}thach} & \textbf{English} & \textbf{Gal\'{a}thach with suffix} & \textbf{English}\\
  \toprule
  caran & friend & caranach & friendly\\
  carath & love & car\'{a}thach & loveable\\
  nerth & strength & nerthach & strong\\
  caun & owl & caunach & owlish\\
  barn & judgement & barnach & judgemental\\
  achaun & stone & achaunach & stony
\end{tabu}
\label{examples_suffix_ach}
\end{table}

\subsubsection{The Suffix \textit{-ch}}
The suffix \textit{-ch} is similar to the suffix -ach. It is used for words ending on a diphthong in \textit{-i} or on \textit{-u}, where -ach would be impractical.
\begin{table}[H]
\centering
\begin{tabu}{c|c|c|c}
  \textbf{Gal\'{a}thach} & \textbf{English} & \textbf{Gal\'{a}thach with suffix} & \textbf{English}\\
  \toprule
  grau & sand & grauch & sandy\\
  t\'{e}i & house & t\'{e}ich & domestic
\end{tabu}
\label{examples_suffix_ch}
\end{table}

\subsubsection{The Suffix \textit{-\'{\i}dhu}}

The suffix \textit{-\'{\i}dhu} is used for words that end on \textit{-ch}, to avoid doubling up of the -ch- sound.
\begin{table}[H]
\centering
\begin{tabu}{c|c|c|c}
  \textbf{Gal\'{a}thach} & \textbf{English} & \textbf{Gal\'{a}thach with suffix} & \textbf{English}\\
  \toprule
  boch & mouth & boch\'{i}dhu & mouthy\\
  coch & leg & cochidhu & leggy\\
  carch & rock & carch\'{i}dhu & rocky\\
  bruch & heather & bruch\'{i}dhu & heathery
\end{tabu}
\label{examples_suffix_iidhu}
\end{table}

\subsubsection{The Suffix \textit{-in}}

The suffix \textit{-in} is only used to describe qualities of living creatures, people, animals etc.
\begin{table}[H]
\centering
\begin{tabu}{c|c|c|c}
  \textbf{Gal\'{a}thach} & \textbf{English} & \textbf{Gal\'{a}thach with suffix} & \textbf{English}\\
  \toprule
  bledh & wolf & bledhin & wolf-like (lupine)\\
  gwir & man & gwirin & masculine\\
  ben & woman & benin & feminine\\
  \'{e}p & horse & \'{e}pin & equine\\
  cun & dog & cunin & canine\\
  ernu & eagle & ern\'{u}in & aquiline
\end{tabu}
\label{examples_suffix_in}
\end{table}

\subsubsection{The Prefixes \textit{su-} And \textit{du-}}

The prefixes \textit{su-} (good) and \textit{du-} (bad) can be attached to nouns to create adjectives.
\begin{table}[H]
\centering
\begin{tabu}{c|c|c|c}
  \textbf{Gal\'{a}thach} & \textbf{English} & \textbf{Gal\'{a}thach with suffix} & \textbf{English}\\
  \toprule
  \'{a}iedh & face & \textbf{su}\'{a}iedh & good looking\\
  & & \textbf{du}\'{a}iedh & ugly
\end{tabu}
\label{examples_suffix_su_du}
\end{table}

If the emphasis in these constructions is not on the vowel /u/ of the prefixes, they become \textit{sw-} and \textit{dw-}
\begin{table}[H]
\centering
\begin{tabu}{c|c}
  \textbf{Gal\'{a}thach with suffix} & \textbf{English}\\
  \toprule
  sw\'{a}iedh & good looking\\
  dw\'{a}iedh & ugly
\end{tabu}
\label{examples_suffix_sw_dw}
\end{table}

\subsubsection{Position Of The Adjective}

An adjective always follows the nouns it says something about.
\begin{table}[H]
\centering
\begin{tabu}{c|c|c|c}
  \textbf{Gal\'{a}thach} & \textbf{English} & \textbf{Gal\'{a}thach with suffix} & \textbf{English}\\
  \toprule
  gwir & man & gwir caranach & a friendly man\\
  mapath & boy & mapath m\'{e}i & a little boy\\
  cun & dog & cun duv & a black dog\\
  \'{e}p & horse & \'{e}p gwin & a white horse
\end{tabu}
\label{examples_adjective_order}
\end{table}

\subsubsection{Mutation Of The Adjective}

If an adjective says something about a feminine noun it undergoes mutation of its initial consonant. This is how gender is indicated.
\begin{table}[H]
\centering
\begin{tabu}{c|c|c|c|c|c}
  \multicolumn{2}{c}{\textbf{Noun}} & \multicolumn{2}{c}{\textbf{Adjective}} & \multicolumn{2}{c}{Mutation with feminine noun}\\
  \toprule
  \textbf{Gal\'{a}thach} & \textbf{English} & \textbf{Gal\'{a}thach} & \textbf{English} & \textbf{Gal\'{a}thach mutation} & \textbf{English}\\
  \toprule
  ben & woman & tech & beautiful & ben dech & a beautiful woman\\
  \'{e}pis & a mare & \'{a}chu & fast & \'{e}pis h\'{a}chu & a fast mare\\
  geneth & girl & gwimp & pretty & geneth chwimp
\end{tabu}
\label{examples_adjective_order}
\end{table}

The adjective does not change for the plural.
\begin{table}[H]
\centering
\begin{tabu}{c|c|c|c|c|c}
  \textbf{Gal\'{a}thach} & \textbf{English}\\
  \toprule
  in ghen\'{e}th\'{e} chwimp & the pretty girls\\
  in wn\'{a} dech & the beautiful women\\
  in gw\'{i}r\'{e} galv & the fat men
\end{tabu}
\label{examples_no_mutation_for_plural}
\end{table}

\subsection{The Verb \textit{to be}}
The verb \textit{to be} is the only irregular verb in modern Gaulish.\\
The verbal root (or infinitive) is \textit{bis}.\\
gw\'{e}la mi bis l\'{a}en = I want to be happy
The present tense form is \textit{esi}. It does not take an \textit{-a}.\\
Esi mi l\'{a}en = I am happy

\subsection{Exercises}
Construct the following sentences. You can check your answers at the end of the lesson.
%The horse is big &gt;
%The dog is fat &gt;
%the woman is young &gt;
%the man is old &gt;
%the girl is small &gt;
%the stone is short &gt;
%the house is long &gt;
%the judgement is bad &gt;
%the owl is low &gt;
%the sand is white &gt;
%the mare is friendly &gt;
%the mouth is big &gt;
%the wolfish man &gt;
%the equine woman &gt;
%the young girl &gt;
%the fat boy &gt;
%the high eagle &gt;
%the low stone &gt;
%the black rock &gt;
%the white heather &gt;
%the ugly mouth &gt;
%the long leg &gt;
%the strong horse &gt;
%the bad love &gt;
%the young girls see the white horses &gt;
%the old men call the fat dogs &gt;
%the friendly women want the little boys &gt;
%the big dogs love the black stones &gt;
%the little girls hold the owlish dogs &gt;
%the bad boys break the long stones &gt;

%<strong>Answers</strong>
%The horse is big &gt; esi in �p m�r
%The dog is fat &gt; esi in cun galv
%the woman is young &gt; esi in ven ch�i�inch
%the man is old &gt; esi in gwir sen
%the girl is small &gt; esi in gheneth w�i
%the stone is short &gt; esi in achaun bir
%the house is long &gt; esi in t�i s�r
%the judgement is bad &gt; esi in varn dhruch
%the owl is low &gt; esi in caun �th
%the sand is white &gt; esi in grau gwin
%the mare is friendly &gt; esi in h�pis garanach
%the mouth is big &gt; esi in boch m�r
%the wolfish man &gt; in gwir bledhin
%the equine woman &gt; in ven h�pin
%the young girl &gt; in gheneth ch�i�inch
%the fat boy &gt; in mapath galv
%the high eagle &gt; in ernu ardhu
%the low stone &gt; in achaun �th
%the black rock &gt; in garch dhuv
%the white heather &gt; in bruch gwin
%the ugly mouth &gt; in boch dw�iedh
%the long leg &gt; in coch s�r
%the strong horse &gt; in �p nerthach
%the bad love &gt; in garath druch
%the young girls see the white horses &gt; ap�sa in ghen�th� ch�i�inch in �p� gwin
%the old men call the fat dogs &gt; g�ra in gw�r� sen in c�n� galv
%the friendly women want the little boys &gt; gw�la in wn� garanach in map�th� m�i
%the big dogs love the black rocks &gt; c�ra in c�n� m�r in garch� dhuv
%the little girls hold the owlish dogs &gt; delgha in ghen�th� w�i in c�n� caunin
%the bad boys break the long stones &gt; br�sa in map�the druch in achaun� s�r

