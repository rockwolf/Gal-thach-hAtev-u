\section{Menghavan 9: Gwepr\'{a}i\'{e}}
(\textit{Lesson 9: Prepositions})\\

In the nineth lesson you will learn about prepositions. A preposition is a word that provides information about the location, situation or position of something.

\subsection{Prepositions}

All prepositions in modern Gaulish cause initial consonant mutation on the word that follows them.
\begin{table}[H]
\centering
\begin{tabu}{c|c}
  \textbf{Gal\'{a}thach} & \textbf{English}\\
  \toprule
  esi & is\\
  \'{e}p & horse\\
  anel & underneath\\
  pren & tree\\
  esi \'{e}p anel \textbf{b}ren & there is a horse underneath a tree (litt. ``a horse is underneath a tree'')\\
  \midrule
  s\'{e}dhi & sit\\
  gwir & man\\
  ur & against\\
  carch & rock\\
  s\'{e}dha gwir ur \textbf{g}arch & a man sits against a rock\\
  \midrule
  esi & is\\
  gwolth & hair\\
  en & in\\
  iuth & soup\\
  esi gwolth en \textbf{ch}'iuth & there is a hair in a soup\\
  \midrule
  esi & am\\
  mi & I\\
  e & from\\
  t\'{o}th & people\\
  r\'{i}u & free\\
  esi mi e \textbf{d}\'{o}th r\'{i}u & I am from a free people\\
  \midrule
  \'{a}ia & go\\
  \'{i} & she\\
  a & to\\
  tr\'{a}ith & beach\\
  \'{a}ia \'{i} a \textbf{d}r\'{a}ith & she goes to a beach
\end{tabu}
\label{examples_prepositions}
\end{table}

Prepositions do not cause mutations on the article \textit{in} or on possessive pronouns.
\begin{table}[H]
\centering
\begin{tabu}{c|c}
  \textbf{Gal\'{a}thach} & \textbf{English}\\
  \toprule
  esi \'{e}p gw\'{o} \textbf{i}n pren & there is a horse under the tree\\
  esi gwolth en \textbf{m}\'{o} ch'iuth & there is a hair in my soup
\end{tabu}
\label{examples_prepositions_no_mutation_on_in}
\end{table}

\subsubsection{With Personal Pronouns}
Prepositions fuse with personal pronouns:
%can: with
%mi: I
%&gt; canim: with me
%gwer: on
%ti: you
%&gt; gwerith: on you
%nes: near
%ni: we
%&gt; nesin: near us

\subsection{Patterns}

Prepositions fuse with personal pronouns according to regular patterns. There are four different categories for these patterns.

\subsubsection{Prepositions Ending In Consonants}
These are:
% TODO: make a table: gal\'{a}thach english  for all of them
%can, ar, ern, ur, cin, \'{o}s, gwer, en, tar, am, \'{e}r, \'{e}chan, enther, uchel, anel
% (with, before, behind, against, after, on, in, through, about, around, without, between, above, below)

These take the following endings:
\begin{quote}
\textit{-im, -ith, -\'{e}, -\'{i}, -in, -s\'{u}, -\'{i}s}
\end{quote}

%canim: with me
%canith: with you
%can\'{e}: with him
%can\'{i}: with her
%canin: with us
%cans\'{u}: with you (pl.)
%can\'{i}s: with them

The emphasis in these forms falls on the last syllable:
\begin{quote}
can\textbf{im}: with \textbf{me}
\end{quote}

\subsubsection{Prepositions Ending In Vowels}
These are:
% TODO: make table with gal\'{a}thach english for all of them
%ri, di, tr\'{e}, co, \'{e}ithra, an\'{o}, ech\'{o} (for, off, across, than/as, beyond, inside, outside)

These take the following endings:
\begin{quote}
\textit{-em, -eth, -ch\'{e}, -ch\'{i}, -en, -s\'{u}, -ch\'{i}s}
\end{quote}

%riem: for me
%rieth: for you
%rich\'{e}: for him
%rich\'{i}: for her
%rien: for us
%ris\'{u}: for you (pl.)
%rich\'{i}s: for them

The preposition \textit{tr\'{e}} takes a special form of this. It adds an \textit{-i-} to the front of the 1\textsuperscript{st}, 2\textsuperscript{nd} and 5\textsuperscript{th} form:
\begin{quote}
\textit{-\textbf{i}em, -\textbf{i}eth, -ch\'{e}, -ch\'{\i}, -\textbf{i}en, -s\'{u}, -ch\'{\i}s}
\end{quote}

%tr\'{e}iem: across me
%tr\'{e}ieth: across you
%tr\'{e}ch\'{e}: across him
%tr\'{e}ch\'{i}: across her
%tr\'{e}ien: across us
%tr\'{e}s\'{u}: across you (pl.)
%tr\'{e}ch\'{i}s: across them
%<ol start="3">

\subsubsection{Prepositions Consisting Of Only One Vowel}
% TODO: make a table with gal\'{a}thach, english
These are: a, e, u, i (to, from, of [with quantity], of [property])

These take the following endings:
\begin{quote}
\textit{-im, -ith, -\'{e}, -\'{i}, -in, -\'{u}, -\'{i}s}
\end{quote}

Also, the root form of the preposition changes:
%a &gt; adh-
%e &gt; ech-
%u &gt; uch-
%i &gt; ich-
%adhim: to me
%adhith: to you
%adh\'{e}: to him
%adh\'{i}: to her
%adhin: to us
%adh\'{u}: to you (pl.)
%adh\'{i}s: to them

\subsubsection{Prepositions Ending On \textit{-u}}
% TODO: make table with gal\'{a}thach, english
These are:
%au, didh\'{\i}u (away from, outside)}

These prepositions don't fuse with personal pronouns. The personal pronouns follow after them without changing. The object pronouns are used.
%au mi: away from me
%au ti: away from you
%au ch\'{e}: away from him
%au ch\'{i}: away from her
%au ni: away from us
%au s\'{u}: away from you (pl.)
%au ch\'{i}s: away from them

\subsubsection{Exercises}
%can, ar, ern, ur, cin, \'{o}s, gwer, en, tar, am, \'{e}r, \'{e}chan, uchel, anel, enther; ri, di, tr\'{e}, co, \'{e}ithra, an\'{o}, ech\'{o}; a, e, u, i; au, didh\'{i}u, nes
%you go with me:
%I stand before you:
%the sun is behind him:
%the wind is against her:
%this happens before it:
%this happens after it:
%the rain falls on us:
%the power is in you (pl.):
%the music goes through them:
%the people talk about me:
%the mountains are around you:
%the girl goes without him:
%the sky is above us:
%the earth is below you (pl.):
%the desert is between them:
%the work is for me:
%the sweat falls off you:
%it passes across him:
%she is as big as her:
%it is beyond us:
%the problem is inside of you (pl.):
%the solution is outside of them:
%she comes to me:
%it comes from you:
%five litres of it:
%the beer is of-him [his]:
%they run away from her
%the demon is outside of us:
%the monster is near them:

%<strong>Answers</strong>
%you go with me: \'{a}ia ti canim
%I stand before you: s\'{a}ia mi arith
%the sun is behind him: esi in s\'{u}el ern\'{e}
%the wind is against her: esi in \'{a}el ur\'{i}
%this happens before it: gw\'{e}ra sin cin\'{i}
%this happens after it: gw\'{e}ra sin \'{o}s\'{i}
%the rain falls on us: c\'{o}ima in hamr gwerin
%the power is in you (pl.): esi in gus ens\'{u}
%the music goes through them: \'{a}ia in ganthl tar\'{i}s
%the people talk about me: lav\'{a}ra in d\'{o}n\'{e} amim
%the mountains are around you: esi in vr\'{i}\'{e} \'{e}rith
%the girl goes without him: \'{a}ia in gheneth echan\'{e}
%the sky is above us: esi in nem uchelin
%the earth is below you (pl.): esi in lithau anels\'{u}
%the desert is between them: esi in dithrev enther\'{i}s
%the work is for me: esi in cerdhl riem
%the sweat falls off you: c\'{o}ima in shw\'{i}s dieth
%it passes across him: gw\'{e}ra \'{i} tr\'{e}ch\'{e}
%she is as big as her: esi \'{i} co w\'{a}r coch\'{i}
%it is beyond us: esi \'{i} \'{e}ithraen
%the problem is inside of you (pl.): esi in dhuchuthas an\'{o}s\'{u}
%the answer is outside of them: esi in hathespath ech\'{o}ch\'{i}s
%she walks to me: c\'{a}ma \'{i} adhim
%it comes from you: di\'{a}ia \'{i} echith
%five litres of it: pimp lithr uch\'{i}
%the beer is of-him [his]: esi in curu ich\'{e}
%they run away from her: r\'{e}tha s\'{i} au ch\'{i}
%the demon is outside of us: esi in dus dhidh\'{i}u ni
%the monster is near you (pl.): esi in havanch nes\'{u}

