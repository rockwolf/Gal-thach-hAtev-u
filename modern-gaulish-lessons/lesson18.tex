\section{Menghavan 18: ?}
(\textit{Lesson 18: Word Formation})\\

In the eighteenth lesson, you will learn how the word formation system works in Gal'{a}thach.

\subsection{Colav\'{a}ru \textendash\ Tr\'{e}lav\'{a}ru}
(Conversation \textendash\ Translation)

This is a conversation between a queen (r\'{\i}an) and a warrior (cingeth).

R\'{\i}an: Duch...  r\'{e} dhith\'{e}chi ti s\'{\i}rach ri'n samu...
(Queen: So... you turned up late for the meeting...)

Cingeth: R\'{e} dhith\'{e}chi mi.
(Warrior: I did.)

R: ... ach a h\'{a}v\'{o} tanch r\'{e} br\'{\i}tha ti n\'{a}thu'm\'{o}lath ri m\'{o} dhiruch en in w\'{a}i-sin, ach r\'{e} gan ti ch\'{\i} adhim en rh\'{e}thi achantha m\'{o} hesedh ...
(Q: ... and to make up for it you composed a praise poem in my honour on the spot and sang it to me while you ran next to my chariot ...)

C: R\'{e} gan mi en rh\'{e}thi.
(W: I did)

R: ... ach ton a dhich\'{e}ni m\'{o} chwesunan r\'{e} rh\'{o}dhi mi ti bulgh u ganech ... 
(Q: ... and then to show my appreciation I gave you a bag of gold ...)

C: R\'{e} rh\'{o}dhi ti ch\'{\i}.
(W: You did.)

R: ... ach esi ti l\'{a}en can in bulgh u ganech s\'{e}?
(Q: ... and are you happy with that bag of gold?)

C: Esi mi, br\'{a}thu r\'{e} h\'{e}lu.
(W: I am, thanks very much.)

R: ... duch, p\'{e}ri a hesi ti en rh\'{e}thi achantha m\'{o} hesedh tr\'{a}iu?
(Q: ... so why are you still running along next to my chariot?)

C: Esi m\'{o} ghran ghl\'{\i}thu en in roth.
(W: My beard is stuck in the wheel.)

\subsection{Combining two words (compound words)}

Two separate words can be combined to make a new word. They are connected with an apostrophe, and the first consonant of the second word undergoes a word-internal mutation. These mutations are different from the initial consonant mutations. They only affect t, c, b, d, g, m and gw. P, n, r, l, s and vowels are unaffected. These consonants change as follows:

t > th
c > ch
b > v
d > dh
g > gh after consonant, i after vowel
m > w after consonant, m after vowel (no change)
gw > chw

The first word is the main word (head word) and the second word is the qualifier word, saying something about the character, nature or quality of the first word.

The conversation above uses the compound word \textit{n\'{a}thu'm\'{o}lath}.

n\'{a}thu: poem
m\'{o}lath: praise
> n\'{a}thu'm\'{o}lath: praise-poem

There is no change to the initial m of m\'{o}lath because n\'{a}thu ends in a vowel.

Other examples are:

t\'{e}i: house
curu: beer
> t\'{e}i'churu: pub (\textit{beer-house})

march: horse
c\'{a}thu: battle
> march'ch\'{a}thu: battle-horse

carnu: horn
t\'{a}ru: bull
carnu'th\'{a}ru: bull-horn

\subsection{Use of prefixes (a word or part of a word attached to the front of another word)}

Prefixes can be used in combination with nouns or verbs to make new words. They are listed with their meanings below.

su-: good
du-: bad
di-: un-, the opposite or negation of something, used for verbs and nouns
ath\'{e}-: again, repetition of something
an-: un-, the opposite or negation of something, used for adjectives
an\'{e}-: very, the intensification of something

The conversation above uses the following:

dithechi: to arrive
> di- \textit{the opposite of} + techi \textit{to depart} > opposite-of-departing = arriving

diruch: honour
> di- \textit{the opposite of} + ruch \textit{shame} > opposite-of-shame = honour

Other examples are:

sw\'{a}el
> su- + \'{a}el > good-wind = welcome (the -u- of su- becomes -w- before a vowel)

suchw\'{\i}s
> su- + gw\'{\i}s > good-wise = clever

dw\'{a}iedh
> du- + \'{a}iedh > bad-face = ugly (the -u- of du- becomes -w- before a vowel)

duchw\'{\i}s
> du- + gw\'{\i}s > bad-wise = stupid

ath\'{e}men
> ath\'{e}- + men > again-think = to rethink

ath\'{a}pis
> ath\'{e}- + \'{a}pis > again-see = to see again (the \'{e} of ath\'{e} is dropped before a vowel)

anchuth
> an- + cuth > un-difficult = easy

ancherth
> an- + certh > un-right = wrong

an\'{e}lonchi
> an\'{e}- + lonchi > very-swallow = to consume

an\'{e}m\'{a}r
> an\'{e} + m\'{a}r > very-big = huge

\subsection{Use of suffixes(words or part of words attached to the end of words)}

There are several suffixes that can be used to create new words out of other words. The conversation above shows the word \textit{s\'{\i}rach}.

\subsubsection{The suffix \textit{-ach}}

The suffix \textit{-ach} makes a word into an adjective or derives another adjective from an existing one. It translates as \textit{-like}.

caran: friend
> caranach: friendly

s\'{\i}r: long
> s\'{\i}r + -ach > long-like > late

\subsubsection{The suffix \textit{-as}}

The suffix \textit{-as} makes a noun out of an adjective.

s\'{\i}r: long
> s\'{\i}ras: length

s\'{\i}rach: late
> s\'{\i}rachas: lateness

\subsubsection{The suffix \textit{-l\'{o}i}}

The suffix \textit{-l\'{o}i} makes a collective out of a noun.

don: person, human being
> donl\'{o}i: humanity (all people)

gwep: word
> gwepl\'{o}i: vocabulary

\subsubsection{The suffix \textit{-\'{\i}u}}

The suffix \textit{-\'{\i}u} creates a general abstract concept from a concrete specific noun.

caran: friend
> caran\'{\i}u: friendship

gwas: servant
> gwas\'{\i}u: servitude

\subsubsection{The suffix \textit{-widh}}

The suffix \textit{-widh} creates a noun which indicates a person who studies and knows about a discipline or science.

d[\'{e}]ru: oak
> *der\'{u}widh > dr\'{u}idh: person who studies and knows about oaks

bith: life
> bithwidh: biologist = person who studies and knows about life
 
anath: soul, psyche
> anathwidh > psychologist = person who studies and knows about the soul or psyche

This can be combined with the suffix -\'{\i}u given above to construct an abstract noun describing a discipline or science.

anath + widh + \'{\i}u > anathwidh\'{\i}u: psychology
bith + widh + \'{\i}u > bithwidh\'{\i}u: biology

\subsubsection{The suffix \textit{-th\'{o}i}}

The suffix \textit{-th\'{o}i} creates an adjective with a quality of ability.

\'{\i}vi: to drink
> iv\'{\i}th\'{o}i: drinkable

cnughni: to read
> cnughn\'{\i}th\'{o}i: legible (``readable'')

This can be combined with the suffix -as given above to construct an abstract noun describing an ability.

\'{\i}vi + th\'{o}i + as > ivith\'{o}ias: drinkability
cnughni + th\'{o}i + as > cnughnith\'{o}ias: legibility (``readability'')

\subsubsection{The suffix \textit{-\'{a}ith}}

The suffix \textit{-\'{a}ith} creates a noun which describes the skill, art, practice, and application of knowledge in a given field, as well as its abstract concept. It is the equivalent of the English suffixes -ry as well as -ism.

c\'{e}th: forest
> c\'{e}t\'{a}ith: forestry (the art and practice of looking after forests)

gwirth\'{o}i: archer
> gwirth\'{o}i\'{a}ith: archery (the art and practice of shooting bows and arrows)

tr\'{u}inan: delivery of babies
> tr\'{u}inan\'{a}ith: midwifery (the art and practice of delivering babies)

cerdhach: professional
> cerdhach\'{a}ith: professionalism (the art and practice of being professional)

lanach: pagan
> lanach\'{a}ith: paganism (the art and practice of being pagan)

penw\'{e}nu: ideal
penwen\'{u}\'{a}ith: idealism (the art and practice of being idealistic)

\subsection{Use of prepositions as prefixes}

Prepositions (words that indicate a position or a movement) can be added to the front of other words.

The conversation above shows the word \textit{achantha}.

cantha: side
a: to, towards, at, by
> a + cantha > achantha: besides, by the side

gar: to call
ar: in front of
> ar + gar > arghar: to present

t\'{a}i: to touch
am: around
> am + t\'{a}i > amth\'{a}i: to wrap

\subsection{Words derived from verbs}

There is a specific and systematical way of constructing words based on verbs. Each class of verbs constructs these words in their own way.

Each verb gives at least:
\begin{enumerate}
\item a verbal noun or infinitive: the root form of the verb
\item a word that describes someone who does the action of the verb, using the suffixes -\'{\i}ath, -eth and -il
\item a word that describes the abstract concept of the verb, using the suffixes -u, -an, -en, -on, -na, -l and -thl
\end{enumerate}

The conversation above shows the word gwesunan, which means \textit{appreciation}. This is derived from the verb gwesuni (to appreciate).

The verb gwesuni is itself derived from the word gw\'{e}su (valuable) + the suffix \textit{-ni}, which can be used to construct a verb from a word that ends in a vowel.

gw\'{e}su: valuable
> gwes\'{u}ni: to appreciate
> gwesun\'{\i}ath: -appreciator-, someone who appreciates
> gwesunan: appreciation

The conversation above also gives other verbs. They are listed here with their derivatives.

dith\'{e}chi: to arrive
dithech\'{\i}ath: arriver
dithechna: arrival
\'{a}v\'{o}: to do, to make
av\'{\i}ath: do-er, maker
avan: deed
prithi: to compose
prith\'{\i}ath: composer
prithan: composition
can: to sing
can\'{\i}ath: singer
c\'{a}nu: song
r\'{e}thi: to run
reth\'{\i}ath: runner
rethan: run
dich\'{e}ni: to show
dicheniath: show-er, person who shows
dichenan: show
r\'{o}dhi: to give
rodh\'{\i}ath: giver
r\'{o}dhl: gift
gl\'{\i}: to stick
gl\'{\i}ath: sticker
gl\'{\i}on: obstruction

The system for derivation of words from the different verb classes is given below.
 
verbal noun / infinitive
 
	agentive form / person who does it 	abstract noun
verbs on -n, -r, -l, -m

%TODO: check website, this is a big table
men: to think
gar: to call
gwel: to want
dam: to endure

	-\'{\i}ath
men\'{\i}ath: thinker
gar\'{\i}ath: caller
gwel\'{\i}ath: wanter
dam\'{\i}ath: endurer
	-u
m\'{e}nu: thought
g\'{a}ru: call
gw\'{e}lu: will(power)
d\'{a}mu: endurance

verbs on -s, -thi, -vi, -\'{o}

\'{a}pis: to see
rethi: to run
gavi: to take
\'{a}v\'{o}: to do
	-\'{\i}ath
apis\'{\i}ath: see-er
reth\'{\i}ath: runner
gav\'{\i}ath: taker
\'{a}v\'{\i}ath: doer
	-an
ap\'{\i}san: sight
rethan: run
gavan: take, taking
\'{a}van: deed

verbs on -Vowel+i

an\'{e}i: to protect
	-\'{\i}ath

an\'{e}iath: protector
	-thl

an\'{e}ithl: protection
verbs on -dhi

s\'{e}dhi: to sit
	-\'{\i}ath

sedh\'{\i}ath: sitter
	-l

sedhl: seat
verbs on -chi

tonchi: to swear
	-\'{\i}ath

tonch\'{\i}ath: swearer
	-na

tonchna: oath, pledge
verbs on -ghe

orghe: to murder
	-eth

orgheth: murderer
	-en

orghen: murder
verbs on -pi

popi: to cook
	-il

popil: cook
	-an

popan: cooking
verbs on -a, noun attested

c\'{a}ra: to love
	-\'{a}iath

car\'{a}iath: lover
	-ath

c\'{a}rath: love
verbs on -a, noun not att.

p\'{e}tha: to ask
	-\'{a}iath

peth\'{a}iath: asker
	-an

pethan: question
verbs on mono syllabic -i

gn\'{\i}: to know
	-\'{\i}ath

gn\'{\i}ath: knower
	-on

gn\'{\i}on: knowledge
verbs on -wi

samwi: to meet, convene
	-w\'{\i}ath

samw\'{\i}ath: convener
	-w\'{\i}an

samw\'{\i}an: convention

\subsection{Verbs derived from other words}

Verbs can be derived from nouns and from adjectives. These can be turned into verbs by the adding of the suffixes -i, -e, ni and -a. -a is only used very rarely.

a) The conversation above has the word s\'{a}mu (meeting).
This noun has a verb derived from it as follows:

s\'{a}mu + -i > samui > samwi: to meet, to convene

b) The conversation above has the word bulgh (bag).

If a word ends on -gh to make a verb out of it the ending -e is used.

bulgh + e > bulghe: to bag

c) The conversation above has the verb gwes\'{u}ni (to appreciate).

This is derived from the adjective gwesu (valuable)
> gw\'{e}su + -ni > gwes\'{u}ni: to appreciate

d) Use of the suffix -a to make a verb:

> cath\'{e}i: projectile
> c\'{a}tha: to throw

\subsection{Excercises}

Translate the following sentences. It's a conversation between a boy and a girl. It illustrates the ritual of courtship in traditional Gaulish society. Use the words given throughout the lesson and the ones listed below. The answers can be found at the end of the lesson.

description: olchravan
picture: suvrich
zero: nev
to study: gnis\'{a}i
au chw\'{a}itham: in spite of
to suffer from: pantha e
to sit down: s\'{e}dhi
chair: sesa
proposal: ar\'{a}dhan
to put: \'{a}dha
throat: r\'{a}iman
to hope: gw\'{o}men
certainly: in shucherth
three: tr\'{\i}
times: aun
thirty: gwochon-dech
year: bl\'{e}dhn

B: I want to go to the pub.
G: Do you want to drink beer from a bull-horn?

B: I will do that when we arrive there.
G: Do you think that will be a clever thing to do?

B: I think not doing it would be stupid.
G: I want to consume lots of beer.

B: Do you think you would like to re-think that?
G: I don't. It will be easy.

B: Drinking beer will be a welcome thing to do.
G: It will be. I need to drink a lot more beer, you're still ugly.

B: Do you think the beer will be drinkable?
G: I'd like to read the description, but the picture has got zero legibility.

B: You have got a great vocabulary.
G: I do. It's because I'm a psychologist.

B: I study biology myself.
G: Is that why [is it for that] you are so friendly?

B: It is, in spite of your lateness.
G: I can see you don't suffer from servitude.

B: That's right, I don't. That's because I'm a beer drinker with professionalism.
G: Is that why you just sat down next to your chair?

B: I want to present you with [present to you] a proposal.
G: What is it?

B: First I want to wrap my arms around you.
G: Really.

B: Then I will sing for you.
G: Then I will run from you.

B: But I'm a very good singer!
G: And I'm a better runner.

B: But I want to give you the gift of my song.
G: And I will give you the obstruction of my run.

B: What if [what about] I put a hold on you?
G: Then I will be the holder of your throat.

B: I see.
G: I certainly hope you do.

B: Would you like to meet again?
G: Not in three times thirty years.

\newpage
Solution:

B: I want to go to the pub. > Gw\'{e}la mi \'{a}i a'n t\'{e}i'churu.
G: Do you want to drink beer from a bull-horn? > A chw\'{e}la ti \'{\i}vi curu e garnu'th\'{a}ru?

B: I will do that when we arrive there. > Avos\'{\i} mi s\'{e} ponch dith\'{e}cha ni ins\'{e}.
G: Do you think that will be a clever thing to do? > A w\'{e}na ti o b\'{\i} s\'{e} peth suchw\'{\i}s a h\'{a}v\'{o}?

B: I think not doing it would be stupid. > M\'{e}na mi o r\'{e} v\'{\i} \'{\i} duchw\'{\i}s n\'{e} h\'{a}v\'{o} ich\'{\i}.
G: I want to consume lots of beer. > Gw\'{e}la mi an\'{e}lonchi curu \'{e}lu.

B: Do you think you would like to re-think that? > A w\'{e}na ti o r\'{e} chwels\'{\i} ti ath\'{e}men s\'{e}?
G: I don't. It will be easy. > N\'{e} w\'{e}na mi. B\'{\i} \'{\i} anchuth.

B: Drinking beer will be a welcome thing to do. > B\'{\i} \'{\i}vi curu peth sw\'{a}el a h\'{a}v\'{o}.
G: It will be. I need to drink a lot more beer, you'e still ugly. > B\'{\i} \'{\i}. Rincha mi \'{\i}vi curu \'{e}th \'{e}lu, esi ti dw\'{a}iedh tr\'{a}iu.

B: Do you think the beer will be drinkable? > A w\'{e}na ti o b\'{\i} in curu iv\'{\i}th\'{o}i?
G: I'd like to read the description, but the picture has got zero legibility. > R\'{e} chwels\'{\i} mi cnughn\'{\i} in holchravan, \'{e}ithr in shuvrich \'{\i}-esi nev cnughith\'{o}ias.

B: You have got a great vocabulary. > Ti-esi gwepl\'{o}i m\'{o}i.
G: I do. It's because I'm a psychologist. > Mi-esi. Esi \'{\i} riveth esi mi anathwidh.

B: I study biology myself. > Gnis\'{a}ia mi bithwidh\'{\i}u mi-s\'{u}\'{e}.
G: Is that why [is it for that] you are so friendly? > A hesi \'{\i} ri sh\'{e} och esi ti co garanach?

B: It is, in spite of your lateness. > Esi \'{\i}, au chwaitham t\'{o} shirachas.
G: I can see you don't suffer from servitude. > G\'{a}la mi \'{a}pis o n\'{e} bantha ti e chwas\'{\i}u.

B: That's right, I don't. That's because I'm a beer drinker with professionalism. > Esi \'{\i} certh, n\'{e} bantha mi. Es\'{\i} s\'{e} riveth esi mi iv\'{\i}ath curu can gerdhach\'{a}ith.
G: Is that why [is it for that] you just sat down next to your chair? > A hesi \'{\i} ri sh\'{e} o r\'{e} sh\'{e}dhi ti achantha t\'{o} shesa r\'{e} nh\'{u}?

B: Hmf. I want to present you with [present to you] a proposal. > Hmf. Gw\'{e}la mi arghar adhith ar\'{a}dhan.
G: What is it? > P\'{e} a hesi \'{\i}?

B: First I want to wrap my arms around you. > In gin gw\'{e}la mi amth\'{a}i m\'{o} dhadhos amith.
G: Really. > In chw\'{\i}r.

B: Then I will sing for you. > Ton cans\'{\i} mi rieth.
G: Then I will run from you. > Ton rethis\'{\i} mi au ti.

B: But I'm a very good singer! > \'{e}ithr esi mi can\'{\i}ath r\'{e} dh\'{a}i!
G: And I'm a better runner. > Ach esi mi reth\'{\i}ath gwer dh\'{a}i.

B: But I want to give you the gift of my song. > \'{e}ithr gw\'{e}la mi r\'{o}dhi adhith r\'{o}dhl m\'{o} g\'{a}nu.
G: And I will give you the obstruction of my run. > Ach rodhis\'{\i} mi adhith gl\'{\i}on m\'{o} rhethan.

B: What if [what about] I put a hold on you? > P\'{e} am a h\'{a}dha mi delghen gwerith?
G: Then I will be the holder of your throat. > Ton b\'{\i} mi delgheth to rh\'{a}iman.

B: I see. > Ap\'{\i}sa mi.
G: I certainly hope you do [you see]. > Gw\'{o}m\'{e}na mi in shucherth och ap\'{\i}sa ti.

B: Would you like to meet again? > A rh\'{e} chwels\'{\i} ti samwi ath\'{e}?
G: Not in three times thirty years. > N\'{e} en dr\'{\i} aun gwochon-dech bl\'{e}dhn.
