\section{Menghavan 4: Alghnas An\'{u}\'{e}}}
(\textit{Lesson 4: Gender Of Nouns})\\

In the fourth lesson you will learn how to determine the gender of nouns.

\subsection{Gender Of Nouns}

In modern Gaulish all nouns have a gender, which is either masculine or feminine. If the meaning of the noun indicates a gender, then that noun is that gender:
%gwir: man
%ben: woman
%mapath: boy
%geneth: girl
%map: son
%dúithir: daughter
%áther: father
%máthir: mother
%moth: penis
%tuthu: vagina

If the meaning of a noun does not indicate a gender, its gender is determined by the last vowel. If the last vowel is /a/ or /i/ the noun is feminine.
%lam: hand &gt; fem.
%bis: finger &gt; fem
%tír: land &gt; fem
%cnam: bone &gt; fem
%If the last vowel is /e/, /o/ or /u/ the noun is masculine.
%pen: head &gt; masc.
%mór: sea &gt; masc.
%cánu: song &gt; masc.
%tráieth: foot: masc.
%coch: leg: masc.
%curu: beer &gt; masc.
%The –i in the diphthongs –ái-, -éi-, -ói- and –úi- is not a vowel, it is a semi-consonant, like /y/ in English. It does not count as a vowel, and its presence does not make a noun’s gender feminine:
%brói: country &gt; last vowel is /o/, the –i is the semi-consonant &gt; masc.
%mái: place, plain &gt; last vowel is /a/ &gt; fem.
%téi: house &gt; last vowel is /e/ &gt; masc.
%gwólúith: strain &gt; last vowel is /u/ &gt; masc.
%Some nouns end in a double consonant where the last consonant is l, n or r. When pronounced there is a dull indistinct sound between the second last consonant and the l, n or r. This sound is called schwa, and is represented by the symbol [?]. It is not considered a vowel and is not written. It does not affect the gender of a noun. The gender of such a noun is determined by the last vowel before the schwa:
%sédhl: seat &gt; last vowel is /e/ &gt; masc.
%sparn: thorn &gt; last vowel is /a/ &gt; fem
%livr: book &gt; last vowel is /i/ &gt; fem.
%Some nouns end in a diphthong followed by a double consonant where the last consonant is l, n or r. The gender of these nouns is determined by the last vowel before the –i of the diphthong:
%anéithl: protection &gt; last vowel is /e/ &gt; masc.
%lúithr: struggle &gt; last vowel is /u/ &gt; masc.
%bóithl: hit &gt; last vowel is /o/ &gt; masc.
%amáithl: service &gt; last vowel is /a/ &gt; fem.
%Nouns of animals are masculine by default, even if the vowels are /a/ or /i/:
%garan: heron &gt; masc.
%cun: dog &gt; masc.
%lóern: fox &gt; masc.
%ép: horse &gt; masc.
%caval: [draught] horse &gt; masc.
%bó: cow [generic name for cattle]
%These nouns can be made feminine by adding the suffix –is:
%garanis: female heron
%cunis: bitch
%lóernis: vixen
%épis: mare (also casich)
%cavalis: female draught horse
%Nouns indicating human functions or activities are also masculine by default:
%drúidh: scholar &gt; masc.
%gwerchovreth: magistrate &gt; masc.
%tiern: boss, chief &gt; masc.
%dan: official, manager &gt; masc.
%These nouns can also be made feminine by adding the suffix –is:
%drúidhis: female scholar
%gwerchovrethis: female magistrate
%tiernis: female boss, chief
%danis: female official, manager
%<strong>Exercises </strong>
%Determine the gender of the following nouns:
%car: car
%sesa: chair
%roth: wheel
%aríthis: table
%dulu: paper
%cumlath: plate
%cladhal: knife
%gaval: fork
%bóthéi: stable
%bochwídhu: spoon
%ethn: bird
%táru: bull
%amáiath: servant
%cerdhíath: worker
%menrodhiath: teacher
%gnisáiath: student
%pethlói: stuff
%pren: tree
%bil: tree trunk
%clétha: ladder
%cilurn: bucket
%scothir: shovel
%cerdhl: work
%tráith: beach
%crósu: wave
%sir: star
%nem: sky
%brí: hill
%brói: country
%bélói: culture
%tengu: language
%tarinch: nail (fastening implement)
%cingeth: warrior
%delgheth: holder
%druthas: courage
%dumnas: darkness
%échal: hoof
%You can check your answers below.
%<strong>Answers</strong>
%car: car &gt; f
%sesa: chair &gt; f
%roth: wheel &gt; m
%aríthis: table &gt; f
%dulu: paper &gt; m
%cumlath: plate &gt; f
%cladhal: knife &gt; f
%gaval: fork &gt; f
%bóthéi: stable &gt; m
%bochwídhu: spoon &gt; m
%ethn: bird &gt; m
%táru: bull &gt; m
%amáiath: servant &gt; m
%cerdhíath: worker &gt; m
%menrodhiath: teacher &gt; m
%gnisáiath: student &gt; m
%pethlói: stuff &gt; m
%pren: tree &gt; m
%bil: tree trunk &gt; f
%clétha: ladder &gt; f
%cilurn: bucket &gt; m
%scothir: shovel &gt; f
%cerdhl: work &gt; m
%tráith: beach &gt; f
%crósu: wave &gt; m
%sir: star &gt; f
%nem: sky &gt; m
%brí: hill &gt; f
%brói: country &gt; m
%bélói: culture &gt; m
%tengu: language &gt; m
%tarinch: nail (for hammering) &gt; f
%cingeth: warrior &gt; m
%delgheth: holder &gt; m
%druthas: courage &gt; f
%dumnas: darkness &gt; f
%échal: hoof &gt; f
%<strong>                                                                                                           </strong>
%
