\section{Menghavan 5: Rithi\'{u}nan In Elwach\'{\i}dhu}
(\textit{Lesson 5: Plural Formation})\\

In the fifth lesson you will learn how to form the plural of nouns.\\

\subsection{Gwepchoprith: Rithi\'{u}nan}
\subsubsection{Plural of nouns}

The plural of the most nouns is formed by adding the plural suffix \textit{-\'{e}} to the noun.
\begin{table}[H]
\centering
\begin{tabular}{cc}
  \toprule
  \textbf{Noun} & \textbf{Plural}\\
  \cmidrule(lr){1-1}\cmidrule(lr){2-2}
  gwir & gwir\textbf{\'{e}}\\
  mapath & mapath\textbf{\'{e}}\\
  \bottomrule
\end{tabular}
\label{example_noun_plural}
\end{table}

Because it is an open vowel it causes the emphasis to \textit{shift one place} closer to the end of the word. Where the plural ending is not separated from the previous syllable by more than one consonant the vowel of that syllable before the ending \textit{-\'{e}} becomes \textit{long} as well as \textit{emphasised}.
\begin{table}[H]
\centering
\begin{tabular}{ccc}
  \toprule
  \textbf{Noun} & \textbf{Plural} & \textbf{Translation}\\
  \cmidrule(lr){1-1}\cmidrule(lr){2-2}\cmidrule(lr){3-3}
  gwir & gw\textbf{\'{\i}}r\textbf{\'{e}} & men\\
  mapath & map\textbf{\'{a}}th\textbf{\'{e}} & boys\\
  geneth & gen\textbf{\'{e}}th\textbf{\'{e}} & girls\\
  map & m\textbf{\'{a}}p\textbf{\'{e}} & sons\\
  d\'{u}ithir & d\'{u}ith\textbf{\'{\i}}r\textbf{\'{e}} & daughters\\
  \bottomrule
\end{tabular}
\label{example_emphasis_shift}
\end{table}

If the plural ending is separated from the previous syllable by more than one consonant the vowel of that syllable is short.
\begin{table}[H]
\centering
\begin{tabular}{ccc}
  \toprule
  \textbf{Noun} & \textbf{Plural} & \textbf{Translation}\\
  \cmidrule(lr){1-1}\cmidrule(lr){2-2}\cmidrule(lr){3-3}
  ethn & e\textbf{thn\'{e}} & birds\\
  carch & ca\textbf{rch\'{e}} & rocks\\
  \bottomrule
\end{tabular}
\label{example_plural_multiple_consonants}
\end{table}

If a word ends in a vowel the ending \textit{-\'{e}} follows immediately after that vowel, making that vowel emphasised and long.
\begin{table}[H]
\centering
\begin{tabular}{ccc}
  \toprule
  \textbf{Noun} & \textbf{Plural} & \textbf{Translation}\\
  \cmidrule(lr){1-1}\cmidrule(lr){2-2}\cmidrule(lr){3-3}
  c\'{a}nu & can\textbf{\'{u}\'{e}} & songs\\
  \bottomrule
\end{tabular}
\label{example_plural_ends_in_vowel}
\end{table}

Most plurals in Gal\'{a}thach are formed in this way. There are two exceptions only: \textit{woman} and \textit{natural pairs}.

\subsubsection{Plural of woman}

The plural of the word for woman is different.
\begin{table}[H]
\centering
\begin{tabular}{ccc}
  \toprule
  \textbf{Noun} & \textbf{Plural} & \textbf{Translation}\\
  \cmidrule(lr){1-1}\cmidrule(lr){2-2}\cmidrule(lr){3-3}
  ben & mn\'{a} & women\\
  \bottomrule
\end{tabular}
\label{example_plural_of_woman}
\end{table}

This is attested as such in \textit{Old Gaulish}.

\subsubsection{Plural of natural pairs}

The plural of things that naturally occur as pairs is formed by adding the prefix \textit{d\'{a}-}, which means ``two''.
\begin{table}[H]
\centering
\begin{tabular}{ccc}
  \toprule
  \textbf{Noun} & \textbf{Plural} & \textbf{Translation}\\
  \cmidrule(lr){1-1}\cmidrule(lr){2-2}\cmidrule(lr){3-3}
  \'{o}p & \textbf{d\'{a}}\'{o}p & eyes\\
  coch & \textbf{d\'{a}}choch & legs\\
  lam & \textbf{d\'{a}}lam & hands\\
  \bottomrule
\end{tabular}
\label{example_plural_of_natural_pairs}
\end{table}

For the word aus ``ear'' the prefix d\'{a}- becomes shortened to \textit{d-}.
\begin{table}[H]
\centering
\begin{tabular}{ccc}
  \toprule
  \textbf{Noun} & \textbf{Plural} & \textbf{Translation}\\
  \cmidrule(lr){1-1}\cmidrule(lr){2-2}\cmidrule(lr){3-3}
  aus & \textbf{d}aus & ears\\
  \bottomrule
\end{tabular}
\label{example_plural_of_ear}
\end{table}

In cases where these things occur in numbers other than two, the normal plural suffix \textit{-\'{e}} is used.
\begin{table}[H]
\centering
\begin{tabular}{ccM{10.0cm}}
  \toprule
  \textbf{Plural} & \textbf{Translation} & \textbf{Explication}\\
  \cmidrule(lr){1-1}\cmidrule(lr){2-2}\cmidrule(lr){3-3}
  \'{o}p\'{e} damathal & $[$the$]$ eyes of a spider & spiders have eight eyes\\
  coch\'{e} \'{e}p & $[$the$]$ legs of a horse & horses have four legs\\
  \'{o}p\'{e} gw\'{\i}r\'{e} & the eyes of men & several men together have more than two eyes\\
  \bottomrule
\end{tabular}
\label{example_plural_more_than_two}
\end{table}

In cases where things are referred to that are not of natural formation and may or may not come in pairs, the normal plural suffix \textit{-\'{e}} is used.
\begin{table}[H]
\centering
\begin{tabular}{ccM{10.0cm}}
  \toprule
  \textbf{Plural} & \textbf{Translation} & \textbf{Explication}\\
  \cmidrule(lr){1-1}\cmidrule(lr){2-2}\cmidrule(lr){3-3}
  lam\'{e} gwep\'{o}r & the hands of a clock & a clock is not a natural creature, and there may be more than two hands on a clock, e.g.\ hours, minutes, seconds\\
  \bottomrule
\end{tabular}
\label{example_plural_may_or_not}
\end{table}

\subsubsection{Plural of collectivity}

A great number of things that is commonly considered as one whole is indicated by the suffix \textit{-l\'{o}i}.
\begin{table}[H]
\centering
\begin{tabular}{cccccc}
  \toprule
  \textbf{Noun} & \textbf{Translation} & \textbf{Collection Term} & \textbf{Translation} & \textbf{Plural} & \textbf{Translation}\\
  \cmidrule(lr){1-1}\cmidrule(lr){2-2}\cmidrule(lr){3-3}\cmidrule(lr){4-4}\cmidrule(lr){5-5}\cmidrule(lr){6-6}
  gwep & word & gwep\textbf{l\'{o}i} & vocabulary & gwep\textbf{l\'{o}i}\'{e} & vocabularies\\
  sir & star & sir\textbf{l\'{o}i} & constellation & sir\textbf{l\'{o}i}\'{e} & constellations\\
  \bottomrule
\end{tabular}
\label{example_plural_one_whole}
\end{table}

\subsubsection{Plural after numbers}

The plural is not used after numbers. Nouns stay in the singular.
\begin{table}[H]
\centering
\begin{tabular}{cccc}
  \toprule
  \textbf{Noun} & \textbf{Translation} & \textbf{A Number Of} & \textbf{Translation}\\
  \cmidrule(lr){1-1}\cmidrule(lr){2-2}\cmidrule(lr){3-3}\cmidrule(lr){4-4}
  \'{e}p & a horse & pethr \'{e}p & four horses\\
  n\'{o}ith & night & dech n\'{o}ith & ten nights\\
  \bottomrule
\end{tabular}
\label{example_plural_one_whole}
\end{table}

\newpage
\subsection{Exercises}

\subsubsection{Put words into the plural}

Put the following words into the plural:
\begin{table}[H]
\centering
%\resizebox{36pc}{!}{%
\begin{tabular}{|c|c|M{10.0cm}|}
  \toprule
  \textbf{Gal\'{a}thach} & \textbf{English} & \textbf{Answer (m/f)}\\
  \toprule
  car & car & \\
  \midrule
  sesa & chair & \\
  \midrule
  roth & wheel & \\
  \midrule
  ar\'{\i}this & table & \\
  \midrule
  dulu & paper & \\
  \midrule
  cumlath & plate & \\
  \midrule
  cladhal & knife & \\
  \midrule
  gaval & fork & \\
  \midrule
  b\'{o}th\'{e}i & stable & \\
  \midrule
  bochw\'{\i}dhu & spoon & \\
  \midrule
  cilurn & bucket & \\
  \midrule
  cerdhl & work & \\
  \midrule
  tarinch & nail & \\
  \midrule
  cr\'{o}su & wave & \\
  \midrule
  br\'{\i} & hill & \\
  \midrule
  coch & leg & \\
  \midrule
  aus & ear & \\
  \midrule
  d\'{o}s & arm & \\
  \midrule
  durn & fist & \\
  \bottomrule
\end{tabular}
%}
\label{exercise_plural_1}
\caption{Exercise: plural 1}
\end{table}

\newpage
\subsubsection{Solution}
\begin{table}[H]
\centering
%\resizebox{24pc}{!}{%
  \rotatebox{180}{%
    \begin{tabular}{|c|c|>{\itshape}c|}
      \toprule
      \textbf{Gal\'{a}thach} & \textbf{English} & \textbf{Answer (m/f)}\\
      \toprule
      car & car & c\'{a}r\'{e}\\
      \midrule
      sesa & chair & ses\'{a}\'{e}\\
      \midrule
      roth & wheel & r\'{o}th\'{e}\\
      \midrule
      ar\'{\i}this & table & arith\'{\i}s\'{e}\\
      \midrule
      dulu & paper & dul\'{u}\'{e}\\
      \midrule
      cumlath & plate & cuml\'{a}th\'{e}\\
      \midrule
      cladhal & knife & cladh\'{a}l\'{e}\\
      \midrule
      gaval & fork & gav\'{a}l\'{e}\\
      \midrule
      b\'{o}th\'{e}i & stable & b\'{o}th\'{e}i\'{e} $[$/o/ of b\'{o} stays long because it is etymologically determined$]$\\
      \midrule
      bochw\'{\i}dhu & spoon & bochwidh\'{u}\'{e}\\
      \midrule
      cilurn & bucket & cilurn\'{e}\\
      \midrule
      cerdhl & work & cerdhl\'{e}\\
      \midrule
      tarinch & nail & tarinch\'{e}\\
      \midrule
      cr\'{o}su & wave & cros\'{u}\'{e}\\
      \midrule
      br\'{\i} & hill & br\'{\i}\'{e}\\
      \midrule
      coch & leg & c\'{o}ch\'{e}\\
      \midrule
      aus & ear & aus\'{e}\\
      \midrule
      d\'{o}s & arm & d\'{o}s\'{e}\\
      \midrule
      durn & fist & durn\'{e}\\
      \bottomrule
    \end{tabular}
  }
%}
\label{solution_plural_1}
\caption{Solution: plural 1}
\end{table}

\newpage
\subsubsection{Construct the right plural}

Construct the right plural:
\begin{table}[H]
\centering
%\resizebox{36pc}{!}{%
\begin{tabular}{|c|M{10.0cm}|}
  \toprule
  \textbf{English} & \textbf{Answer (m/f)}\\
  \toprule
  legs of a woman & \\
  \midrule
  ears of a girl & \\
  \midrule
  arms of a boy & \\
  \midrule
  eyes of a man & \\
  \midrule
  legs of a dog & \\
  \midrule
  ears of a horse & \\
  \midrule
  arms of a river & \\
  \midrule
  eyes of a crab & \\
  \midrule
  legs of girls & \\
  \midrule
  ears of men & \\
  \midrule
  arms of women & \\
  \midrule
  eyes of boys & \\
  \bottomrule
\end{tabular}
%}
\label{exercise_plural_2}
\caption{Exercise: plural 2}
\end{table}

\newpage
\subsubsection{Solution}

\begin{table}[H]
\centering
%\resizebox{24pc}{!}{%
  \rotatebox{180}{%
    \begin{tabular}{|c|>{\itshape}c|}
      \toprule
      \textbf{English} & \textbf{Answer (m/f)}\\
      \toprule
      legs of a woman & d\'{a}choch ben\\
      \midrule
      ears of a girl & daus geneth\\
      \midrule
      arms of a boy & d\'{a}dh\'{o}s mapath\\
      \midrule
      eyes of a man & d\'{a}\'{o}p gwir\\
      \midrule
      legs of a dog & c\'{o}ch\'{e} cun\\
      \midrule
      ears of a horse & aus\'{e} \'{e}p\\
      \midrule
      arms of a river & d\'{o}s\'{e} \'{a}von\\
      \midrule
      eyes of a crab & \'{o}p\'{e} carchu\\
      \midrule
      legs of girls & c\'{o}ch\'{e} gen\'{e}th\'{e}\\
      \midrule
      ears of men & aus\'{e} gw\'{\i}r\'{e}\\
      \midrule
      arms of women & d\'{o}s\'{e} mn\'{a}\\
      \midrule
      eyes of boys & \'{o}p\'{e} mapath\'{e}\\
      \bottomrule
    \end{tabular}
  }
%}
\label{solution_plural_2}
\caption{Solution: plural 2}
\end{table}
\newpage

\subsubsection{Construct plural phrases}

Use the following numbers to construct plural phrases:\\
\begin{quote}
d\'{a} (two), tr\'{\i} (three), pethr (four), pimp (five)
\end{quote}

\begin{table}[H]
\centering
%\resizebox{36pc}{!}
\label{exercise_plural_3}
\caption{Exercise: plural 3}
\end{table}

\newpage
\subsubsection{Solution}

\begin{table}[H]
\centering
%\resizebox{24pc}{!}{%
  \rotatebox{180}
\label{solution_plural_3}
\caption{Solution: plural 3}
\end{table}
