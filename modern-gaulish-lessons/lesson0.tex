\section{Menghavan 0: Swaus\'{e} In Tengu}
(\textit{Lesson 0: The Sounds Of The Language})\\

\noindent In this preliminary lesson you will learn what the sounds of the language are.

\subsection{Vowels}
\noindent Modern Gaulish has five vowels. They can be short or long. These are the short ones. The table shows how they are written, what their phonetic value is, and what they sound like using examples in English and other languages.

\begin{table}[H]
\begin{center}
\begin{tabu}{|c|c|c|}
  \toprule
  vowel & phonetic value (IPA) & sound examples\\
  \midrule
  a & [a] & pat\\
  o & [o] & pot\\
  u & [u] & put\\
  e & [e] & pet\\
  i & [i] & pit\\
  \bottomrule
\end{tabu}
\end{center}
\caption{Vowels}
\label{phonology_vowels}
\end{table}

\noindent This table shows the long vowels. They are indicated with diacritics over the vowel, e.g. \'{a} is long a.

\begin{table}[H]
\begin{center}
\begin{tabu}{|c|c|c|}
  \toprule
  vowel & phonetic value & sound examples\\
  \midrule
  \'{a} & [a:] & part\\
  \'{o} & [o:] & pole\\
  \'{u} & [u:] & pool\\
  \'{e} & [e:] & pay without the final y\\
  \'{\i} & [i:] & peel\\
  \bottomrule
\end{tabu}
\end{center}
\caption{Long vowels}
\label{phonology_long_vowels}
\end{table}

\noindent Modern Gaulish has five diphthongs. A diphthong is a group of two vowels written and pronounced together. This table shows them.

\begin{table}[H]
\begin{center}
\begin{tabu}{|c|c|c|}
  \toprule
  diphthong & phonetic value (IPA) & sound examples\\
  \'{a}i & [a:j] & bye\\
  \'{o}i & [o:j] & boy\\
  \'{u}i & [u:j] & brouillard (French)\\
  \'{e}i & [e:j] & bay\\
  au & [au] & cow\\
  \bottomrule
\end{tabu}
\end{center}
\caption{Diphthongs}
\label{phonology_diphthongs}
\end{table}

\subsection{Consonants}

\noindent Modern Gaulish has a large number of consonants. The table below shows how they are written, gives their phonetic description, and gives sound examples in English and other languages. It is not possible to provide examples for every sound.

%<table>
%<tbody>
%<tr>
%<td width="205">consonant</td>
%<td width="205">phonetic value (IPA)</td>
%<td width="205">sound examples</td>
%</tr>
%<tr>
%<td width="205">p
%\noindent t
%\noindent c
%\noindent b
%\noindent d
%\noindent g
%\noindent v
%\noindent dh
%\noindent gh
%\noindent f
%\noindent th
%\noindent ch
%\noindent fh
%\noindent m
%\noindent w
%\noindent s
%\noindent sh
%\noindent n
%\noindent r
%\noindent l
%\noindent nh
%\noindent rh
%\noindent lh
%\noindent ng</td>
%<td width="205">[p]
%\noindent [t]
%\noindent [k]
%\noindent [b]
%\noindent [d]
%\noindent [g]
%\noindent [v]
%\noindent [�]
%\noindent [?]
%\noindent [f]
%\noindent [?]
%\noindent [x]
%\noindent [?]
%\noindent [m]
%\noindent [w]
%\noindent [s]
%\noindent [?]
%\noindent [n]
%\noindent [r]
%\noindent [l]
%\noindent [xn]
%\noindent [xr]
%\noindent [xl]
%\noindent [?]</td>
%<td width="205">pit
%\noindent tit
%\noindent kit
%\noindent boar
%\noindent door
%\noindent gore
%\noindent very
%\noindent there
%\noindent * ???, ego, modern Greek �I�
%\noindent fin
%\noindent thin
%\noindent * loch, Scottish; ich, German
%\noindent * f with no tongue on teeth
%\noindent may
%\noindent way
%\noindent sit
%\noindent shit
%\noindent nose
%\noindent rose
%\noindent lose
%\noindent * [x] followed by [n]
%\noindent * [x] followed by [r]
%\noindent * [x] followed by [l]
%\noindent sing</td>
%</tr>
%</tbody>
%</table>
%\noindent <strong>Vowel length variation</strong>
%\noindent The length of a vowel can change. In a word of two syllables or more the emphasis will be on the second last syllable. Often this will make the vowel of that syllable long. Examples are given below.
%\noindent men: to think &gt; vowel /e/ is short
%\noindent m�nu: thought &gt; emphasis on first vowel /e/ which becomes long
%\noindent men��: thoughts &gt; emphasis shifts to second last vowel /u/ which becomes long
%

