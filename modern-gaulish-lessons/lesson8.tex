\section{Menghavan 8: Rivrethr\'{e}}
(\textit{Lesson 8: Adverbs})\\

In the eighth lesson you will learn about adverbs.

\subsection{Adverbs}

An adverb is a word that describes a quality or a characteristic of a verbal action.\\
English examples are: quickly, quietly, calmly, naturally, normally. In English they are usually formed with the suffix \textit{-ly}.\\
In Gal\'{a}thach adverbs are formed by placing the particle \textit{in} in front of an adjective. The adjective undergoes mutation of its initial consonant.
\begin{table}[H]
\centering
\begin{tabular}{cc}
  \toprule
  \textbf{Gal\'{a}thach} & \textbf{English}\\
  \cmidrule(lr){1-1}\cmidrule(lr){2-2}
  \'{a}chu & quick\\
  in h\'{a}chu & quickly\\
  \'{a}va mi ch\'{\i} & I do it\\
  \'{a}va mi ch\'{\i} in h\'{a}chu & I do it quickly\\
  \cmidrule(lr){1-1}\cmidrule(lr){2-2}
  tau & quiet\\
  in dau & quietly\\
  sp\'{a} \'{\i} ch\'{\i} & she says it\\
  sp\'{a} \'{\i} ch\'{\i} in dau & she says it quietly\\
  \cmidrule(lr){1-1}\cmidrule(lr){2-2}
  aram & calm\\
  in haram & calm\\
  r\'{e}na in avon & the river flows\\
  r\'{e}na in avon in haram & the river flows calmly\\
  \cmidrule(lr){1-1}\cmidrule(lr){2-2}
  amv\'{\i}thach & natural\\
  in hamv\'{\i}thach & naturally\\
  gw\'{o}ra cr\'{a}r\'{e} mel & bees produce honey\\
  gw\'{o}ra cr\'{a}r\'{e} mel in hamv\'{\i}thach & bees produce honey naturally\\
  \cmidrule(lr){1-1}\cmidrule(lr){2-2}
  suves & normal\\
  in shuves & normally\\
  n\'{e} chwergha \'{\i} co sh\'{e} & she doesn't act like that\\
  n\'{e} chwergha \'{\i} co sh\'{e} in shuves & she doesn't normally act like that\\
  \bottomrule
\end{tabular}
\label{examples_adverb}
\end{table}

\subsubsection{Position Of The Adverb}

V = verb \\
S = subject\\
O = object \\
A= adverb \\

The adverb always follows the verb as closely as possible, after the subject and object of the phrase.
\begin{table}[H]
\centering
\begin{tabular}{ccccccccc}
  gw\'{o}ra & cr\'{a}r\'{e} & mel & in hamv\'{\i}thach & $\rightarrow$ & produce & bees & honey & naturally\\
  V & S & O & A & & V & S & O & A
\end{tabular}
\label{examples_adverb_order}
\end{table}

\newpage
\subsubsection{Exercises}

Construct the adverbial form of the following adjectives. You can check your answers at the end of the lesson.

\begin{table}[H]
\centering
\begin{tabular}{|c|c|M{5.0cm}|M{5.0cm}|}
  \toprule
  \multicolumn{2}{|c}{\textbf{Adjective}} & \multicolumn{2}{|c|}{\textbf{Adverb}}\\
  \toprule
  \textbf{Gal\'{a}thach} & \textbf{English} & \textbf{Gal\'{a}thach} & \textbf{English}\\
  \toprule
  c\'{o}il & narrow & & \\
  \midrule
  lithan & wide & & \\
  \midrule
  dianauch & poor & & \\
  \midrule
  t\'{e}ithw\'{a}r & wealthy & & \\
  \midrule
  ardhu & high & & \\
  \midrule
  \'{\i}th & low & & \\
  \midrule
  pethrarpenach & square & & \\
  \midrule
  r\'{o}thach & round & & \\
  \midrule
  d\'{a}i & good & & \\
  \midrule
  druch & bad & & \\
  \bottomrule
\end{tabular}
\label{exercise_adverbs}
\caption{Exercise: adverbs}
\end{table}

\newpage
Solution:
\begin{table}[H]
\centering
\rotatebox{180}{%
  \begin{tabular}{|c|c|>{\itshape}c|>{\itshape}c|}
    \toprule
    \multicolumn{2}{|c}{\textbf{Adjective}} & \multicolumn{2}{|c|}{\textit{\textbf{Adverb}}}\\
    \toprule
    \textbf{Gal\'{a}thach} & \textbf{English} & \textbf{Gal\'{a}thach} & \textbf{English}\\
    \toprule
    c\'{o}il & narrow & in g\'{o}il & narrowly\\
    \midrule
    lithan & wide & in lhithan & widely\\
    \midrule
    dianauch & poor & in dhianauch & poorly\\
    \midrule
    t\'{e}ithw\'{a}r & wealthy & in d\'{e}ithw\'{a}r & wealthily\\
    \midrule
    ardhu & high & in hardhu & highly\\
    \midrule
    \'{\i}th & low & in h\'{\i}th & lowly\\
    \midrule
    pethrarpenach & square & in betharpenach & squarely\\
    \midrule
    r\'{o}thach & round & in rhothach & roundly\\
    \midrule
    d\'{a}i & good & in dh\'{a}i & well\\
    \midrule
    druch & bad & in dhruch & badly\\
    \bottomrule
  \end{tabular}
}
\label{solution_adverbs}
\caption{Solution: adverbs}
\end{table}
