\section{Menghavan 17: Cos\'{a}n \textendash\ Coviras \textendash\ R\'{i}m\'{e}}
(\textit{Lesson 17: Comparison \textendash\ Diminutive \textendash\ Numbers})\\

In the seventeenth lesson, you will learn how te comparison, diminutive and numbers systems work in Gal'{a}thach. 

Lesson 17 In Da Tei

Tr\'{e}lav\'{a}ru / Translation:

``My house is more beautiful than yours.''
``It is not. My house is bigger than yours.''
``It is. But your house is shit.''

\subsection{Gwepchoprith: Conversation}
\subsubsection{Conversation}

%\tcbox[colback=red!10!green!10, colframe=green!20!black!80]
\begin{table}[H]
\centering
    \begin{tabular}{M{3.0cm}M{10.0cm}M{3.0cm}}
    \cellcolor{lightgreen} & \cellcolor{lightgreen} & \cellcolor{lightgreen}\\
    \cellcolor{lightgreen}\textcolor{darkgreen}{\textbf{Anerghu}} & \cellcolor{lightgreen} & \cellcolor{lightgreen}\textcolor{darkgreen}{\textbf{Bochu}}\\
    \cellcolor{lightgreen} & \cellcolor{lightgreen} & \cellcolor{lightgreen}\\
    \cellcolor{lightgreen} & \cellcolor{lightgreen} & \cellcolor{lightgreen}\\
    \cellcolor{lightgreen} & \cellcolor{lightgreen} & \cellcolor{lightgreen}\\
    \cellcolor{lightgreen} & \cellcolor{lightgreen} & \cellcolor{lightgreen}\\
    \cellcolor{lightgreen} & \cellcolor{lightgreen} & \cellcolor{lightgreen}\\
    \cellcolor{lightgreen} & \cellcolor{lightgreen} & \cellcolor{lightgreen}\\
%    \cellcolor{lightgreen}\multirow{-7}{*}{\includegraphics[height=4.0cm]{img/menghavan11_1}} & \cellcolor{lightgreen} & \cellcolor{lightgreen}\multirow{-7}{*}{\includegraphics[height=4.0cm]{img/menghavan11_2}}\\
    \cellcolor{lightgreen} & \cellcolor{lightgreen} & \cellcolor{lightgreen}
    \end{tabular}
\end{table}

\begingroup
\fontsize{10pt}{12pt}\selectfont
\begin{leftbubbles}Esi m\'{o} d\'{e}i t\'{e}i tech.\end{leftbubbles}
\begin{rightbubbles}Esi \'{\i}, in chw\'{\i}r. \'{e}ithr esi m\'{o} d\'{e}i im\'{\i} gwer dech.\end{rightbubbles}
\begin{leftbubbles}N\'{e} hesi, esi t\'{o} d\'{e}i co dech co in \'{\i}mi, \'{e}ithr n\'{e} chwer dech.\end{leftbubbles}
\begin{rightbubbles}Mi-esi com\'{o}inan, esi ti ancherth. Esi m\'{o} d\'{e}i im\'{\i} in t\'{e}i techam.\end{rightbubbles}
\begin{leftbubbles}Ti-esi t\'{o} ben en in nem\'{u}\'{e}. N\'{e} hesi t\'{o} d\'{e}i t\'{e}i bithw\'{\i}r cotham, esi \'{\i} in h\'{o}nach t\'{e}ial.\end{leftbubbles}
\begin{rightbubbles}B\'{u} m\'{o} d\'{e}i in t\'{e}i cin o b\'{u} carn\'{\i}thu.\end{rightbubbles}
\begin{leftbubbles}Gals\'{\i} \'{\i} bis certh o b\'{u} m\'{o} d\'{e}i in t\'{e}i c\'{\i}al o b\'{u} carn\'{\i}thu, \'{e}ithr esi \'{\i} gwer dh\'{a}i \'{e}lu co in ith\'{\i}.\end{leftbubbles}
\begin{rightbubbles}\'{E}ithr carnis\'{\i} mi t\'{e}i tr\'{\i}thu, ach b\'{\i} \'{\i} gwer dh\'{a}i ath\'{e}.\end{rightbubbles}
\begin{leftbubbles}Surathu adhith. Cans\'{o}ithis\'{\i} mi aven dech, ach ton anthas\'{\i} mi in colav\'{a}ru duchw\'{\i}s sin ach techis\'{\i} mi. On, d\'{a}, tr\'{\i}...\end{leftbubbles}
\begin{rightbubbles}Ton gwerthis\'{\i} mi in t\'{e}i\'{e} pethuar ach pimpeth en dhun\'{e}.\end{rightbubbles}
% TODO: do something about the text  between square brackets.
\begin{leftbubbles}... pethr, pimp, swech ... [\'{a}dha \'{e} \'{o} vis\'{e} en \'{o} dhachlus]\end{leftbubbles}
\begin{rightbubbles}Ton, b\'{\i} in t\'{e}i\'{e} swechweth, s\'{e}ithweth ach \'{o}ithweth b\'{o}th\'{e}i\'{e} ri m\'{o} gwochon-pimp \'{e}p ...\end{rightbubbles}
\begin{leftbubbles}... s\'{e}ith, \'{o}ith ...\end{leftbubbles}
\begin{rightbubbles}... en in t\'{e}i nameth delghes\'{\i} mi mo dachwochon-tridhech cun sonithi ...\end{rightbubbles}
\begin{leftbubbles}... n\'{a} ...\end{leftbubbles}
\begin{rightbubbles}... ach b\'{\i} in t\'{e}i dechweth t\'{e}i'churu, p\'{e}m\'{a}i delghes\'{\i} mi m\'{o} trichwochon-pimdhech tun u guru.\end{rightbubbles}
\begin{leftbubbles}\'{a}! P\'{e}r\'{\i} n\'{e} a 'p\'{a}thu ti s\'{e} in gov\'{\i}on. \'{a}i ni a gh\'{a}vi curu! T\'{o} chwerthan a dioprithi!\end{leftbubbles}
\endgroup

\newpage
\subsubsection{Colav\'{a}ru \textendash\ Tr\'{e}lav\'{a}ru}
(Conversation \textendash\ Translation)\\

This is a dialogue between two men, Anerghu (The Very Red One) and Bochu (The Mouth).

Anerghu: Esi m\'{o} d\'{e}i t\'{e}i tech.
(A: My house is a beautiful house.)

Bochu: Esi \'{\i}, in chw\'{\i}r. \'{e}ithr esi m\'{o} d\'{e}i im\'{\i} gwer dech.
(B: Yes, it is. But my house [of-me] is more beautiful.)

A: N\'{e} hesi, esi t\'{o} d\'{e}i co dech co in \'{\i}mi, \'{e}ithr n\'{e} chwer dech.
(A: No, your house is as beautiful as mine [the of-me], but not more beautiful.)

B: Mi-esi com\'{o}inan, esi ti ancherth. Esi m\'{o} d\'{e}i im\'{\i} in t\'{e}i techam.
(B: I'm sorry, you are wrong. My house is the most beautiful house.)

A: Ti-esi t\'{o} ben en in nem\'{u}\'{e}. N\'{e} hesi t\'{o} d\'{e}i t\'{e}i bithw\'{\i}r cotham, esi \'{\i} in h\'{o}nach t\'{e}ial.
(A: You are deluded [your head is in the clouds]. Your house is not even a real house, it's only a shack [houselet].)

B: B\'{u} m\'{o} d\'{e}i in t\'{e}i cin o b\'{u} carn\'{\i}thu.
(B: My house was the first house that was built.)

A: Gals\'{\i} \'{\i} bis certh o b\'{u} m\'{o} d\'{e}i in t\'{e}i c\'{\i}al o b\'{u} carn\'{\i}thu, \'{e}ithr esi \'{\i} gwer dh\'{a}i \'{e}lu co in ith\'{\i}.
(A: My house might have been the second house that was built, but it is much better than yours [the of-you].)

B: \'{e}ithr carnis\'{\i} mi t\'{e}i tr\'{\i}thu, ach b\'{\i} \'{\i} gwer dh\'{a}i ath\'{e}.
(B: But I will build a third house, and it will be better again.)

A: Surathu adhith. Cans\'{o}ithis\'{\i} mi aven dech, ach ton anthas\'{\i} mi in colav\'{a}ru duchw\'{\i}s sin ach techis\'{\i} mi. On, d\'{a}, tr\'{\i}...
(A: Good luck. I will count to ten, and then I will finish this stupid conversation and walk away. One, two, three, ...)

B: Ton gwerthis\'{\i} mi in t\'{e}i\'{e} pethuar ach pimpeth en dhun\'{e}.
(B: Then the fourth and fifth houses I will turn into fortresses.)

A: ... pethr, pimp, swech ... [\'{a}dha \'{e} \'{o} vis\'{e} en \'{o} dhachlus]
(A: ... four, five, six ... [puts his fingers in his ears])

B: Ton, b\'{\i} in t\'{e}i\'{e} swechweth, s\'{e}ithweth ach \'{o}ithweth b\'{o}th\'{e}i\'{e} ri m\'{o} gwochon-pimp \'{e}p ...
(B: Then the sixth, seventh and eighth houses will be stables for my twenty-five horses ...)

A: ... s\'{e}ith, \'{o}ith ...
(A: ... seven, eight ...)

B: ... en in t\'{e}i nameth delghes\'{\i} mi mo dachwochon-tridhech cun sonithi ...
(B: ... in the nineth house I will keep my fifty-three hunting dogs ...)

A: ... n\'{a} ...
(A: ... nine ...)

B: ... ach b\'{\i} in t\'{e}i dechweth t\'{e}i'churu, p\'{e}m\'{a}i delghes\'{\i} mi m\'{o} trichwochon-pimdhech tun u guru.
(B: ... and the tenth house will be a pub, where I will keep my seventy-five barrels of beer.)

A: \'{a}! P\'{e}r\'{\i} n\'{e} a 'p\'{a}thu ti s\'{e} in gov\'{\i}on. \'{a}i ni a gh\'{a}vi curu! T\'{o} chwerthan a dioprithi!
(A: Ah! Why didn't you say that immediately! Let's go for a beer! Your turn to pay!)

Anerghu < Andergus, \textit{the very red one}, attested.
Bochu < Bocco, \textit{the mouth}, attested.

\subsection{Gwepchoprith: Comparison}

\subsubsection{Comparison}

There are three levels of comparison in Gal\'{a}thach:
\begin{enumerate}
\item one thing is \textit{as good as} another thing
\item one thing is \textit{better} then another thing
\item one thing is \textit{the best}
\end{enumerate}

\subsubsection{One thing is as good as another thing}

To express this, the word \textit{co}, meaning \textit{as, like, same, similar} is used two times. \textit{Co} causes Initial Consonant Mutation.

tech: beautiful
canech: gold
\begin{quote}
esi in t\'{e}i co dech co ganech (the house is as beautiful as gold)
\end{quote}

\'{a}chu: fast
l\'{o}cheth: lightning
\begin{quote}
esi in \'{e}p co h\'{a}chu co lh\'{o}cheth: the horse is as fast as lightning
\end{quote}

\subsubsection{One thing is better than another thing}

To express this, the word \textit{gwer}, meaning \textit{on, over} is used. It causes ICM. It can be used by itself, without reference to other things.

\begin{quote}
esi in t\'{e}i-sin gwer dech (this house is more beautiful)
esi in \'{e}p-sin gwer h\'{a}chu (this horse is faster)
\end{quote}

It can also be used in combination with the word \textit{co} to refer to other things.

\begin{quote}
esi in t\'{e}i-sin gwer dech co ganech (this house is more beautiful than gold)
esi in \'{e}p-sin gwer hachu co lh\'{o}cheth (this horse is faster than lightning)
\end{quote}

\subsubsection{A thing is the best}

To express this the suffix \textit{-am} is used.

\begin{quote}
esi in t\'{e}i-sin techam (this house is most beautiful) [this house is the most beautiful one]
esi in \'{e}p-sin ach\'{u}am (this horse is fastest) [this horse is the fastest one]
\end{quote}

This system is also used in the comparitive value system described in lesson 15:

\begin{quote}
d\'{a}i - gwer dh\'{a}i- d\'{a}isam $\rightarrow$ good - better - best
mes - gwer wes - mesam $\rightarrow$ bad - worse - worst
\end{quote}

Examples from the conversation above:

\begin{quote}
\'{e}ithr esi m\'{o} d\'{e}i im\'{\i} gwer dech (but my house [of-me] is more beautiful)
\end{quote}
In this sentence the word \textit{im\'{\i}}, meaning \textit{of-me}, \textit{mine}, is added to give emphasis to the statement.

\begin{quote}
N\'{e} hesi, esi t\'{o} d\'{e}i co dech co in \'{\i}mi: No, your house is as beautiful as mine.
\end{quote}
In this sentence the phrase \textit{n\'{e} hesi}, meaning \textit{[it] is not} is used to say ``no''.
The phrase \textit{in im\'{\i}}, meaning \textit{the of-me} is used to say ``mine''. It is used to refer to a thing without repeating it. It is the equivalent of English \textit{mine} and French \textit{le mien}.

\begin{quote}
Esi m\'{o} d\'{e}i im\'{\i} in t\'{e}i techam: my house is the most beautiful house.
\end{quote}
In this sentence the phrase \textit{in t\'{e}i techam} means ``the most beautiful house''. The word \textit{techam}, meaning \textit{most beautiful}, is an adjective and therefore follows the noun it says something about.

\subsection{Gwepchoprith: Diminutive}
\subsubsection{Diminutive}

The diminutive in Gal\'{a}thach is formed with the suffix -al.

Example from the conversation above:

\begin{quote}
N\'{e} hesi t\'{o} d\'{e}i t\'{e}i bithw\'{\i}r cotham, esi \'{\i} in h\'{o}nach t\'{e}ial (your house is not even a real house, it is only a shack)
\end{quote}

%TODO: tables
t\'{e}i: house
> t\'{e}ial: shack [little house]

It is often used for words that describe a \textit{smaller version} of something:
\'{e}p: horse
> \'{e}pal: foal

gnath: child
> gnathal: baby

\subsection{Gwepchoprith: Numbers}
\subsubsection{Cardinal numbers}

%TODO: tables
The cardinal numbers of Gal\'{a}thach are listed throughout the conversation above. A person counts to ten:
> on, d\'{a}, tr\'{\i}, pethr, pimp, swech, s\'{e}ith, \'{o}ith, n\'{a}, dech
one, two, three, four, five, six, seven, eight, nine, ten

The numbers from one to twenty form the basis for the formation of all other numbers.
> onech, d\'{a}dhech, tr\'{\i}dhech, pethrdhech, pimdhech,
eleven, twelve, thirteen, fourteen,  fifteen
> swechdhech, s\'{e}idhech, \'{o}idhech, n\'{a}dhech, gwochon
sixteen,  seventeen, eighteen, nineteen, twenty

After twenty the numbering starts again:
> gwochon-on, gwochon-d\'{a}, gwochon-tr\'{\i}, gwochon-pethr, gwochon-pimp
twenty-one, twenty-two, twenty-three, twenty-four, twenty-five
> gwochon-swech, gwochon-s\'{e}ith, gwochon-\'{o}ith, gwochon-n\'{a}, gwochon-dech
twenty-six, twenty-seven, twenty-eight, twenty-nine, thirty.

The word thirty is made up of twenty + ten: gwochon-dech [twenty-ten].

The numbers from thirty to forty are made by combining twenty with the numbers eleven to nineteen:
> gwochon-onech, gwochon-d\'{a}dhech, gwochon-tridhech, gwochon-pethrdech, gwochon-pimdhech
thirty-one, thirty-two, thirty-four, thirty-four, thirty-five

and so on.

The number forty is made up of the number two + twenty: d\'{a}chwochon [two-twenty].

Then the pattern starts again. The number fifty is two + twenty + ten: dachwochon-dech [two-twenty-ten].

The numbers from fifty to sixty are made by combining the numbers two-twenty with the numbers eleven to nineteen:
> d\'{a}chwochon-onech, d\'{a}chochon-dadhech, d\'{a}chochon-tr\'{\i}dhech ...
fifty-one, fifty-two, fifty-three
[two-twenty-eleven, two-twenty-twelve, two-twenty-thirteen ...]

The number sixty is made up of the number three + twenty: trichwochon.

Then the pattern starts again.

The number forty is made up of the number four + twenty: pethrchwochon

Then the pattern starts again.

The number one hundred is \textit{can}.

The number one thousand is \textit{mil}.

\subsubsection{Ordinal numbers}

The ordinal numbers are listed throughout the conversation above. The ordinal numbers one to five (first to fifth) are irregular.

cin: first
c\'{\i}al: second
tr\'{\i}thu: third
peth\'{u}ar: fourth
pimpeth: fifth

All other ordinal numbers except ones ending on \textit{nine} are formed regularly by adding the suffix \textit{-weth} to the cardinal number.

swechweth: sixth
s\'{e}ithweth: seventh
\'{o}ithweth: eighth
dechweth: tenth
onechweth: eleventh
dadhechweth: twelfth
pimdhechweth: fifteenth
swechdhechweth: sixteenth
n\'{a}dhechweth: nineteenth
gwochonweth: twentieth
dachwochonweth: fortieth
trichwochonweth: sixtieth
pethrchwochonweth: eightieth
canweth: hundredth
milweth: thousandth

Ordinal numbers ending on \textit{nine} are formed by adding the suffix \textit{-meth} to the cardinal number.

n\'{a}meth: nineth
gwochon-n\'{a}meth: twenty-nineth
dachwochon-n\'{a}meth: forty-nineth
trichwochon-n\'{a}meth: sixty-nineth

\subsection{Exercises}

\subsubsection{Vocabulary}

\begin{table}[H]
\centering
\begin{tabular}{cc}
  \toprule
  \textbf{Gal\'{a}thach} & \textbf{English}\\
  t\'{e}i & house\\
  \'{e}p & horse\\
  br\'{\i} & mountain\\
  moch & pig\\
  depri & to eat\\
  bargh & haystack\\
  m\'{o}rchun & dolphin\\
  caur'w\'{o}r & whale\\
  tr\'{a}ieth & beach\\
  cingeth & warrior\\
  bathi & to fight\\
  derthol & naked\\
  dr\'{u}idh & druid\\
  c\'{e}th & forest\\
  n\'{o}iv & sacred\\
  d\'{\i} & day\\
  \'{o}s & after\\
  sam & summer\\
  s\'{a}n's\'{u}el & solstice\\
  gwin & white\\
  t\'{a}ru & bull\\
  averthwi & to sacrifice\\
  \bottomrule
\end{tabular}
\label{vocab_exercise_lesson17}
\end{table}
 
\subsubsection{Translate}

Translate the following phrases.

The first house is bigger than the second house.
The third horse is as fast as the fourth horse.
The fifth mountain is the highest one.
The seven little pigs have eaten fourteen haystacks.
The twenty-three little dolphins jumped out of the water.
There are thirty-one little whales dead on the beach.
The forty-eighth warrior was fighting naked. [*attention: naked is here used as an adverb]
There were seventy-five druids in the sacred forest.
On the eighty-nineth day after the summer solstice the little white bull was sacrificed.

\newpage
\subsubsection{Solution}

Esi in t\'{e}i cin gwer w\'{a}r co'n t\'{e}i c\'{\i}al.
Esi in \'{e}p tr\'{\i}thu co h\'{a}chu co in \'{e}p peth\'{u}ar.
Esi in vr\'{\i} pimpeth hardh\'{u}am.
Depr\'{\i}thu in s\'{e}ith mochal pethrdhech bargh.
R\'{e} shuling in gwochon-tr\'{\i} m\'{o}rchunal e in duvr.
Esi gwochon-onech caur'w\'{o}ral marus gwer in dr\'{a}ith.
B\'{u} in dachwochon-\'{o}ithweth cingeth en vathi in dherthol.
B\'{u} trichwochon-pimdhech dr\'{u}idh en in c\'{e}th n\'{o}iv.
A'n dh\'{\i} pethrchwochon-nameth \'{o}s in sh\'{a}n's\'{u}el sam b\'{u} averthw\'{\i}thu in tar\'{u}al gwin.
