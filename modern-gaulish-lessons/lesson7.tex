\section{Menghavan 7: Achath\'{e}n\'{e} \textemdash\ In Br\'{e}thr \textit{esi}}
(\textit{Lesson 7: Adjectives \textemdash\ The Verb \textit{to be}})\\

In the seventh lesson you will learn about adjectives, and how to use the verb \textit{to be}.

\subsection{Gwepchoprith \textendash\ Adjectives}

An adjective is a word that describes a quality or a characteristic of a noun.
\begin{table}[H]
\centering
\begin{tabular}{cc}
  \toprule
  \textbf{Gal\'{a}thach} & \textbf{English}\\
  \cmidrule(lr){1-1}\cmidrule(lr){2-2}
  sen & old\\
  i\'{o}inch & young\\
  s\'{\i}r & long\\
  bir & short\\
  m\'{a}r & big\\
  m\'{e}i & small\\
  ardhu & high\\
  \'{\i}th & low\\
  duv & black\\
  gwin & white\\
  galv & fat\\
  d\'{a}i & good\\
  druch & bad\\
  \bottomrule
\end{tabular}
\label{examples_adjective}
\end{table}

There are different kinds of adjectives. The ones listed above are all natural adjectives. Adjectives can also be formed from other words by using suffixes and prefixes.

\subsubsection{The Suffix \textit{-ach}}

The suffix \textit{-ach} can be used to make an adjective from another word, most usually a noun.
\begin{table}[H]
\centering
\begin{tabular}{cccc}
  \toprule
  \multicolumn{2}{c}{\textbf{Noun}} & \multicolumn{2}{c}{\textbf{Noun with suffix (adjective)}}\\
  \midrule
  \textbf{Gal\'{a}thach} & \textbf{English} & \textbf{Gal\'{a}thach} & \textbf{English}\\
  \cmidrule(lr){1-1}\cmidrule(lr){2-2}\cmidrule(lr){3-3}\cmidrule(lr){4-4}
  caran & friend & caranach & friendly\\
  carath & love & car\'{a}thach & loveable\\
  nerth & strength & nerthach & strong\\
  caun & owl & caunach & owlish\\
  barn & judgement & barnach & judgemental\\
  achaun & stone & achaunach & stony\\
  \bottomrule
\end{tabular}
\label{examples_suffix_ach}
\end{table}

\subsubsection{The Suffix \textit{-ch}}
The suffix \textit{-ch} is similar to the suffix -ach. It is used for words ending on a diphthong in \textit{-i} or on \textit{-u}, where -ach would be impractical.
\begin{table}[H]
\centering
\begin{tabular}{cccc}
  \toprule
  \multicolumn{2}{c}{\textbf{Noun}} & \multicolumn{2}{c}{\textbf{Noun with suffix (adjective)}}\\
  \midrule
  \textbf{Gal\'{a}thach} & \textbf{English} & \textbf{Gal\'{a}thach} & \textbf{English}\\
  \cmidrule(lr){1-1}\cmidrule(lr){2-2}\cmidrule(lr){3-3}\cmidrule(lr){4-4}
  grau & sand & grauch & sandy\\
  t\'{e}i & house & t\'{e}ich & domestic\\
  \bottomrule
\end{tabular}
\label{examples_suffix_ch}
\end{table}

\subsubsection{The Suffix \textit{-\'{\i}dhu}}

The suffix \textit{-\'{\i}dhu} is used for words that end on \textit{-ch}, to avoid doubling up of the -ch- sound.
\begin{table}[H]
\centering
\begin{tabular}{cccc}
  \toprule
  \multicolumn{2}{c}{\textbf{Noun}} & \multicolumn{2}{c}{\textbf{Noun with suffix (adjective)}}\\
  \midrule
  \textbf{Gal\'{a}thach} & \textbf{English} & \textbf{Gal\'{a}thach} & \textbf{English}\\
  \cmidrule(lr){1-1}\cmidrule(lr){2-2}\cmidrule(lr){3-3}\cmidrule(lr){4-4}
  boch & mouth & boch\'{\i}dhu & mouthy\\
  coch & leg & cochidhu & leggy\\
  carch & rock & carch\'{\i}dhu & rocky\\
  bruch & heather & bruch\'{\i}dhu & heathery\\
  \bottomrule
\end{tabular}
\label{examples_suffix_iidhu}
\end{table}

\subsubsection{The Suffix \textit{-in}}

The suffix \textit{-in} is only used to describe qualities of living creatures, people, animals etc.
\begin{table}[H]
\centering
\begin{tabular}{cccc}
  \toprule
  \multicolumn{2}{c}{\textbf{Noun}} & \multicolumn{2}{c}{\textbf{Noun with suffix (adjective)}}\\
  \midrule
  \textbf{Gal\'{a}thach} & \textbf{English} & \textbf{Gal\'{a}thach} & \textbf{English}\\
  \cmidrule(lr){1-1}\cmidrule(lr){2-2}\cmidrule(lr){3-3}\cmidrule(lr){4-4}
  bledh & wolf & bledhin & wolf-like (lupine)\\
  gwir & man & gwirin & masculine\\
  ben & woman & benin & feminine\\
  \'{e}p & horse & \'{e}pin & equine\\
  cun & dog & cunin & canine\\
  ernu & eagle & ern\'{u}in & aquiline\\
  \bottomrule
\end{tabular}
\label{examples_suffix_in}
\end{table}

\subsubsection{The Prefixes \textit{su-} And \textit{du-}}

The prefixes \textit{su-} (good) and \textit{du-} (bad) can be attached to nouns to create adjectives.
\begin{table}[H]
\centering
\begin{tabular}{cccc}
  \toprule
  \multicolumn{2}{c}{\textbf{Noun}} & \multicolumn{2}{c}{\textbf{Noun with suffix (adjective)}}\\
  \midrule
  \textbf{Gal\'{a}thach} & \textbf{English} & \textbf{Gal\'{a}thach} & \textbf{English}\\
  \cmidrule(lr){1-1}\cmidrule(lr){2-2}\cmidrule(lr){3-3}\cmidrule(lr){4-4}
  \'{a}iedh & face & \textbf{su}\'{a}iedh & good looking\\
  & & \textbf{du}\'{a}iedh & ugly\\
  \bottomrule
\end{tabular}
\label{examples_suffix_su_du}
\end{table}

If the emphasis in these constructions is not on the vowel /u/ of the prefixes, they become \textit{sw-} and \textit{dw-}
\begin{table}[H]
\centering
\begin{tabular}{cc}
  \toprule
  \multicolumn{2}{c}{\textbf{Noun with suffix (adjective)}}\\
  \midrule
  \textbf{Gal\'{a}thach} & \textbf{English}\\
  \cmidrule(lr){1-1}\cmidrule(lr){2-2}
  \textbf{sw}\'{a}iedh & good looking\\
  \textbf{dw}\'{a}iedh & ugly\\
  \bottomrule
\end{tabular}
\label{examples_suffix_sw_dw}
\end{table}

\subsubsection{Position Of The Adjective}

An adjective always follows the nouns it says something about.
\begin{table}[H]
\centering
\begin{tabular}{cccc}
  \toprule
  \multicolumn{2}{c}{\textbf{Noun}} & \multicolumn{2}{c}{\textbf{Noun followed by adjective}}\\
  \midrule
  \textbf{Gal\'{a}thach} & \textbf{English} & \textbf{Gal\'{a}thach} & \textbf{English}\\
  \cmidrule(lr){1-1}\cmidrule(lr){2-2}\cmidrule(lr){3-3}\cmidrule(lr){4-4}
  gwir & man & gwir caranach & a friendly man\\
  mapath & boy & mapath m\'{e}i & a little boy\\
  cun & dog & cun duv & a black dog\\
  \'{e}p & horse & \'{e}p gwin & a white horse\\
  \bottomrule
\end{tabular}
\label{examples_adjective_order}
\end{table}

\subsubsection{Mutation Of The Adjective}

If an adjective says something about a feminine noun it undergoes mutation of its initial consonant. This is how gender is indicated.
\begin{table}[H]
\centering
\begin{tabular}{cccccc}
  \toprule
  \multicolumn{2}{c}{\textbf{Noun}} & \multicolumn{2}{c}{\textbf{Adjective}} & \multicolumn{2}{c}{\textbf{Mutation with feminine noun}}\\
  \midrule
  \textbf{Gal\'{a}thach} & \textbf{English} & \textbf{Gal\'{a}thach} & \textbf{English} & \textbf{Gal\'{a}thach} & \textbf{English}\\
  \cmidrule(lr){1-1}\cmidrule(lr){2-2}\cmidrule(lr){3-3}\cmidrule(lr){4-4}\cmidrule(lr){5-5}\cmidrule(lr){6-6}
  ben & woman & tech & beautiful & ben dech & a beautiful woman\\
  \'{e}pis & a mare & \'{a}chu & fast & \'{e}pis h\'{a}chu & a fast mare\\
  geneth & girl & gwimp & pretty & geneth chwimp\\
  \bottomrule
\end{tabular}
\label{examples_adjective_order}
\end{table}

The adjective does not change for the plural.
\begin{table}[H]
\centering
\begin{tabular}{cc}
  \toprule
  \textbf{Gal\'{a}thach} & \textbf{English}\\
  \cmidrule(lr){1-1}\cmidrule(lr){2-2}
  in ghen\'{e}th\'{e} chwimp & the pretty girls\\
  in wn\'{a} dech & the beautiful women\\
  in gw\'{\i}r\'{e} galv & the fat men\\
  \bottomrule
\end{tabular}
\label{examples_no_mutation_for_plural}
\end{table}

\subsection{Gwepchoprith \textendash\ The Verb \textit{to be}}
\subsubsection{The Verb \textit{to be}}
The verb \textit{to be} is the only irregular verb in modern Gaulish.\\
The verbal root (or infinitive) is \textit{bis}.\\
gw\'{e}la mi bis l\'{a}en = I want to be happy
The present tense form is \textit{esi}. It does not take an \textit{-a}.\\
Esi mi l\'{a}en = I am happy

\newpage
\subsection{Exercises}

\subsubsection{Construct sentences}

Construct the following sentences.
\begin{table}[H]
\centering
\begin{tabular}{|l|M{10.0cm}|}
  \toprule
  \textbf{Phrase (English)} & \textbf{Answer (Gal\'{a}thach)}\\
  \toprule
  the horse is big & \\
  \midrule
  the dog is fat & \\
  \midrule
  the woman is young & \\
  \midrule
  the man is old & \\
  \midrule
  the girl is small & \\
  \midrule
  the stone is short & \\
  \midrule
  the house is long & \\
  \midrule
  the judgement is bad & \\
  \midrule
  the owl is low & \\
  \midrule
  the sand is white & \\
  \midrule
  the mare is friendly & \\
  \midrule
  the mouth is big & \\
  \midrule
  the wolfish man & \\
  \midrule
  the equine woman & \\
  \midrule
  the young girl & \\
  \midrule
  the fat boy & \\
  \midrule
  the high eagle & \\
  \midrule
  the low stone & \\
  \midrule
  the black rock & \\
  \midrule
  the white heather & \\
  \midrule
  the ugly mouth & \\
  \midrule
  the long leg & \\
  \midrule
  the strong horse & \\
  \midrule
  the bad love & \\
  \midrule
  the young girls see the white horses & \\
  \midrule
  the old men call the fat dogs & \\
  \midrule
  the friendly women want the little boys & \\
  \midrule
  the big dogs love the black stones & \\
  \midrule
  the little girls hold the owlish dogs & \\
  \midrule
  the bad boys break the long stones & \\
  \bottomrule
\end{tabular}
\label{exercise_adjectives}
\caption{Exercise: adjectives}
\end{table}

\newpage
\subsubsection{Solution}
\begin{table}[H]
\centering
\rotatebox{180}{%
  \begin{tabular}{|l|>{\itshape}c|}
    \toprule
    \textbf{Phrase (English)} & \textbf{Answer (Gal\'{a}thach)}\\
    \toprule
    the horse is big & esi in \'{e}p m\'{a}r\\
    \midrule
    the dog is fat & esi in cun galv\\
    \midrule
    the woman is young & esi in ven ch'i\'{o}inch\\
    \midrule
    the man is old & esi in gwir sen\\
    \midrule
    the girl is small & esi in gheneth w\'{e}i\\
    \midrule
    the stone is short & esi in achaun bir\\
    \midrule
    the house is long & esi in t\'{e}i s\'{\i}r\\
    \midrule
    the judgement is bad & esi in varn dhruch\\
    \midrule
    the owl is low & esi in caun \'{\i}th\\
    \midrule
    the sand is white & esi in grau gwin\\
    \midrule
    the mare is friendly & esi in h\'{e}pis garanach\\
    \midrule
    the mouth is big & esi in boch m\'{a}r\\
    \midrule
    the wolfish man & in gwir bledhin\\
    \midrule
    the equine woman & in ven h\'{e}pin\\
    \midrule
    the young girl & in gheneth ch'i\'{o}inch\\
    \midrule
    the fat boy & in mapath galv\\
    \midrule
    the high eagle & in ernu ardhu\\
    \midrule
    the low stone & in achaun \'{\i}th\\
    \midrule
    the black rock & in garch dhuv\\
    \midrule
    the white heather & in bruch gwin\\
    \midrule
    the ugly mouth & in boch dw\'{a}iedh\\
    \midrule
    the long leg & in coch s\'{\i}r\\
    \midrule
    the strong horse & in \'{e}p nerthach\\
    \midrule
    the bad love & in garath druch\\
    \midrule
    the young girls see the white horses & ap\'{\i}sa in ghen\'{e}th\'{e} ch'i\'{o}inch in \'{e}p\'{e} gwin\\
    \midrule
    the old men call the fat dogs & g\'{a}ra in gw\'{\i}r\'{e} sen in c\'{u}n\'{e} galv\\
    \midrule
    the friendly women want the little boys & gw\'{e}la in wn\'{a} garanach in map\'{a}th\'{e} m\'{e}i\\
    \midrule
    the big dogs love the black rocks & c\'{a}ra in c\'{u}n\'{e} m\'{a}r in garch\'{e} dhuv\\
    \midrule
    the little girls hold the owlish dogs & delgha in ghen\'{e}th\'{e} w\'{e}i in c\'{u}n\'{e} caunin\\
    \midrule
    the bad boys break the long stones & br\'{\i}sa in map\'{a}the druch in achaun\'{e} s\'{\i}r\\
    \bottomrule
  \end{tabular}
}
\label{solution_adjectives}
\caption{Solution: adjectives}
\end{table}
