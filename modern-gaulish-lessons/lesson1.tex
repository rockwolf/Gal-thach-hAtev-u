\section{Menghavan 1: Br\'{e}thr\'{e} - Aman Dhathach:  Gweran\'{u}\'{e} Donach}
(\textit{Lesson 1: Verbs - Present Tense: Personal Pronouns})\\

\noindent In the first lesson you will learn how to put a verb in the present tense and how to use it with a personal pronoun.

\subsection{Verbs in the present tense}

\subsubsection{Root form}

\noindent Each verb has a basic form or \textit{root form}, known as a \textit{verbal noun}. It has the same function as the infinitve in English.

\noindent Verbal nouns can end in a consonant, in -i, -a, -e, and in just one case in -\'{o}, never in -u.\\\\
\noindent Examples are below:\\
\begin{table}[H]
\begin{tabu}{ll}
\'{a}pis: & to see\\
men: & to think\\
gwel: & to want\\
gar: & to call\\
carni: & to build\\
argha: & to shine\\
delghe: & to hold\\
\'{a}v\'{o}: & to do, to make\\
berwi: & to boil\\
gn\'{i}: & to know\\
\end{tabu}
\label{examples_verbal_nouns}
\end{table}

\subsubsection{Present tense}

\noindent To form the present tense of these verbs an -a is added to the verbal noun in the following ways. Note that vowels in modern Gaulish can be either long or short. Vowel length changes with emphasis. The emphasis is always on the second last syllable. When words are extended the emphasis shifts accordingly.\\

\noindent \textbf{Verbs on a consonant:}\\
%\noindent ápis &gt; apísa
%\noindent men &gt; ména
%\noindent gwel &gt; gwéla
%\noindent gar &gt; gára
\noindent \textbf{Verbs on -i:}\\
%\noindent carni &gt; carna
\noindent \textbf{Verbs on -a:}\\
%\noindent argha &gt; argha &#8211; nothing changes
\noindent \textbf{Verbs on -e:}\\
%\noindent Verbs on -e:
%\noindent delghe &gt; delgha
\noindent \textbf{Verbs on -\'{o} exchange the \'{o} for an -a:}\\
%\noindent ávó &gt; áva
%\noindent Verbs on -wi retain the final -i:
%\noindent berwi &gt; berwía
%\noindent Verbs on -i where the -i is the only vowel retain the -i:
%\noindent gní &gt; gnía
%\noindent <strong>Exercises</strong>
%\noindent Put the following verbs into the present tense:
%\noindent prin (to buy)
%\noindent ber (to carry)
%\noindent gal (to be able to do)
%\noindent brís (to break)
%\noindent ívi (to drink)
%\noindent cára (to love)
%\noindent cinge (to wage war)
%\noindent ávó (to do, to make)
%\noindent camwi (to bend, to curve)
%\noindent lí (to lie down)
%<ol start="2">
\subsection{Personal pronouns}
%</ol>
%\noindent The personal pronouns when used as subject are as follows:
%\noindent mi: I
%\noindent ti: you
%\noindent é: he
%\noindent í: she
%\noindent í: it
%\noindent ni: we
%\noindent sú: you pl.
%\noindent sí: they
%\noindent There is no difference in the third pronoun plural between the masculine and the feminine form.
%\noindent In modern Gaulish the personal pronoun follows the verb it accompanies:
%\noindent apísa mi: I see
%\noindent ména ti: you think
%\noindent gwéla é: he wants
%\noindent gára í: she calls
%\noindent carna ni: we build
%\noindent argha sú: you shine
%\noindent delgha sí: they hold
%\noindent áva í: it  does- it makes
%\noindent berwía í: it boils
%\noindent gnía í: it knows
%\noindent <strong>Exercises</strong>
%\noindent Make the following phrases:
%\noindent I buy
%\noindent you carry
%\noindent he can
%\noindent she breaks
%\noindent we drink
%\noindent you (pl.) love
%\noindent they wage war
%\noindent it does, it makes
%\noindent it bends
%\noindent it lies down

