\section{Menghavan 0: Swaus\'{e} In Tengu}
(\textit{Lesson 0: The Sounds Of The Language})\\

\noindent In this preliminary lesson you will learn what the sounds of the language are.

\subsection{Vowels}
\noindent Modern Gaulish has five vowels. They can be short or long. These are the short ones. The table shows how they are written, what their phonetic value is, and what they sound like using examples in English and other languages.

\begin{table}[H]
\begin{center}
\begin{tabular}{ccc}
  \toprule
  \textbf{vowel} & \textbf{phonetic value (IPA)} & \textbf{sound examples}\\
  \cmidrule(lr){1-1}\cmidrule(lr){2-2}\cmidrule(lr){3-3}
  a & [a] & p\textbf{at}\\
  o & [o] & p\textbf{o}t\\
  u & [u] & p\textbf{u}t\\
  e & [e] & p\textbf{e}t\\
  i & [i] & p\textbf{i}t\\
  \bottomrule
\end{tabular}
\end{center}
\caption{Vowels}
\label{phonology_vowels}
\end{table}

This table shows the long vowels. They are indicated with diacritics over the vowel, e.g. \'{a} is long a.

\begin{table}[H]
\begin{center}
\begin{tabular}{ccc}
  \toprule
  \textbf{vowel} & \textbf{phonetic value} & \textbf{sound examples}\\
  \cmidrule(lr){1-1}\cmidrule(lr){2-2}\cmidrule(lr){3-3}
  \'{a} & [a:] & p\textbf{a}rt\\
  \'{o} & [o:] & p\textbf{o}le\\
  \'{u} & [u:] & p\textbf{oo}l\\
  \'{e} & [e:] & p\textbf{a}y without the final y\\
  \'{\i} & [i:] & p\textbf{ee}l\\
  \bottomrule
\end{tabular}
\end{center}
\caption{Long vowels}
\label{phonology_long_vowels}
\end{table}

Modern Gaulish has five diphthongs. A diphthong is a group of two vowels written and pronounced together. This table shows them.

\begin{table}[H]
\begin{center}
\begin{tabular}{ccc}
  \toprule
  \textbf{diphthong} & \textbf{phonetic value (IPA)} & \textbf{sound examples}\\
  \cmidrule(lr){1-1}\cmidrule(lr){2-2}\cmidrule(lr){3-3}
  \'{a}i & [a:j] & b\textbf{ye}\\
  \'{o}i & [o:j] & b\textbf{oy}\\
  \'{u}i & [u:j] & br\textbf{oui}llard (French)\\
  \'{e}i & [e:j] & b\textbf{ay}\\
  au & [au] & c\textbf{ow}\\
  \bottomrule
\end{tabular}
\end{center}
\caption{Diphthongs}
\label{phonology_diphthongs}
\end{table}

\subsection{Consonants}

Modern Gaulish has a large number of consonants. The table below shows how they are written, gives their phonetic description, and gives sound examples in English and other languages. It is not possible to provide examples for every sound.

\begin{table}[H]
\begin{center}
\begin{tabular}{ccc}
  \toprule
  \textbf{consonant} & \textbf{phonetic value (IPA)} & \textbf{sound examples}\\
  \cmidrule(lr){1-1}\cmidrule(lr){2-2}\cmidrule(lr){3-3}
  p & [p] & \textbf{p}it\\
  t & [t] & \textbf{t}it\\
  c & [k] & \textbf{k}it\\
  b & [b] & \textbf{b}oar\\
  d & [d] & \textbf{d}oor\\
  g & [g] & \textbf{g}ore\\
  v & [v] & \textbf{v}ery\\
  dh & [\dh] & \textbf{th}ere\\
  gh & [$\upgamma$] & * $\acute{\varepsilon}\gamma\acute{\omega}$, ego, modern Greek ``I''\\
  f & [f] & \textbf{f}in\\
  th & [$\uptheta$] & \textbf{th}in\\
  ch & [x] & * lo\textbf{ch}, Scottish; i\textbf{ch}, German\\
  fh & [$\upphi$] & * f, with no tongue on teeth\\
  m & [m] & \textbf{m}ay\\
  w & [w] & \textbf{w}ay\\
  s & [s] & \textbf{s}it\\
  sh & [$\int$] & \textbf{sh}it\\
  n & [n] & \textbf{n}ose\\
  r & [r] & \textbf{r}ose\\
  l & [l] & \textbf{l}ose\\
  nh & [xn] & * [x] followed by [n]\\
  rh & [xr] & * [x] followed by [r]\\
  lh & [xl] & * [x] followed by [l]\\
  ng & [\ng] & si\textbf{ng}\\
  \bottomrule
\end{tabular}
\end{center}
\caption{Consonants}
\label{phonology_consonants}
\end{table}

\subsection{Vowel length variation}

The length of a vowel can change. In a word of two syllables or more the emphasis will be on the second last syllable. Often this will make the vowel of that syllable long. Examples are given below.\\

men: to think $\rightarrow$ vowel /e/ is short\\
m\'{e}nu: thought $\rightarrow$ emphasis on first vowel /e/ which becomes long\\
men\'{u}\'{e}: thoughts $\rightarrow$ emphasis shifts to second last vowel /u/ which becomes long\\
