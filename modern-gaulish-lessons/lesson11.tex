\section{Menghavan 11: Colav\'{a}ru \textemdash\ M\'{e}su P\'{e}thach \textendash\ Inchoran Chwoghn\'{\i}thach}
(\textit{Lesson 11: Conversation \textemdash\ Interrogative Mode \textendash\ Subordinate Clause})\\

In the eleventh lesson you will learn how to construct a question, and how to construct sentences within sentences.

\subsection{Gwepchoprith: Conversation}

\subsubsection{Conversation}

Below is a conversation between two people. Bren is a man, Chiom\'{a}ra is a woman. Both are traditional Gaulish names. Bren was the leader of the attack on Delphi in 279 BCE, Chiom\'{a}ra was a woman from Galatia. This conversation shows you how questions are constructed.
%<img data-attachment-id="166" data-permalink="https://moderngaulishlessons.wordpress.com/modern-gaulish-lessons-in-english/bren-and-chiomara/" data-orig-file="https://moderngaulishlessons.files.wordpress.com/2016/10/bren-and-chiomara.jpg?w=900" data-orig-size="1038,717" data-comments-opened="1" data-image-meta="{&quot;aperture&quot;:&quot;0&quot;,&quot;credit&quot;:&quot;&quot;,&quot;camera&quot;:&quot;&quot;,&quot;caption&quot;:&quot;&quot;,&quot;created_timestamp&quot;:&quot;0&quot;,&quot;copyright&quot;:&quot;&quot;,&quot;focal_length&quot;:&quot;0&quot;,&quot;iso&quot;:&quot;0&quot;,&quot;shutter_speed&quot;:&quot;0&quot;,&quot;title&quot;:&quot;&quot;,&quot;orientation&quot;:&quot;0&quot;}" data-image-title="Bren and Chiomara" data-image-description="" data-medium-file="https://moderngaulishlessons.files.wordpress.com/2016/10/bren-and-chiomara.jpg?w=900?w=300" data-large-file="https://moderngaulishlessons.files.wordpress.com/2016/10/bren-and-chiomara.jpg?w=900?w=900" class="alignnone size-full wp-image-166" src="https://moderngaulishlessons.files.wordpress.com/2016/10/bren-and-chiomara.jpg?w=900" alt="Bren and Chiomara" srcset="https://moderngaulishlessons.files.wordpress.com/2016/10/bren-and-chiomara.jpg?w=900 900w, https://moderngaulishlessons.files.wordpress.com/2016/10/bren-and-chiomara.jpg?w=150 150w, https://moderngaulishlessons.files.wordpress.com/2016/10/bren-and-chiomara.jpg?w=300 300w, https://moderngaulishlessons.files.wordpress.com/2016/10/bren-and-chiomara.jpg?w=768 768w, https://moderngaulishlessons.files.wordpress.com/2016/10/bren-and-chiomara.jpg?w=1024 1024w, https://moderngaulishlessons.files.wordpress.com/2016/10/bren-and-chiomara.jpg 1038w" sizes="(max-width: 900px) 100vw, 900px"   />

% TODO: add images
%\tcbox[colback=red!10!green!10, colframe=green!20!black!80]
\begin{table}[H]
\centering
    \begin{tabular}{M{3.0cm}M{10.0cm}M{3.0cm}}
    \cellcolor{lightgreen} & \cellcolor{lightgreen} & \cellcolor{lightgreen}\\
    \cellcolor{lightgreen}\textcolor{darkgreen}{\textbf{Bren}} & \cellcolor{lightgreen} & \cellcolor{lightgreen}\textcolor{darkgreen}{\textbf{Chiom\'{a}ra}}\\
    \cellcolor{lightgreen} & \cellcolor{lightgreen} & \cellcolor{lightgreen}\\
    \cellcolor{lightgreen} & \cellcolor{lightgreen} & \cellcolor{lightgreen}\\
    \cellcolor{lightgreen} & \cellcolor{lightgreen} & \cellcolor{lightgreen}\\
    \cellcolor{lightgreen} & \cellcolor{lightgreen} & \cellcolor{lightgreen}\\
    \cellcolor{lightgreen} & \cellcolor{lightgreen} & \cellcolor{lightgreen}\\
    \cellcolor{lightgreen} & \cellcolor{lightgreen} & \cellcolor{lightgreen}\\
%    \cellcolor{lightgreen}\multirow{-7}{*}{\includegraphics[height=4.0cm]{img/menghavan11_1}} & \cellcolor{lightgreen} & \cellcolor{lightgreen}\multirow{-7}{*}{\includegraphics[height=4.0cm]{img/menghavan11_2}}\\
    \cellcolor{lightgreen} & \cellcolor{lightgreen} & \cellcolor{lightgreen}
    \end{tabular}
\end{table}

\begingroup
\fontsize{10pt}{12pt}\selectfont
\begin{leftbubbles}Di wath. P\'{e} gaman a hesi ti?\end{leftbubbles}
\begin{rightbubbles}Esi mi in rh\'{e} dh\'{a}i, br\'{a}thu. Ach ti-s\'{u}\'{e}?\end{rightbubbles}
\begin{leftbubbles}Esi mi in rh\'{e} dh\'{a}i, br\'{a}thu. Ach ti-s\'{u}\'{e}?\end{leftbubbles}
\begin{rightbubbles}Esi mi in dh\'{a}i c\'{o}\'{e}th, br\'{a}thu. A ghn\'{\i}a ti o ti-esi pen r\'{e} dech?\end{rightbubbles}
\begin{leftbubbles}Gn\'{\i}a mi... p\'{e} a chw\'{e}la ti?\end{leftbubbles}
\begin{rightbubbles}A gh\'{a}la mi b\'{e} ich\'{\i} a brenuchi ich\'{\i} gwer m\'{o} sh\'{e}dhl'\'{e}p?\end{rightbubbles}
\begin{leftbubbles}Conechughri!\end{leftbubbles}
\endgroup

\newpage
\subsubsection{Colav\'{a}ru \textendash\ Tr\'{e}lav\'{a}ru}
(Conversation \textendash\ Translation)

Bren: Di wath. P\'{e} gaman a hesi ti?
(Bren: Good day. How are you?)

Chiom\'{a}ra: Esi mi in rh\'{e} dh\'{a}i, br\'{a}thu. Ach ti-s\'{u}\'{e}?
(Chiomara: I am very well, thank you. And yourself?)

Bren: Esi mi in dh\'{a}i c\'{o}\'{e}th, br\'{a}thu. A ghn\'{\i}a ti o ti-esi pen r\'{e} dech?
(Bren: I am well too, thank you. Do you know that you have a beautiful head?)

Chiom\'{a}ra: Gn\'{\i}a mi... p\'{e} a chw\'{e}la ti?
(Chiomara: I know... what do you want?)

Bren: A gh\'{a}la mi b\'{e} ich\'{\i} a brenuchi ich\'{\i} gwer m\'{o} sh\'{e}dhl'\'{e}p?
(Bren: Can I cut if [off] to hang it on my saddle?)

Chiom\'{a}ra: Conechughri!
(Chiom\'{a}ra: Fuck off!)

\subsubsection{Vocabulary}

p\'{e}: what (causes ICM on following word)
caman: road, way > p\'{e} gaman: how

d\'{\i}: day
good
> d\'{\i} wath: good day (ICM because \textit{d\'{\i}} is a feminine word)

d\'{a}i: good
r\'{e}: very (causes ICM on the following word)
in: adverbial marker (causes ICM on the following word)
> in rh\'{e} dh\'{a}i: very well

br\'{a}thu: thank you (thanks)

ti: you
s\'{u}\'{e}: self
> ti-s\'{u}\'{e}: yourself

ach: and

c\'{o}\'{e}th: also, too, as well

gn\'{\i}: to know
gn\'{\i}a ti: you know
a: question particle (causes ICM on the following word)
> a ghn\'{\i}a ti: do you know

o: relative pronoun for subordinate clauses $\rightarrow$ \textit{that, which}; see further below

ti-esi: you have $\rightarrow$ see below

pen: head
tech: beautiful
> pen r\'{e} dech: a very beautiful head

gw\'{e}la ti: you want
p\'{e} a chw\'{e}la ti: what do you want

g\'{a}la mi: I can
a gh\'{a}la mi: can I?

b\'{e}: to cut

prenuchi: to hang

gwer: on

s\'{e}dhl: seat
\'{e}p: horse
> s\'{e}dhl'\'{e}p: saddle (horse-seat)

conechughri: to fuck off (idiomatic expression)

\subsection{Gwepchoprith: Question formation}

\subsubsection{Question formation}

As you can see in the vocabulary, questions are made by putting the question particle \textit{a} in front of the verb in the sentence.

gn\'{\i}a ti: you know > statement.
a ghn\'{\i}a ti?: do you know? > question.

The word order doesn't change. The only difference is the question particle \textit{a}. It causes an initial consonant mutation (ICM) on the following word.

g\'{a}la mi: I can > statement.
a gh\'{a}la mi?: can I? > question.

\subsubsection{Question words}

p\'{e}: what, which
p\'{e} gaman: how
p\'{\i}: who
p\'{e}ri: why
p\'{e}m\'{a}i: where
ponch: when

\subsection{Gwepchoprith: The verb to have}
\subsubsection{The verb to have}

Gal\'{a}thach does not have a specific verb \textit{to have}. Instead it uses a compound construction using the verb \textit{to be}. It is made of the personal pronoun of the entity that \textit{has} something, followed by the verb \textit{to be}. The two are connected with a hyphen.

mi-esi \'{e}p: I have a horse
> mi-esi translates as \textit{with-me, to-me}
> mi-esi \'{e}p = \textit{to-me} is a horse

The conjugation is as follows:

mi-esi: I have
ti-esi: you have
\'{e}-esi: he has
\'{\i}-esi: she has
ni-esi: we have
s\'{u}-esi: you (pl.)
s\'{\i}-esi: they have

If the subject of a phrase is specified it comes first, followed by the pronoun+to-be construction.

in gwir: the man
\'{e}-esi: he has
cun: dog
> in gwir \'{e}-esi cun: the man has a dog

in d\'{o}n\'{e}: the people
s\'{\i}-esi: they have
gavr\'{e}: goats
> in d\'{o}n\'{e} s\'{\i}-esi gavr\'{e}: the people have goats

In a question the question particle \textit{a} comes before the pronoun+to-be construction. It causes mutation to the first letter of the construction.

p\'{e} a di-esi: what do you have?
p\'{e}ri a sh\'{u}-esi \'{e}p\'{e}: why do you (pl.) have horses?

In a question where the subject is specified the question particle comes between the subject and the pronoun+to-be construction.

p\'{e}ri in gwir a h\'{e}-esi cun: why does the man have a dog?
p\'{e}ri in d\'{o}n\'{e} a sh\'{\i}-esi gavr\'{e}: why do the people have goats?

\subsection{Exercises 1}

\subsubsection{Vocabulary}

anu: name
bithi: to live
esi: to be
caran: friend
insin: here
pen'vl\'{e}dhn: birthday (pen + bl\'{e}dhn \textit{head-year})
cun: dog
\'{e}p: horse
b\'{o}: cow
curu: beer
gw\'{\i}n: wine
n\'{e}veth: nothing
pen: head
m\'{a}r: big
s\'{\i}r: long
trughn: nose
galv: fat
trai\'{e}th\'{e}: feet (of more than one person)
luthrach: dirty
l\'{a}m\'{e}: hands (of more than one person)
m\'{e}i: small
gwolth: hair
carnu: horn
bir: short

\subsubsection{Construct phrases}

Construct the following phrases.

What's your name?
Where do you live?
How are you?
Who is your friend?
Why are you here?
When is your birthday?

I have a name:
you have a dog:
he has a horse:
she has a cow:
we have beer:
you (pl.) have wine:
they have nothing:

the horse has long ears:
the man has a fat nose:
the children have dirty feet:

Why do you have a big head?
Why do they have small hands?
Why do the women have long hair?
Why do the cows have short horns?

\newpage
\subsubsection{Solution}

%What�s your name: P\'{e} a hesi t\'{o} hanu?
%Where do you live: P\'{e}m\'{a}i a v\'{\i}tha ti?
%How are you: P\'{e} gaman a hesi ti?
%Who is your friend: P\'{\i} a hesi t\'{o} garan?
%Why are you here: P\'{e}ri a hesi ti insin?
%When is your birthday: Ponch a hesi t\'{o} ben�vl\'{e}dhn?
%
%I have a name: mi-esi anu
%you have a dog: ti-esi cun
%he has a horse: \'{e}-esi \'{e}p
%she has a cow: \'{\i}-esi b\'{o}
%we have beer: ni-esi curu
%you (pl.) have wine: s\'{u}-esi gw\'{\i}n
%they have nothing: s\'{\i}-esi n\'{e}veth
%
%the horse has long ears: in \'{e}p \'{e}-esi daus s\'{\i}r
%the man has a fat nose: in gwir \'{e}-esi trughn galv
%the children have dirty feet: in gn\'{a}th\'{e} s\'{\i}-esi tr\'{a}i\'{e}th\'{e} luthrach
%
%Why do you have a big head: P\'{e}ri a di-esi pen m\'{a}r?
%Why do they have small hands: p\'{e}ri a sh\'{\i}-esi l\'{a}m\'{e} w\'{e}i?
%Why do the women have long hair: P\'{e}ri in wn\'{a} a sh\'{\i}-esi gwolth s\'{\i}r?
%Why do the cows have short horns: P\'{e}ri in b\'{o}\'{e} a sh\'{\i}-esi carn\'{u}\'{e} bir?

\subsection{Gwepchoprith: Subordinate clauses}

\subsubsection{Subordinate clauses}

A subordinate clause is a sentence inside of another sentence. It has a verb, subject and object that are independent of the main sentence. The two are linked with the particle \textit{o}, which translates in English as \textit{that} or \textit{which}.

gn\'{\i}a ti: you know
o: that
ti-esi: you have
pen: a head
tech: beautiful
> gn\'{\i}a ti \textbf{o} ti-esi pen tech: you know \textbf{that} you have a beautiful head

ap\'{\i}sa mi: I see
o: that
n\'{e} hesi: there is not
curu: beer
\'{e}th: more
> ap\'{\i}sa mi \textbf{o} n\'{e} hesi curu \'{e}th: I see \textbf{that} there is no more beer

In the subordinate clause the second sentence (\textit{there is no more beer}) can stand alone independently from the first one (\textit{I see}). The two are connected by the particle \textit{o}.

If the particle \textit{o} is followed by a word that starts with a vowel it becomes \textit{och}.

gw\'{\i}dha mi: I understand
o: that
esi ti: you are
lisc: tired
> gw\'{\i}dha mi \textbf{och} esi ti lisc: I understand \textbf{that} you are tired

\subsection{Excercises 2}

\subsubsection{Vocabulary}

gwesc\'{a}ra: to deserve
b\'{o}i: to hit
mesc: drunk
cl\'{u}i: to hear
duchan: to howl
gw\'{o}men: to hope
men: to think
suchw\'{\i}s: clever
duchw\'{\i}s: stupid

\subsubsection{Construct phrases}

Construct the following phrases.

you deserve that I hit you:
the man sees that the horse is drunk:
the children hear that the dogs howl:
the women know that their hair is long:
the goats hope that they have horns:
he thinks that he is clever:
she knows that he is stupid:
the people don't know that they have fat noses:

\newpage
\subsubsection{Solution}

%<strong>Answer 2</strong>
%
%you deserve that I hit you: gwesc\'{a}ra ti o b\'{o}ia mi ti
%the man sees that the horse is drunk: ap\'{\i}sa in gwir och esi in \'{e}p mesc
%the children hear that the dogs howl: cl\'{u}ia in gn\'{a}th\'{e} o duch\'{a}na in c\'{u}n\'{e}
%the women know that their hair is long: gn\'{\i}a in wn\'{a} och esi s\'{o} chwolth s\'{\i}r
%the goats hope that they have horns: gw\'{o}m\'{e}na in gavr\'{e} o s\'{\i}-esi carn\'{u}\'{e}
%he thinks that he is clever: m\'{e}na \'{e} och esi \'{e} suchw\'{\i}s
%she knows that he is stupid: gn\'{\i}a \'{\i} och esi \'{e} duchw\'{\i}s
%the people don�t know that they have fat noses: n\'{e} ghn\'{\i}a in d\'{o}n\'{e} o s\'{\i}-esi trughn\'{e} galv

