\section{Menghavan 3: An\'{u}\'{e} \textendash\ T\'{e}ith \textendash\ In hAmosanal}
(\textit{Lesson 3: Nouns \textendash\ Possession \textendash\ The Article})

In the third lesson you will learn what a noun is, how it is possessed, and what the article is.\\

\subsection{Nouns}

The word ``noun'' means \textit{name}. It is a word that refers to anything that can have a name, such as a person, place, thing, state or quality. In lesson 1 and 2 we learned about subjects and objects. Nouns are things that can be subjects or objects of a sentence.\\

Examples:
\begin{table}[H]
\centering
\begin{tabu}{c|c}
  \textbf{Gal\'{a}thach} & \textbf{English}\\
  \toprule
  gwir & man\\
  cun & dog\\
  \'{e}p & horse\\
  c\'{a}nu & song\\
  m\'{e}nu & thought\\
  coch & leg\\
  duvr & water\\
  pen & head
\end{tabu}
\label{examples_nouns}
\end{table}

There is no indefinite article like English ``a, an'' in modern Gaulish:
\begin{table}[H]
\centering
\begin{tabu}{c|c}
  \textbf{Gal\'{a}thach} & \textbf{English}\\
  \toprule
  gwir & man\\
  gwir & a man\\
  cun & dog\\
  cun & a dog
\end{tabu}
\label{examples_no_indefinite_article}
\end{table}

We can use the verbs we learned in the previous lessons to construct sentences where the subject and the object are nouns instead of pronouns:
\begin{table}[H]
\begin{tabu}{cccc}
  g\'{a}ra & gwir & cun & (a man calls a dog)\\
  V & S & O &
\end{tabu}
\label{examples_vso}
\end{table}

Examples:
\begin{table}[H]
\centering
\begin{tabu}{c|c}
  \textbf{Gal\'{a}thach} & \textbf{English}\\
  \toprule
  ap\'{\i}sa cun \'{e}p & a dog sees a horse\\
  c\'{a}na gwir c\'{a}nu & a man sings a song\\
  m\'{e}na gwir m\'{e}nu & a man thinks a thought
\end{tabu}
\label{examples_vso_more_examples}
\end{table}

\newpage
\subsubsection{Exercises}

Construct the following phrases with the verbs given in the previous lessons and the following nouns given:

\begin{quote}
ben (woman), gnath (child), mapath (boy), geneth (girl), curu (beer), cuchul (hat)
\end{quote}

\begin{table}[H]
\centering
\begin{tabu}{|l|M{10.0cm}|}
  \toprule
  \textbf{Phrase (English)} & \textbf{Answer (Gal\'{a}thach)}\\
  \toprule
  a man buys a beer & \\
  \midrule
  a woman holds a child & \\
  \midrule
  a boy wants a hat & \\
  \midrule
  a girl sings a song & \\
  \midrule
  a horse drinks water & \\
  \midrule
  a dog breaks a leg & \\
  \midrule
  a child loves a horse & \\
  \midrule
  a man sees a woman & \\
  \midrule
  a horse carries a boy & \\
  \midrule
  a woman calls a dog & \\
  \bottomrule
\end{tabu}
\label{exercise_no_indefinite_article}
\caption{Exercise: no indefinite article}
\end{table}

\newpage
Solution:\\
\begin{table}[H]
\centering
\rotatebox{180}{%
  \begin{tabu}{|l|>{\itshape}c|}
    \toprule
    \textbf{Phrase (English)} & \textbf{Answer (Gal\'{a}thach)}\\
    \toprule
    a man buys a beer & pr\'{\i}na gwir curu\\
    \midrule
    a woman holds a child & delgha ben gnath\\
    \midrule
    a boy wants a hat & gw\'{e}la mapath cuchul\\
    \midrule
    a girl sings a song & c\'{a}na geneth c\'{a}nu\\
    \midrule
    a horse drinks water & \'{\i}va \'{e}p duvr\\
    \midrule
    a dog breaks a leg & br\'{\i}sa cun coch\\
    \midrule
    a child loves a horse & c\'{a}ra gnath \'{e}p\\
    \midrule
    a man sees a woman & ap\'{\i}sa gwir ben\\
    \midrule
    a horse carries a boy & b\'{e}ra \'{e}p mapath\\
    \midrule
    a woman calls a dog & g\'{a}ra ben cun\\
    \bottomrule
  \end{tabu}
}
\label{solution_no_indefinite_article}
\caption{Solution: no indefinite article}
\end{table}
\newpage

\subsection{Possession}
In lesson 2, we saw that when a pronoun was used as an object, it had a special possession particle i-. This particle is not used with anything else, only with the pronoun. When we use a noun we just replace the pronoun and the particle with a noun:
\begin{table}[H]
\centering
\begin{tabu}{c|c}
  \textbf{Gal\'{a}thach} & \textbf{English}\\
  \toprule
  c\'{a}na mi c\'{a}nu & I sing a song\\
  gw\'{e}la mi can ich\'{\i} & I want to sing it\\
  gw\'{e}la mi can c\'{a}nu & I want to sing a song
\end{tabu}
\label{examples_possession_particle_when_using_noun}
\end{table}

The phrase \textit{can ``c\'{a}nu''} means \textit{$[$the$]$ singing of a song}. The English word $[$the$]$ is not used.\\

This phrase has two nouns:
\begin{enumerate}
 \item{the verbal noun \textit{can}}
 \item{the noun \textit{c\'{a}nu}}
\end{enumerate}
In this phrase, the first noun \textit{can} is possessed by the second noun \textit{c\'{a}nu}. In English this is indicated by the word \textit{of}. In modern Gaulish this is indicated by the position of the word: the \textit{second} word \textit{possesses} the \textit{first} word.

The same can be done with any two nouns:
\begin{table}[H]
\centering
\begin{tabu}{c|c}
  \textbf{Gal\'{a}thach} & \textbf{English}\\
  \toprule
  curu gwir & a beer of a man $[$a man's beer$]$\\
  gnath ben & a child of a woman $[$a woman's child$]$\\
  \'{e}p geneth & a horse of a girl $[$a girl's horse$]$
\end{tabu}
\label{examples_possession_word_position}
\end{table}

\newpage
\subsubsection{Exercises}

Using the words learned in all the lessons make the following phrases:
\begin{table}[H]
\centering
\begin{tabu}{|l|M{10.0cm}|}
  \toprule
  \textbf{Phrase (English)} & \textbf{Answer (Gal\'{a}thach)}\\
  \toprule
  a leg of a dog $[$a dog's leg$]$ & \\
  \midrule
  a dog of a man $[$a man's dog$]$ & \\
  \midrule
  a head of a horse $[$a horse's head$]$ & \\
  \midrule
  a hat of a woman $[$a woman's hat$]$ & \\
  \midrule
  a thought of a child $[$a child's thought$]$ & \\
  \midrule
  a song of a girl $[$a girl's song$]$ & \\
  \midrule
  a horse of a boy $[$a boy's horse$]$ & \\
  \midrule
  a man of a woman $[$a woman's man$]$ & \\
  \midrule
  a child of a man $[$a man's child$]$ & \\
  \midrule
  a hat of a child $[$a child's hat$]$ & \\
  \bottomrule
\end{tabu}
\label{exercise_possession}
\caption{Exercise: possession}
\end{table}

\newpage
Solution:
\begin{table}[H]
\centering
\rotatebox{180}{%
  \begin{tabu}{|l|>{\itshape}c|}
    \toprule
    \textbf{Phrase (English)} & \textbf{Answer (Gal\'{a}thach)}\\
    \toprule
    a leg of a dog $[$a dog's leg$]$ & coch cun\\
    \midrule
    a dog of a man $[$a man's dog$]$ & cun gwir\\
    \midrule
    a head of a horse $[$a horse's head$]$ & pen \'{e}p\\
    \midrule
    a hat of a woman $[$a woman's hat$]$ & cuchul ben\\
    \midrule
    a thought of a child $[$a child's thought$]$ & m\'{e}nu gnath\\
    \midrule
    a song of a girl $[$a girl's song$]$ & c\'{a}nu geneth\\
    \midrule
    a horse of a boy $[$a boy's horse$]$ & \'{e}p mapath\\
    \midrule
    a man of a woman $[$a woman's man$]$ & gwir ben\\
    \midrule
    a child of a man $[$a man's child$]$ & gnath gwir\\
    \midrule
    a hat of a child $[$a child's hat$]$ & cuchul gnath\\
    \bottomrule
  \end{tabu}
}
\label{solution_possession}
\caption{Solution: possession}
\end{table}
\newpage

\subsection{The Article}

Modern Gaulish has one article: ``in''. It does not change for any reason.\\
Examples:
\begin{table}[H]
\centering
\begin{tabu}{c|c}
  \textbf{Gal\'{a}thach} & \textbf{English}\\
  \toprule
  in gwir & the man\\
  in \'{e}p & the horse\\
  in mapath & the boy\\
  in curu & the beer\\
  in pen & the head\\
  in duvr & the water
\end{tabu}
\label{examples_possession_particle_when_using_noun}
\end{table}

The article \textit{in} can be used in cases of possession. It can only be used with the second noun, which is the one possessing the first noun. The first noun can never have the article in front of it.\\
Examples:
\begin{table}[H]
\centering
\begin{tabu}{c|c}
  \textbf{Gal\'{a}thach} & \textbf{English}\\
  \toprule
  cun gwir & a dog of a man $[$a man's dog$]$\\
  cun in gwir & a dog of the man $[$the man's dog$]$
\end{tabu}
\label{examples_possession_first_noun_no_particle_in_front}
\end{table}

The English phrase between brackets [\dots] shows a very good translation of the modern Gaulish phrase. It only uses one article and can only ever use one article. It is not possible to say ``the man's the dog''.\\

The second noun possesses the first noun. The second noun is the only noun that can have the article.

\newpage
\subsubsection{Exercises}

Construct the following phrases, using all the words learned so far:
\begin{table}[H]
\centering
\begin{tabu}{|l|M{10.0cm}|}
  \toprule
  \textbf{Phrase (English)} & \textbf{Answer (Gal\'{a}thach)}\\
  \toprule
  the head of the horse & \\
  \midrule
  the leg of the dog & \\
  \midrule
  the beer of the man & \\
  \midrule
  the hat of the boy & \\
  \midrule
  the water of the horse & \\
  \midrule
  the song of the boy & \\
  \midrule
  the thought of the man & \\
  \midrule
  the horse of the song & \\
  \midrule
  the dog of the boy & \\
  \midrule
  the hat of the horse & \\
  \bottomrule
\end{tabu}
\label{exercise_article_in}
\caption{Exercise: article in}
\end{table}

\newpage
Solution:
\begin{table}[H]
\centering
\rotatebox{180}{%
  \begin{tabu}{|l|>{\itshape}c|}
    \toprule
    \textbf{Phrase (English)} & \textbf{Answer (Gal\'{a}thach)}\\
    \toprule
    the head of the horse & pen in \'{e}p\\
    \midrule
    the leg of the dog & coch in cun\\
    \midrule
    the beer of the man & curu in gwir\\
    \midrule
    the hat of the boy & cuchul in mapath\\
    \midrule
    the water of the horse & duvr in \'{e}p\\
    \midrule
    the song of the boy & c\'{a}nu in mapath\\
    \midrule
    the thought of the man & m\'{e}nu in gwir\\
    \midrule
    the horse of the song & \'{e}p in c\'{a}nu\\
    \midrule
    the dog of the boy & cun in mapath\\
    \midrule
    the hat of the horse & cuchul in \'{e}p\\
    \bottomrule
  \end{tabu}
}
\label{solution_article_in}
\caption{Solution: article in}
\end{table}
