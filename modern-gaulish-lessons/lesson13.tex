\section{Menghavan 13: Inchor\'{a}n\'{e} Gwepsin \textendash\ Inchor\'{a}n\'{e} Gwep P\'{e}than}
(\textit{Lesson 13: Conjunction Clauses \textendash\ Question Word Clause})\\

In the thirteenth lesson you will learn what conjunction and question word clauses are, and how they are constructed in Gal\'{a}thach.

\subsection{Gwepchoprith: Conversation}
\subsubsection{Conversation}

Below is a conversation between two people. Marthal is a man and Suvron is a woman. Both are attested Gaulish names. The conversation shows you how conjunction and question word clauses are constructed.

%<img data-attachment-id="171" data-permalink="https://moderngaulishlessons.wordpress.com/modern-gaulish-lessons-in-english/collage/" data-orig-file="https://moderngaulishlessons.files.wordpress.com/2016/10/collage.jpg?w=900" data-orig-size="3000,4000" data-comments-opened="1" data-image-meta="{&quot;aperture&quot;:&quot;0&quot;,&quot;credit&quot;:&quot;&quot;,&quot;camera&quot;:&quot;&quot;,&quot;caption&quot;:&quot;&quot;,&quot;created_timestamp&quot;:&quot;0&quot;,&quot;copyright&quot;:&quot;&quot;,&quot;focal_length&quot;:&quot;0&quot;,&quot;iso&quot;:&quot;0&quot;,&quot;shutter_speed&quot;:&quot;0&quot;,&quot;title&quot;:&quot;&quot;,&quot;orientation&quot;:&quot;0&quot;}" data-image-title="collage" data-image-description="" data-medium-file="https://moderngaulishlessons.files.wordpress.com/2016/10/collage.jpg?w=900?w=225" data-large-file="https://moderngaulishlessons.files.wordpress.com/2016/10/collage.jpg?w=900?w=768" class="alignnone size-full wp-image-171" src="https://moderngaulishlessons.files.wordpress.com/2016/10/collage.jpg?w=900" alt="collage" srcset="https://moderngaulishlessons.files.wordpress.com/2016/10/collage.jpg?w=900 900w, https://moderngaulishlessons.files.wordpress.com/2016/10/collage.jpg?w=1800 1800w, https://moderngaulishlessons.files.wordpress.com/2016/10/collage.jpg?w=113 113w, https://moderngaulishlessons.files.wordpress.com/2016/10/collage.jpg?w=225 225w, https://moderngaulishlessons.files.wordpress.com/2016/10/collage.jpg?w=768 768w" sizes="(max-width: 900px) 100vw, 900px"   />

%\tcbox[colback=red!10!green!10, colframe=green!20!black!80]
\begin{table}[H]
\centering
    \begin{tabular}{M{3.0cm}M{10.0cm}M{3.0cm}}
    \cellcolor{lightgreen} & \cellcolor{lightgreen} & \cellcolor{lightgreen}\\
    \cellcolor{lightgreen}\textcolor{darkgreen}{\textbf{Suvron}} & \cellcolor{lightgreen} & \cellcolor{lightgreen}\textcolor{darkgreen}{\textbf{Marthal}}\\
    \cellcolor{lightgreen} & \cellcolor{lightgreen} & \cellcolor{lightgreen}\\
    \cellcolor{lightgreen} & \cellcolor{lightgreen} & \cellcolor{lightgreen}\\
    \cellcolor{lightgreen} & \cellcolor{lightgreen} & \cellcolor{lightgreen}\\
    \cellcolor{lightgreen} & \cellcolor{lightgreen} & \cellcolor{lightgreen}\\
    \cellcolor{lightgreen} & \cellcolor{lightgreen} & \cellcolor{lightgreen}\\
    \cellcolor{lightgreen} & \cellcolor{lightgreen} & \cellcolor{lightgreen}\\
%    \cellcolor{lightgreen}\multirow{-7}{*}{\includegraphics[height=4.0cm]{img/menghavan11_1}} & \cellcolor{lightgreen} & \cellcolor{lightgreen}\multirow{-7}{*}{\includegraphics[height=4.0cm]{img/menghavan11_2}}\\
    \cellcolor{lightgreen} & \cellcolor{lightgreen} & \cellcolor{lightgreen}
    \end{tabular}
\end{table}

\begingroup
\fontsize{10pt}{12pt}\selectfont
\begin{leftbubbles}Anghn\'{i}tha mi ma chw\'{e}la ti suling canim?\end{leftbubbles}
\begin{rightbubbles}N\'{e} ghn\'{i}a mi p\'{e} a chw\'{e}la mi \'{a}v\'{o}. A chw\'{e}la mi suling gw\'{e} n\'{e} a chw\'{e}la mi?\end{rightbubbles}
\begin{leftbubbles}Gw\'{e}la mi suling canith ach n\'{e} a chw\'{e}la ti?\end{leftbubbles}
\begin{rightbubbles}Gw\'{e}la ti suling canim \'{e}ithr n\'{e} hesi suling s\'{e} o gw\'{e}la mi av\'{o}.\end{rightbubbles}
\begin{leftbubbles}A ghn\'{i}a ti p\'{i} a hesi mi?\end{leftbubbles}
\begin{rightbubbles}N\'{e} ghn\'{i}a mi. C\'{o}\'{e}th, n\'{e} ghn\'{i}a mi p\'{e} shulingen a hesi s\'{e}.\end{rightbubbles}
\begin{leftbubbles}A ghn\'{i}a ti p\'{e} gaman a shuling?\end{leftbubbles}
\begin{rightbubbles}N\'{e} ghn\'{i}a mi. N\'{e} ghn\'{i}a mi diaman p\'{e}m\'{a}i a h\'{a}dha m\'{o} dh\'{a}lam, ach n\'{e} ghn\'{i}a mi diaman ponch a wantha m\'{o} dh\'{a}thr\'{a}ieth.\end{rightbubbles}
\begin{leftbubbles}A ghn\'{i}a ti p\'{e}ri a chw\'{e}la mi o sulinga ti canim?\end{leftbubbles}
\begin{rightbubbles}N\'{e} ghn\'{i}a mi. A chw\'{e}la ti gn\'{i} pethi b\'{a}n\'{e} u guru a h\'{i}va mi pap d\'{i}?\end{rightbubbles}
\begin{leftbubbles}Esi mi certh o n\'{e} chw\'{e}la mi gn\'{i}. Duch, p\'{e} a chw\'{e}la ti \'{a}v\'{o}?\end{leftbubbles}
\begin{rightbubbles}Gw\'{e}la mi \'{i}vi curu \'{e}th.\end{rightbubbles}
\endgroup

\newpage
\subsubsection{Colav\'{a}ru \textendash\ Tr\'{e}lav\'{a}ru}
(Conversation \textendash\ Translation)

Suvron: Anghn\'{i}tha mi ma chw\'{e}la ti suling canim?
(Suvron: I wonder if you want to dance with me?)

Marthal: N\'{e} ghn\'{i}a mi p\'{e} a chw\'{e}la mi \'{a}v\'{o}. A chw\'{e}la mi suling gw\'{e} n\'{e} a chw\'{e}la mi?
(Marthal: I don't know what I want to do. Do I want to dance or don't I?)

Suvron: Gw\'{e}la mi suling canith ach n\'{e} a chw\'{e}la ti?
(Suvron: I want to dance with you and you don't want to?)

Marthal: Gw\'{e}la ti suling canim \'{e}ithr n\'{e} hesi suling s\'{e} o gw\'{e}la mi av\'{o}.
(Marthal: You want to dance with me but dancing is not what I want to do.)

Suvron: A ghn\'{i}a ti p\'{i} a hesi mi?
(Suvron: Do you know who I am?)

Marthal: N\'{e} ghn\'{i}a mi. C\'{o}\'{e}th, n\'{e} ghn\'{i}a mi p\'{e} shulingen a hesi s\'{e}.
(Marthal: I don't know. Also, I don't know what dance that is.)

Suvron: A ghn\'{i}a ti p\'{e} gaman a shuling?
(Suvron: Do you know how to dance?)

Marthal: N\'{e} ghn\'{i}a mi. N\'{e} ghn\'{i}a mi diaman p\'{e}m\'{a}i a h\'{a}dha m\'{o} dh\'{a}lam, ach n\'{e} ghn\'{i}a mi diaman ponch a wantha m\'{o} dh\'{a}thr\'{a}ieth.
(Marthal: I don't. I never know where to put my hands, and I never know when to move my feet.)

Suvron: A ghn\'{i}a ti p\'{e}ri a chw\'{e}la mi o sulinga ti canim?
(Suvron: Do you know why I want you to dance with me?)

Marthal: N\'{e} ghn\'{i}a mi. A chw\'{e}la ti gn\'{i} pethi b\'{a}n\'{e} u guru a h\'{i}va mi pap d\'{i}?
(Marthal: I don't. Do you want to know how many glasses of beer I drink every day?)

Suvron: Esi mi certh o n\'{e} chw\'{e}la mi gn\'{i}. Duch, p\'{e} a chw\'{e}la ti \'{a}v\'{o}?
(Suvron: I'm sure that I don't want to know. So, what do you want to do?)

Marthal: Gw\'{e}la mi \'{i}vi curu \'{e}th.
(Marthal: I want to drink more beer.)

Meaning of names:
Marthal: Big Forehead ($\leftarrow$ Marotalus)
Suvron: Good Breast ($\leftarrow$ Subroni)

\subsubsection{Vocabulary}

\begin{table}[H]
\centering
\begin{tabular}{cc}
  \toprule
  \textbf{Gal\'{a}thach} & \textbf{English}\\
  \cmidrule(lr){1-1}\cmidrule(lr){2-2}
  anghn\'{\i}thi & to wonder\\
  gwel & to want\\
  suling & to dance\\
  gn\'{\i} & to know\\
  \'{a}v\'{o} & to do\\
  \'{e}ithr & but\\
  c\'{o}\'{e}th & also\\
  diaman & never\\
  ma & if\\
  p\'{e} & what\\
  gw\'{e} & or\\
  ach & and\\
  p\'{\i} & who\\
  p\'{e} shulingen & what (which) dance\\
  p\'{e} gaman & how (which way)\\
  p\'{e}m\'{a}i & where\\
  ponch & when\\
  \'{a}dha & to put\\
  d\'{a}lam & hands (of a person $\rightarrow$ a pair of hands)\\
  mantha & to move\\
  d\'{a}thr\'{a}ieth & feet (of a person $\rightarrow$ a pair of feet)\\
  p\'{e}ri & why\\
  pethi & how much/how many\\
  pan & cup\\
  u & of (quantity)\\
  curu & beer\\
  \'{\i}vi & to drink\\
  pap & every\\
  d\'{\i} & day\\
  duch & so, anyway, go on then, therefore\\
  \'{e}th & more\\
  \bottomrule
\end{tabular}
\label{vocab_conversation_lesson13}
\caption{Vocabulary conversation lesson 13}
\end{table}

\subsection{Gwepchoprith: Conjunction clauses \textendash\ question word clauses}

\subsubsection{Conjunction clauses}

A conjunction clause is a phrase that is linked to a phrase before it with a conjunction, combining into one sentence. Conjunctions are:

\begin{itemize}
  \item ma: if
  \item gw\'{e}: or
  \item ach: and
  \item \'{e}ithr: but
  \item s\'{e} o: what/which (``that which'')
\end{itemize}

Conjunction clauses found in the conversation above:
\begin{enumerate}
  \item Suvron: Anghn\'{\i}tha mi \textbf{ma} chw\'{e}la ti suling canim?
(Suvron: I wonder \textbf{if} you want to dance with me?)
  \item Marthal: A chw\'{e}la mi suling \textbf{gw\'{e}} n\'{e} a chw\'{e}la mi?
(Marthal: Do I want to dance \textbf{or} don't I?)
  \item Suvron: Gw\'{e}la mi suling canith \textbf{ach} n\'{e} a chw\'{e}la ti?
(Suvron: I want to dance with you \textbf{and} you don't want to?)
\item Marthal: Gw\'{e}la ti suling canim \textbf{\'{e}ithr} n\'{e} hesi suling \textbf{s\'{e} o} gw\'{e}la mi av\'{o}.
(Marthal: You want to dance with me \textbf{but} dancing is not \textbf{what} I want to do.)
\end{enumerate}

The conjunctions \textit{what} and \textit{which} are very important. They can be used in two ways: as a word that refers to something else and as a question.

The English words \textit{what} and \textit{which} that refer to something else are translated by the phrase \textit{s\'{e} o}, which literally means \textit{that which}. It is used in the last example above.

Marthal:n\'{e} hesi suling \textbf{s\'{e} o} gw\'{e}la mi av\'{o}.
(Marthal: dancing is not \textbf{what} I want to do.)

In the phrase textit{dancing is not what I want to do} the word \textit{what} is translated by \textit{s\'{e} o}.

In the English translation the word textit{what} can be replaced by the phrase \textit{that which} without changing the meaning of the sentence. If that can be done you know that the word \textit{what} is used as a conjunction, and not as a question word. If that is the case the word \textit{what} is translated in Gal\'{a}thach as \textit{s\'{e} o}.

If the word \textit{what} in the English translation cannot be translated by the phrase \textit{that which}, and a question is implied in the phrase, it needs to be translated by a question word.

\subsubsection{Question word clauses}

A question word clause is a phrase that is linked to a phrase before it with a question word, combining into one sentence. Usually there is an aspect of something unknown implied in the phrase, suggesting a question. Question words are:

\begin{itemize}
  \item p\'{\i}: who
  \item p\'{e}: what
  \item p\'{e} gaman: how
  \item p\'{e}m\'{a}i: where
  \item ponch: when
  \item p\'{e}ri: why
  \item pethi: how many/how much
\end{itemize}

They are followed by the question particle \textit{a} if a question is implied, which is most of the time. In the case of \textit{pethi} (how many) the question particle \textit{a} comes after the specified object (how many/much of something).

If a question is not implied they are not followed by the particle \textit{a}. The words \textit{p\'{e}m\'{a}i} (where) and \textit{ponch} (when) can be used narratively, to tell a story. In that case they don't indicate a question and they are not followed by the particle \textit{a}.

\begin{enumerate}
  \item Suvron: A ghn\'{\i}a ti p\'{\i} a hesi mi?
  (Suvron: Do you know who I am?)
  \item Marthal: N\'{e} ghn\'{\i}a mi. C\'{o}\'{e}th, n\'{e} ghn\'{\i}a mi p\'{e} shulingen a hesi s\'{e}.
  (Marthal: I don't know. Also, I don't know what dance that is.)
  \item Suvron: A ghn\'{\i}a ti p\'{e} gaman a shuling?
  (Suvron: Do you know how to dance?)
  \item Marthal: N\'{e} ghn\'{\i}a mi. N\'{e} ghn\'{\i}a mi diaman p\'{e}m\'{a}i a h\'{a}dha m\'{o} dh\'{a}lam, ach n\'{e} ghn\'{\i}a mi diaman ponch a wantha m\'{o} dh\'{a}thr\'{a}ieth.
  (Marthal: I don't. I never know where to put my hands, and I never know when to move my feet.)
  \item Suvron: A ghn\'{\i}a ti p\'{e}ri a chw\'{e}la mi o sulinga ti canim?</li>
  (Suvron: Do you know why I want you to dance with me?)
  \item Marthal: N\'{e} ghn\'{\i}a mi. A chw\'{e}la ti gn\'{\i} pethi b\'{a}n\'{e} u guru a h\'{\i}va mi pap d\'{\i}?
  (Marthal: I don't. Do you want to know how many glasses of beer I drink every day?)
\end{enumerate}

\subsection{Excercises}

\subsubsection{Vocabulary}

\begin{table}[H]
\centering
\begin{tabular}{cc}
  \toprule
  \textbf{Gal\'{a}thach} & \textbf{English}\\
  \cmidrule(lr){1-1}\cmidrule(lr){2-2}
  mer & crazy\\
  tech & beautiful\\
  dal & blind\\
  galv & fat\\
  c\'{a}ra & to love\\
  mesc & drunk\\
  duchw\'{\i}s & stupid\\
  cl\'{u}i & to hear\\
  sp\'{a} & to say\\
  tiern & boss\\
  dich\'{e}ni & to show\\
  gw\'{\i}n & wine\\
  \'{\i}vi & to drink\\
  s\'{o}ni & to sleep\\
  coimi & to fall\\
  can & to sing\\
  s\'{u}el & sun\\
  \'{\i}thi & to go down\\
  \bottomrule
\end{tabular}
\label{vocab_exercise_lesson13}
\end{table}

\subsubsection{Translate}

Translate the following sentences.

\begin{table}[H]
\centering
\begin{tabular}{|c|M{5.0cm}|}
  \toprule
  \textbf{English} & \textbf{Gal\'{a}thach}\\
  \toprule
  I want to know if you are crazy & \\
  \midrule
  Are you (fem.) beautiful or am I blind & \\
  \midrule
  I am fat and I love it & \\
  \midrule
  I am drunk but you are stupid & \\
  \midrule
  I can't hear what you say & \\
  \midrule
  I want to know who is the boss & \\
  \midrule
  I don't understand what you want & \\
  \midrule
  Can you show me how to do it & \\
  \midrule
  I can't believe how much wine you drink & \\
  \midrule
  I don't know why you do this & \\
  \midrule
  I want to sleep where I fall & \\
  \midrule
  I sing when the sun goes down & \\
  \bottomrule
\end{tabular}
\label{exercise_conjunction_and_question_clauses}
\caption{Exercise: conjunction- and question clauses}
\end{table}

\newpage
\subsubsection{Solution}

\begin{table}[H]
\centering
\rotatebox{180}{%
  \begin{tabular}{|c|>{\itshape}c|}
  \toprule
  \textbf{English} & \textbf{Gal\'{a}thach}\\
  \toprule
  I want to know if you are crazy & Gw\'{e}la mi gn\'{\i} ma hesi ti mer.\\
  \midrule
  Are you (fem.) beautiful or am I blind & A hesi ti dech gw\'{e} a hesi mi dal?\\
  \midrule
  I am fat and I love it & Esi mi galv ach c\'{a}ra mi ch\'{\i}.\\
  \midrule
  I am drunk but you are stupid & Esi mi mesc \'{e}ithr esi ti duchw\'{\i}s.\\
  \midrule
  I can't hear what you say & N\'{e} gh\'{a}la mi cl\'{u}i s\'{e} o sp\'{a} ti.\\
  \midrule
  I want to know who is the boss & Gw\'{e}la mi gn\'{\i} p\'{\i} a hesi in tiern\\
  \midrule
  I don't understand what you want & N\'{e} chw\'{\i}dha mi p\'{e} a chw\'{e}la ti.\\
  \midrule
  Can you show me how to do it & A gh\'{a}la ti dich\'{e}ni adhim p\'{e} gaman a h\'{a}v\'{o} ich\'{\i}?\\
  \midrule
  I can't believe how much wine you drink & N\'{e} gh\'{a}la mi cr\'{e}dhi pethi chwin a h\'{\i}va ti.\\
  \midrule
  I don't know why you do this & N\'{e} ghn\'{\i}a mi p\'{e}ri a h\'{a}va ti sin.\\
  \midrule
  I want to sleep where I fall & Gw\'{e}la mi s\'{o}ni p\'{e}m\'{a}i c\'{o}ima mi.\\
  \midrule
  I sing when the sun goes down & C\'{a}na mi ponch \'{\i}tha in s\'{u}el.\\
  \bottomrule
  \end{tabular}
}
\label{solution_conjunction_and_question_clauses}
\caption{Solution: conjuction- and question clauses}
\end{table}
