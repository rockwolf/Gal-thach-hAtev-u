\section{Menghavan 4: Alghnas An\'{u}\'{e}}
(\textit{Lesson 4: Gender Of Nouns})\\

In the fourth lesson you will learn how to determine the gender of nouns.

\subsection{Gwepchoprith: Gender of nouns}
\subsubsection{Gender of nouns}

In modern Gaulish all nouns have a gender, which is either masculine or feminine. If the meaning of the noun indicates a gender, then that noun is that gender:
\begin{table}[H]
\centering
\begin{tabular}{ccc}
  \toprule
  \textbf{Gal\'{a}thach} & \textbf{English} & \textbf{Gender}\\
  \cmidrule(lr){1-1}\cmidrule(lr){2-2}\cmidrule(lr){3-3}
  gwir & man & masculine\\
  ben & woman & feminine\\
  mapath & boy & masculine\\
  geneth & girl & feminine\\
  map & son & masculine\\
  d\'{u}ithir & daughter & feminine\\
  \'{a}ther & father & masculine\\
  m\'{a}thir & mother & feminine\\
  moth & penis & masculine\\
  tuthu & vagina & feminine\\
  \bottomrule
\end{tabular}
\caption{Example: meaning of gender}
\label{example_gender_meaning}
\end{table}

If the meaning of a noun does not indicate a gender, its gender is determined by the last vowel. If the last vowel is /a/ or /i/ the noun is \textbf{feminine}.
\begin{table}[H]
\centering
\begin{tabular}{ccc}
  \toprule
  \textbf{Gal\'{a}thach} & \textbf{English} & \textbf{Gender}\\
  \cmidrule(lr){1-1}\cmidrule(lr){2-2}\cmidrule(lr){3-3}
  lam & hand & feminine\\
  bis & finger & feminine\\
  t\'{\i}r & land & feminine\\
  cnam & bone & feminine\\
  \bottomrule
\end{tabular}
\caption{Example: gender, last vowel is \textit{a} or \textit{i}}
\label{example_gender_last_vowel_ai}
\end{table}

If the last vowel is /e/, /o/ or /u/ the noun is \textbf{masculine}.
\begin{table}[H]
\centering
\begin{tabular}{ccc}
  \toprule
  \textbf{Gal\'{a}thach} & \textbf{English} & \textbf{Gender}\\
  \cmidrule(lr){1-1}\cmidrule(lr){2-2}\cmidrule(lr){3-3}
  pen & head & masculine\\
  m\'{o}r & sea & masculine\\
  c\'{a}nu & song & masculine\\
  tr\'{a}ieth & foot & masculine\\
  coch & leg & masculine\\
  curu & beer & masculine\\
  \bottomrule
\end{tabular}
\caption{Example: gender, last vowel is \textit{e}, \textit{o} or \textit{u}}
\label{example_gender_last_vowel_eou}
\end{table}

The -i in the diphthongs \textit{\'{a}i}, \textit{\'{e}i}, \textit{\'{o}i} and \textit{\'{u}i} is not a vowel, it is a semi-consonant, like /y/ in English. It does not count as a vowel, and its presence does not make a nouns gender feminine:
\begin{table}[H]
\centering
\begin{tabular}{cccc}
  \toprule
  \textbf{Gal\'{a}thach} & \textbf{English} & & \textbf{Gender}\\
  \cmidrule(lr){1-1}\cmidrule(lr){2-2}\cmidrule(lr){4-4}
  br\'{o}i & country & last vowel is /o/, the -i is the semi-consonant & masculine\\
  m\'{a}i & place, plain & last vowel is /a/ & feminine\\
  t\'{e}i & house & last vowel is /e/ & masculine\\
  gw\'{o}l\'{u}ith & strain & last vowel is /u/ & masculine\\
  \bottomrule
\end{tabular}
\caption{Example: gender, last vowel is a semi-consonant}
\label{example_gender_last_vowel_semi_consonant}
\end{table}

Some nouns end in a \textit{double consonant} where the last consonant is \textit{l}, \textit{n} or \textit{r}. When pronounced there is a dull indistinct sound between the second last consonant and the l, n or r. This sound is called \textit{schwa}, and is represented by the symbol $[$\textschwa$]$. It is not considered a vowel and is not written. It does not affect the gender of a noun. The gender of such a noun is determined by the last vowel before the schwa:
\begin{table}[H]
\centering
\begin{tabular}{cccc}
  \toprule
  \textbf{Gal\'{a}thach} & \textbf{English} & & \textbf{Gender}\\
  \cmidrule(lr){1-1}\cmidrule(lr){2-2}\cmidrule(lr){4-4}
  s\'{e}dhl & seat & last vowel is /e/ & masculine\\
  sparn & thorn & last vowel is /a/ & feminine\\
  livr & book & last vowel is /i/ & feminine\\
  \bottomrule
\end{tabular}
\caption{Example: gender, when ending in double consonant}
\label{example_gender_when_ending_in_double_consonant}
\end{table}

Some nouns end in a \textit{diphthong followed by a double consonant} where the last consonant is \textit{l}, \textit{n} or \textit{r}. The gender of these nouns is determined by the last vowel before the -i of the diphthong:
\begin{table}[H]
\centering
\begin{tabular}{cccc}
  \toprule
  \textbf{Gal\'{a}thach} & \textbf{English} & & \textbf{Gender}\\
  \cmidrule(lr){1-1}\cmidrule(lr){2-2}\cmidrule(lr){4-4}
  an\'{e}ithl & protection & last vowel is /e/ & masculine\\
  l\'{u}ithr & struggle & last vowel is /u/ & masculine\\
  b\'{o}ithl & hit & last vowel is /o/ & masculine\\
  am\'{a}ithl & service & last vowel is /a/ & feminine\\
  \bottomrule
\end{tabular}
\caption{Example: gender, when ending in a diphthong + double consonant}
\label{example_gender_when_ending_in_diphthong_double_consonant}
\end{table}

Nouns of animals are masculine by default, even if the vowels are /a/ or /i/:
\begin{table}[H]
\centering
\begin{tabular}{ccc}
  \toprule
  \textbf{Gal\'{a}thach} & \textbf{English} & \textbf{Gender}\\
  \cmidrule(lr){1-1}\cmidrule(lr){2-2}\cmidrule(lr){3-3}
  garan & heron & masculine\\
  cun & dog & masculine\\
  l\'{o}ern & fox & masculine\\
  \'{e}p & horse & masculine\\
  caval & $[$draught$]$ horse & masculine\\
  b\'{o} & cow $[$generic name for cattle$]$ & masculine\\
  \bottomrule
\end{tabular}
\caption{Example: gender, animals}
\label{example_gender_animals}
\end{table}

These nouns can be made feminine by adding the suffix \textit{-is}:
\begin{table}[H]
\centering
\begin{tabular}{ccc}
  \toprule
  \textbf{Gal\'{a}thach} & \textbf{English} & \textbf{Gender}\\
  \cmidrule(lr){1-1}\cmidrule(lr){2-2}\cmidrule(lr){3-3}
  garanis & female heron & feminine\\
  cunis & bitch & feminine\\
  l\'{o}ernis & vixen & feminine\\
  \'{e}pis & mare (also casich) & feminine\\
  cavalis & female draught horse & feminine\\
  \bottomrule
\end{tabular}
\caption{Example: gender, femalize animals}
\label{example_gender_animals_femalize}
\end{table}

Nouns indicating human functions or activities are also masculine by default:
\begin{table}[H]
\centering
\begin{tabular}{ccc}
  \toprule
  \textbf{Gal\'{a}thach} & \textbf{English} & \textbf{Gender}\\
  \cmidrule(lr){1-1}\cmidrule(lr){2-2}\cmidrule(lr){3-3}
  dr\'{u}idh & scholar & masculine\\
  gwerchovreth & magistrate & masculine\\
  tiern & boss, chief & masculine\\
  dan & official, manager & masculine\\
  \bottomrule
\end{tabular}
\caption{Example: gender, human functions}
\label{example_gender_human_functions}
\end{table}

These nouns can also be made feminine by adding the suffix \textit{-is}:
\begin{table}[H]
\centering
\begin{tabular}{ccc}
  \toprule
  \textbf{Gal\'{a}thach} & \textbf{English} & \textbf{Gender}\\
  \cmidrule(lr){1-1}\cmidrule(lr){2-2}\cmidrule(lr){3-3}
  dr\'{u}idhis & female scholar & feminine\\
  gwerchovrethis & female magistrate & feminine\\
  tiernis & female boss, chief & feminine\\
  danis & female official, manager & feminine\\
  \bottomrule
\end{tabular}
\caption{Example: gender, femalize human functions}
\label{example_gender_human_functions_femalize}
\end{table}

\newpage
\subsection{Exercises: Gender of nouns}

\subsubsection{Determine gender}

Determine the gender of the following nouns:
\begin{table}[H]
\centering
\resizebox{36pc}{!}{%
\begin{tabular}{|c|c|M{10.0cm}|}
  \toprule
  \textbf{Gal\'{a}thach} & \textbf{English} & \textbf{Answer (m/f)}\\
  \toprule
  car & car & \\
  \midrule
  sesa & chair & \\
  \midrule
  roth & wheel & \\
  \midrule
  ar\'{\i}this & table & \\
  \midrule
  dulu & paper & \\
  \midrule
  cumlath & plate & \\
  \midrule
  cladhal & knife & \\
  \midrule
  gaval & fork & \\
  \midrule
  b\'{o}th\'{e}i & stable & \\
  \midrule
  bochw\'{\i}dhu & spoon & \\
  \midrule
  ethn & bird & \\
  \midrule
  t\'{a}ru & bull & \\
  \midrule
  am\'{a}iath & servant & \\
  \midrule
  cerdh\'{\i}ath & worker & \\
  \midrule
  menrodhiath & teacher & \\
  \midrule
  gnis\'{a}iath & student & \\
  \midrule
  pethl\'{o}i & stuff & \\
  \midrule
  pren & tree & \\
  \midrule
  bil & tree trunk & \\
  \midrule
  cl\'{e}tha & ladder & \\
  \midrule
  cilurn & bucket & \\
  \midrule
  scothir & shovel & \\
  \midrule
  cerdhl & work & \\
  \midrule
  tr\'{a}ith & beach & \\
  \midrule
  cr\'{o}su & wave & \\
  \midrule
  sir & star & \\
  \midrule
  nem & sky & \\
  \midrule
  br\'{\i} & hill & \\
  \midrule
  br\'{o}i & country & \\
  \midrule
  b\'{e}l\'{o}i & culture & \\
  \midrule
  tengu & language & \\
  \midrule
  tarinch & nail (fastening implement) & \\
  \midrule
  cingeth & warrior & \\
  \midrule
  delgheth & holder & \\
  \midrule
  druthas & courage & \\
  \midrule
  dumnas & darkness & \\
  \'{e}chal & hoof & \\
  \bottomrule
\end{tabular}
}
\label{exercise_gender}
\caption{Exercise: gender}
\end{table}

\newpage
\subsubsection{Solution}

\begin{table}[H]
\centering
\resizebox{24pc}{!}{%
  \rotatebox{180}{%
    \begin{tabular}{|c|c|>{\itshape}c|}
      \toprule
      \textbf{Gal\'{a}thach} & \textbf{English} & \textbf{Answer (m/f)}\\
      \toprule
      car & car & f\\
      \midrule
      sesa & chair & f\\
      \midrule
      roth & wheel & m\\
      \midrule
      ar\'{\i}this & table & f\\
      \midrule
      dulu & paper & m\\
      \midrule
      cumlath & plate & f\\
      \midrule
      cladhal & knife & f\\
      \midrule
      gaval & fork & f\\
      \midrule
      b\'{o}th\'{e}i & stable & m\\
      \midrule
      bochw\'{\i}dhu & spoon & m\\
      \midrule
      ethn & bird & m\\
      \midrule
      t\'{a}ru & bull & m\\
      \midrule
      am\'{a}iath & servant & m\\
      \midrule
      cerdh\'{\i}ath & worker & m\\
      \midrule
      menrodhiath & teacher & m\\
      \midrule
      gnis\'{a}iath & student & m\\
      \midrule
      pethl\'{o}i & stuff & m\\
      \midrule
      pren & tree & m\\
      \midrule
      bil & tree trunk & f\\
      \midrule
      cl\'{e}tha & ladder & f\\
      \midrule
      cilurn & bucket & m\\
      \midrule
      scothir & shovel & f\\
      \midrule
      cerdhl & work & m\\
      \midrule
      tr\'{a}ith & beach & f\\
      \midrule
      cr\'{o}su & wave & m\\
      \midrule
      sir & star & f\\
      \midrule
      nem & sky & m\\
      \midrule
      br\'{\i} & hill & f\\
      \midrule
      br\'{o}i & country & m\\
      \midrule
      b\'{e}l\'{o}i & culture & m\\
      \midrule
      tengu & language & m\\
      \midrule
      tarinch & nail (fastening implement) & f\\
      \midrule
      cingeth & warrior & m\\
      \midrule
      delgheth & holder & m\\
      \midrule
      druthas & courage & f\\
      \midrule
      dumnas & darkness & f\\
      \'{e}chal & hoof & f\\
      \bottomrule
    \end{tabular}
  }
}
\label{solution_gender}
\caption{Solution: gender}
\end{table}
